\chapter{Simulation}
\label{sec:simulation}
In this chapter I discusses how particle physicists simulate high energy particle physics
interactions. Simulated interactions, based on knowledge of the Standard Model, provide 
the foundation of our predictions in high energy particle physics. 
The wealth of particle physics knowledge accumulated over the last 
60 years provides has provided a testing grounds to compare experimental results with
simulated particle interactions. Based on extensive work from the simulation community,
we are able to simulate and model the the proton-proton collisions which take place in the
CMS detector with a high degree of precision. Knowledge of what processes should take place
in our detector allows us to construct a prediction of what we will find. In the following
analyses, we are specifically interested in comparing two physics scenarios: the Standard Model 
without the existance of a 125 \GeV Higgs boson versus the Standard Model with the existance
of a 125 \GeV Higgs boson. Both of these predictions rely on the Standard Model processing and
Higgs boson processes be well modeled.
In this chapter, I specifically focus on how events are simulated for proton-proton collisions,
how the initial products of the collision decay, and how the decay products are modeled
to interact with a simulation of the CMS detector.


\section{Hard Scattering Process}
Monte Carlo Generator Programs
FSR ISR

\subsection{Parton Distribution Functions}
One of the unique complexities present at a hadron collider which is avoided with a
lepton collider is accounting for the substructure of the colliding hadrons. For proper simulations
of LHC proton-proton collisions we must account for this. The internal structure of the proton
has been probed over the past half century using deep inelastic scattering~\cite{Breidenbach:1969kd, PhysRevLett.23.930}.
Experimental results showed the internal structure of protons revealing the existance of
quarks and gluons. The internal structure of the colliding protons at the LHC are taken into
account using what are called Parton Distribution Function (PDFs). PDFs give the probability
density for finding a particle a parton with a certain longitudinal momentum fraction $x$ at a given
energy scale. The momentum fraction $x$ is the the parton's fractional momentum with respect
to the hadron under consideration, protons for CMS related simulations. 
%Figure~\ref{fig:sim_pdf} 
%shows example PDFs for two given energy scales. 
The specific PDFs which are used in the
following analyses are provided by \texttt{NNPDF3.0} with the exact PDF set being 
\texttt{NNPDF30\_nlo\_as\_0118}~\cite{Ball:2014uwa, Ball:2011uy}. The \texttt{NNPDF3.0} PDFs
use a global dataset including, but not limited to, data from HERA, ZEUS, ATLAS, LHCb, and CMS.
Functional forms derived from theoretical QCD predictions with electroweak corrections are fit
to the available data resulting in the provided PDFs used by CMS for event simulation~\cite{Ball:2014uwa}.



\section{Parton Showers}
    Shower evolution is viewed as a probabilistic process which occures with unit total probability
    the cross section is not directly affected, but indirectly it is via the changed event shape


\section{Underlying Event}
In addition to the particles resulting from the hard scattering portion of a proton-proton collision,
there is what is called the underlying event. The underlying event consists of particles coming from
the hadronization of the partons in the colliding protons which are not the two partons associated the
hard scattering event, this is often called the beam-beam remenants. The underlying event also
consists of the particles resulting from multiple-parton interactions. The underlying event in 
event generators predominantly characterize the kinematics and compostion of soft ``jets''.
There are specific ``jet'' related physics observabes which are sensitive to the characteristics of the underlying
event~\cite{Khachatryan:2015pea, Field:cdf2008}. For a description of ``jets'' see Section~\ref{sec:obj_reco_jets}.
The initial underlying event tune for \POWHEG 8 is the Monash Tune. A CMS specific tuning of 
\POWHEG 8 has been constructed using the parameters of the Monash Tune. The tune is based
on data from CDF and 7 \TeV CMS data and is called \texttt{CUETP8M1}. The differences between
\texttt{CUETP8M1} and the Monash Tune related to the treatement of the energy-dependence 
parameters in the fit. Considering \texttt{CUETP8M1} is derived based largely on data from
CMS, there is very good agreement between CMS data and simulations based on the
\texttt{CUETP8M1} tune~\cite{Khachatryan:2015pea}. 
 



Hard process
Resonance Decays

\section{Particle Decays}
Color flow
Hadronization - quark confinement
Ordinary decays
    Tau decays
    B jet

\section{Detector Simulation}
GEANT


Signal and background processes are modeled with samples of simulated events.
The signal samples with a Higgs boson produced through gluon fusion ($\cPg\cPg\PH$), vector boson fusion (VBF),
or in association with a $\PW$ or $\PZ$ boson ($\PW\PH$ or $\PZ\PH$), are generated at next-to-leading order (NLO) in perturbative quantum chromodynamics (pQCD) with the \POWHEG 2.0~\cite{Nason:2004rx,Frixione:2007vw, Alioli:2010xd, Alioli:2010xa, Alioli:2008tz} generator. The \textsc{minlo hvJ}~\cite{Luisoni:2013kna} extension of \POWHEG 2.0 is used for the $\PW\PH$ and $\PZ\PH$ simulated samples. 
The $\ttbar\PH$ process is negligible.
The various production cross sections and branching fractions for the SM Higgs boson production, and their corresponding uncertainties are taken from Refs.~\cite{deFlorian:2016spz,Denner:2011mq,Ball:2011mu} and references therein.

The \aMCATNLO~\cite{Alwall:2014hca} generator is used for $\PZ+$jets and $\PW+$jets processes. They are simulated at leading order (LO) with the MLM jet matching and merging~\cite{Alwall:2007fs}.
The \aMCATNLO generator is also used for diboson production simulated at next-to-LO (NLO) with the FxFx jet matching and merging~\cite{Frederix:2012ps}, whereas \POWHEG 2.0 and 1.0 are used for $\ttbar$ and single top quark production, respectively.
The generators are interfaced with \PYTHIA 8.212 ~\cite{Sjostrand:2014zea} to model the parton showering and fragmentation, as well as the decay of the $\Pgt$ leptons.
The \PYTHIA parameters affecting the description of the underlying event are set to the {CUETP8M1} tune~\cite{Khachatryan:2015pea}.

Generated events are processed through a simulation of the CMS detector based on
GEANTfour~\cite{Agostinelli:2002hh}, and are reconstructed with the same algorithms used for data.
The simulated samples include additional $\Pp\Pp$ interactions per bunch
crossing, referred to as ``pileup''.
The effect of pileup is taken into account by generating concurrent minimum bias collision events generated with \PYTHIA.
The simulated events are weighted such that the distribution of the number of additional pileup interactions, estimated from the measured instantaneous luminosity for each bunch crossing, matches that in data, with an average of approximately 27 interactions per bunch crossing.

