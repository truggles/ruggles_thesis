%%%%%%%%%%%%%%%%%%%%%%%%%%%%%%%%%%%%%%%%%%%%%%%%%%%%%%%%%%%%%%%%%%%%%%%%%%%%%
%%%% Preamble
%%%%%%%%%%%%%%%%%%%%%%%%%%%%%%%%%%%%%%%%%%%%%%%%%%%%%%%%%%%%%%%%%%%%%%%%%%%%%

%%%% The uwthesis.sty file relies on the memoir class!
%%%% You should be using the memoir class anyway; it makes life easier:
%%%% http://www.ctan.org/tex-archive/macros/latex/contrib/memoir/

%\begin{Huge}
%\begin{huge}
%\begin{LARGE}
%\begin{Large}
%\begin{large}
%\begin{normalsize} % (default)
%\begin{small}
%\begin{footnotesize}
%\begin{scriptsize}
%\begin{tiny}

\newcommand{\pp}{\Pp\Pp\xspace}
\newcommand{\PH}{\ensuremath{H}}
\newcommand{\PZ}{\ensuremath{Z}}
\newcommand{\PW}{\ensuremath{W}}
\newcommand{\Pgg}{\ensuremath{\gamma}}
\newcommand{\Pg}{\ensuremath{\text{g}}}
\newcommand{\Pq}{\ensuremath{\text{q}}}
\newcommand{\cPg}{\ensuremath{g}}
\newcommand{\Pgt}{\ensuremath{\tau}}
\newcommand{\tauh}{\ensuremath{\tau_{h}}}
\newcommand{\Pgm}{\ensuremath{\mu}}
\newcommand{\Pe}{\ensuremath{e}}
\newcommand{\PGpz}{\ensuremath{\pi^{0}}}
\newcommand{\Pp}{\ensuremath{p}}
\newcommand{\pt}{\ensuremath{p_\mathrm{T}}\xspace}
\newcommand{\kt}{\ensuremath{k_\mathrm{T}}\xspace}
\newcommand{\PT}{\ensuremath{P_\mathrm{T}}\xspace}
\newcommand{\dr}{\ensuremath{\Delta \mathrm{R}}\xspace}
\newcommand{\TeV}{\ensuremath{\mathrm{TeV}}\xspace}
\newcommand{\GeV}{\ensuremath{\mathrm{GeV}}\xspace}
\newcommand{\MeV}{\ensuremath{\mathrm{MeV}}\xspace}
\newcommand{\cmsq}{\ensuremath{\mathrm{cm}^{2}}\xspace}
\newcommand{\etaphi}{\ensuremath{\mathrm{(}\eta, \phi\mathrm{)}}\xspace}
\newcommand{\NA}{\textsc{N/A}}
\newcommand{\stat}{\ensuremath{(stat.)}}
\newcommand{\syst}{\ensuremath{(syst.)}}
\newcommand{\thy}{\ensuremath{(theory)}}
\newcommand{\ttbar}{\ensuremath{t\bar{t}}\xspace}
\newcommand{\bbbar}{\ensuremath{b\bar{b}}\xspace}
\newcommand{\fbinv}{\ensuremath{fb^{-1}}\xspace}
\newcommand{\MT}{\ensuremath{m_\mathrm{T}}}
\newcommand{\mvis}{\ensuremath{m_\text{vis}}}
\newcommand{\mtt}{\ensuremath{m_{\Pgt\Pgt}}}
\newcommand{\hww}{\ensuremath{\PH\to\PW\PW}}
\newcommand{\hzz}{\ensuremath{\PH\to\PZ\PZ}}
\newcommand{\htt}{\ensuremath{\PH\to\Pgt\Pgt}}
\newcommand{\pth}{\ensuremath{\pt^{\tau\tau}}}
\newcommand{\mjj}{\ensuremath{m_\mathrm{jj}}}
\newcommand{\MGAMCNLO}{\textsc{MadGraph5}\_\textsc{aMC@NLO}\xspace}
\newcommand{\POWHEG}{\textsc{Powheg}\xspace}
\newcommand{\PYTHIA}{\textsc{Pythia}\xspace}
\newcommand{\HERWIG}{\textsc{Herwig}\xspace}
\newcommand{\FASTJET}{\textsc{Fastjet}\xspace}
\newcommand{\emu}{\ensuremath{\Pe\Pgm}}
\newcommand{\etvecmiss}{\ensuremath{\vec{E}^{miss}_T}\xspace}
\newcommand{\MET}{\ensuremath{\etvecmiss}}
\newlength\cmsTabSkip
\setlength\cmsTabSkip{1.6ex}
\newcommand{\ptmiss}{\ensuremath{\pt^\text{miss}}}
\newcommand{\ptvec}{\ensuremath{\vec{\pt}}}
\providecommand{\mH}{\ensuremath{m_{\PH}}}

\documentclass[oneside, letterpaper, 12pt, oldfontcommands]{memoir}

\setsecnumdepth{subsection}

%%%% Import uwthesis.sty to get official formatting, then set your variables.
\usepackage{uwthesis}
\usepackage{mathtools}
\usepackage{xspace}
\usepackage{lineno}
\usepackage{outlines} % For nested lists with bullets
\linenumbers
%\usepackage{amsmath}
\DeclarePairedDelimiter{\abs}{\lvert}{\rvert}

\settitle{A Study of the Standard Model Higgs Boson Decaying to a Pair of Tau Leptons with the CMS Detector at the LHC}
\setauthor{Tyler Ruggles}
\setdepartment{Physics}
\doctors % or \masters
\setgraddate{2018}
\setdefensedate{10 May 2018} % or whatever format you want

%%%% Members of the Final Oral Committee (FOC)
%%%% Give name, rank, and department
%%%% 
\setfoca{Sridhara Dasu}{Professor}{Physics} % <- Your advisor
\setfocb{Wesley H. Smith}{Professor}{Physics}
\setfocc{Matthew F. Herndon}{Professor}{Physics}
\setfocd{Aki Hashimoto}{Professor}{Physics}
\setfoce{Marshall F. Onellion}{Professor}{Physics}


%%%% Your abstract, used for the UMI abstract and in your front matter
\setabstract{%
  An analysis of the Standard Model Higgs boson observed decaying to 
  tau leptons. The analysis utilizes the full 2016 data set collected
  by the CMS experiment.
}

%%%%%%%%%%%%%%%%%%%%%%%%%%%%%%%%%%%%%%%%%%%%%%%%%%%%%%%%%%%%%%%%%%%%%%%%%%%%%
%%%% Document
%%%%%%%%%%%%%%%%%%%%%%%%%%%%%%%%%%%%%%%%%%%%%%%%%%%%%%%%%%%%%%%%%%%%%%%%%%%%%

\begin{document}

% Tell the memoir class to set up lowercase roman for pagination, etc.
\frontmatter

%%%% Uncomment this to create a UMI abstract page.
%%%% If you are submitting electronically, however, this page is unnecessary.
% \theumiabstract

% The title page
\thetitlepage
\clearpage

% The copyright page, if you want to pay the fee and register copyright.
\thecopyrightpage
\cleardoublepage

% These above pages should not be counted, so we reset the counter to 1.
\setcounter{page}{1}

% An abstract may be required by your department.
\section{Abstract}
\uwabstract
\cleardoublepage

% Acknowledgements go here if you want to include them.
\section{Acknowledgements}
This is where any acknowledgements would go.
\clearpage

% Table of contents
\maxtocdepth{subsection}
\tableofcontents* % the * means that there isn't an entry for the TOC itself
% \clearpage
% \listoffigures  % if you have any figures
% \clearpage
% \listoftables   % if you have any tables

% Tell the memoir class to set up normal pagination, etc. for the main doc
\mainmatter

\chapter{Introduction}

In the standard model (SM) of particle physics~\cite{Glashow:1961tr,SM1,SM3},
electroweak symmetry breaking is achieved via the Brout--Englert--Higgs
mechanism~\cite{Englert:1964et,Higgs:1964ia,Higgs:1964pj,Guralnik:1964eu,Higgs:1966ev,Kibble:1967sv},
leading, in its minimal version, to the prediction of the existence of one physical neutral scalar particle,
commonly known as the Higgs boson ($\PH$).
A particle compatible with such a boson was observed by the ATLAS and CMS experiments at the CERN LHC
in the $\PZ\PZ$, $\Pgg \Pgg$, and $\PW\PW$ decay channels~\cite{Aad:2012tfa, Chatrchyan:2012xdj, Chatrchyan:2013lba},
during the proton-proton ($\Pp\Pp$) data taking period in 2011 and 2012
at center-of-mass energies of $\sqrt{s} = 7$ and 8\TeV, respectively.
Subsequent results from both experiments, described in
Refs.~\cite{Aad:2015gba, Khachatryan:2014jba, Chatrchyan:2012jja, Aad:2013xqa, Khachatryan:2014kca,Sirunyan:2017exp},
established that the measured properties of the new particle,
including its spin, CP properties,
and coupling strengths to SM particles, are consistent with those expected for the Higgs boson predicted by the SM.
The mass of the Higgs boson has been determined to be
$125.09\pm0.21\stat\pm0.11\syst\GeV$, from a combination of
ATLAS and CMS measurements~\cite{Aad:2015zhl}.

To establish the mass generation mechanism for fermions,
 it is necessary to probe the direct coupling of
the Higgs boson to such particles.
The most promising decay channel is $\Pgt^+\Pgt^-$,
because of the large event rate expected in the SM compared to the $\Pgm^+\Pgm^-$ decay channel ($\mathcal{B}(\PH\to\Pgt^+\Pgt^-)=6.3$\% for a mass of 125.09\GeV), and of the smaller contribution from background events
with respect to the $\bbbar$ decay channel.

Searches for a Higgs boson decaying to a $\Pgt$ lepton pair were performed at the LEP~\cite{Barate:2000ts,Abdallah:2003ip,Achard:2001pj,Abbiendi:2000ac},
Tevatron~\cite{Aaltonen:2012jh, Abazov:2012zj}, and LHC colliders.
Using $\Pp\Pp$ collision data at $\sqrt{s}=7$ and $8\TeV$, the CMS Collaboration showed evidence for this process with an observed\,(expected)
significance of 3.2\,(3.7) standard deviations (s.d.)~\cite{Chatrchyan:2014nva}. The ATLAS
experiment reported evidence for Higgs bosons decaying into pairs
of $\Pgt$ leptons with an observed (expected) significance of 4.5 (3.4)
s.d. for a Higgs boson mass of 125\GeV~\cite{Aad:2015vsa}.
The combination of the results from both experiments yields an observed (expected)
significance of 5.5\,(5.0) s.d.~\cite{Khachatryan:2016vau}.

This Letter reports on a measurement of the $\PH\to\Pgt\Pgt$ signal strength.
The analysis targets both the gluon fusion and the vector boson fusion production mechanisms.
The analyzed data set corresponds to an integrated luminosity of 35.9\fbinv, and was collected in 2016 in $\Pp\Pp$ collisions at a center-of-mass energy of
13\TeV.
In the following, the symbol $\ell$ refers to electrons or muons,
the symbol $\tauh$ refers to $\Pgt$ leptons reconstructed in their hadronic decays, and
$\PH\to\Pgt^+\Pgt^-$  and $\PH\to\PW^+\PW^-$ are simply denoted as $\PH\to\Pgt\Pgt$  and $\PH\to\PW\PW$, respectively.
All possible $\Pgt\Pgt$ final states are studied, except for those with two muons or two electrons because of the low branching fraction and large background contribution. The analysis covers about 94\% of all possible $\Pgt\Pgt$ final states.

\section{Theory Context: Standard Model of Particle Physics}

\subsection{Electroweak Symmetry Breaking}

\subsubsection{W/Z Higgs Associated Production}

\subsection{Cross Sections and Decay Rates}

\subsection{QCD and Proton Structure}

\section{Experimental Context: Previous Higgs Measurements}

%\subsection{D0 and CDF}

\subsection{CMS and ATLAS at 7 and 8 TeV}






%\chapter{Phenomology of Processes}
\label{sec:pheno}

Dasu -
The pheno chapter need not start from the Dirac equation and build up. It should have crisp intro to Higgs phenomenology starting from that portion of the Lagrangian. You don’t need all the myriad details of SM like the quark mixing matrices, etc.

\section{Higgs Yukawa Couplings}

\section{Higgs Production}

\subsection{Gluon Fusion}

\subsection{Vector Boson Fusion}

\subsection{Associated Production}

\section{Higgs Decays}

\subsection{Higgs to $\tau\tau$ Decay Process}

CROSS SECTIONS
The various production cross sections and branching fractions for the SM Higgs 
boson production, and their corresponding uncertainties are taken from 
References.~\cite{deFlorian:2016spz,Denner:2011mq,Ball:2011mu} and references therein.



\subsection{Electroweak Symmetry Breaking}

electroweak symmetry breaking is achieved via the Brout--Englert--Higgs
mechanism~\cite{Englert:1964et,Higgs:1964ia,Higgs:1964pj,Guralnik:1964eu,Higgs:1966ev,Kibble:1967sv},
leading, in its minimal version, to the prediction of the existence of one physical neutral scalar particle,
commonly known as the Higgs boson ($\PH$).



\subsubsection{W/Z Higgs Associated Production}

\subsection{Cross Sections and Decay Rates}

\subsection{QCD and Proton Structure}



To establish the mass generation mechanism for fermions,
 it is necessary to probe the direct coupling of
the Higgs boson to such particles.
The most promising decay channel is $\Pgt^+\Pgt^-$,
because of the large event rate expected in the SM compared to the $\Pgm^+\Pgm^-$ decay channel ($\mathcal{B}(\PH\to\Pgt^+\Pgt^-)=6.3$\% for a mass of 125.09\GeV), and of the smaller contribution from background events
with respect to the $\bbbar$ decay channel.


%\chapter{The CMS Experiment and the CERN LHC}

\section{The LHC}

\subsection{LHC Pre-Acceleration}

\subsection{LHC Acceleration}

\section{The CMS Experiment}

The central feature of the CMS apparatus is a superconducting solenoid of 6 meters internal diameter, providing a magnetic field of 3.8 Tesla. Within the solenoid volume, there are a silicon pixel and strip tracker, a lead tungstate crystal electromagnetic calorimeter (ECAL), and a brass and scintillator hadron calorimeter (HCAL), each composed of a barrel and two endcap sections. Forward calorimeters extend the pseudorapidity coverage provided by the barrel and endcap detectors. Muons are detected in gas-ionization chambers embedded in the steel flux-return yoke outside the solenoid.

Events of interest are selected using a two-tiered trigger system~\cite{Khachatryan:2016bia}. The first level (L1), composed of custom hardware processors, uses information from the calorimeters and muon detectors to select events at a rate of around 100 kHz within a time interval of less than 4 microseconds. The second level, known as the high-level trigger (HLT), consists of a farm of processors running a version of the full event reconstruction software optimized for fast processing, and reduces the event rate to about 1 kHz before data storage.

Significant upgrades of the L1 trigger during the first long shutdown of the LHC have benefitted this analysis, especially in the $\tauh\tauh$ channel. These upgrades improved the $\tauh$ identification at L1 by giving more flexibility to object isolation, allowing new techniques to suppress the contribution from additional $\Pp\Pp$ interactions per bunch
crossing, and to reconstruct the L1 $\tauh$ object in a fiducial region that matches more closely that of a true hadronic $\Pgt$ decay. The flexibility is achieved by employing high bandwidth optical links for data communication and large field-programmable gate arrays (FPGAs) for data processing.

A more detailed description of the CMS detector, together with a definition of the coordinate system used and the relevant kinematic variables, can be found in Ref.~\cite{Chatrchyan:2008zzk}.

\subsection{Geometry}

\subsection{Magnet}
\subsection{Inner Tracking System}
\subsubsection{Pixels}
\subsubsection{Strips}
\subsection{Electromagnetic Calorimeter}
\subsection{Hadronic Calorimeter}
\subsection{Muon System}
\subsubsection{Drift Tubes}
\subsubsection{Cathode Strip Chambers}
\subsubsection{Resistive Plate Chambers}
\subsection{Trigger and Data Acquisition}
\subsubsection{Level-1 Trigger}
\subsubsection{Aside for CaloL1 Duties and Online SW}
\subsubsection{HLT}
\subsubsection{Aside for 2018 Tau Trigger Updates}
\subsubsection{Aside for Phase-2 L1EG Discussion}

%\chapter{Simulation}
\label{sec:simulation}
In this chapter I discusses how particle physicists simulate high energy particle physics
interactions. Simulated interactions, based on knowledge of the Standard Model, provide 
the foundation of our predictions in high energy particle physics. 
The wealth of particle physics knowledge accumulated over the last 
60 years provides has provided a testing grounds to compare experimental results with
simulated particle interactions. Based on extensive work from the simulation community,
we are able to simulate and model the the proton-proton collisions which take place in the
CMS detector with a high degree of precision. Knowledge of what processes should take place
in our detector allows us to construct a prediction of what we will find. In the following
analyses, we are specifically interested in comparing two physics scenarios: the Standard Model 
without the existance of a 125 \GeV Higgs boson versus the Standard Model with the existance
of a 125 \GeV Higgs boson. Both of these predictions rely on the Standard Model processing and
Higgs boson processes be well modeled.
In this chapter, I specifically focus on how events are simulated for proton-proton collisions,
how the initial products of the collision decay, and how the decay products are modeled
to interact with a simulation of the CMS detector.


\section{Hard Scattering Process}
Monte Carlo Generator Programs
FSR ISR

\subsection{Parton Distribution Functions}
One of the unique complexities present at a hadron collider which is avoided with a
lepton collider is accounting for the substructure of the colliding hadrons. For proper simulations
of LHC proton-proton collisions we must account for this. The internal structure of the proton
has been probed over the past half century using deep inelastic scattering~\cite{Breidenbach:1969kd, PhysRevLett.23.930}.
Experimental results showed the internal structure of protons revealing the existance of
quarks and gluons. The internal structure of the colliding protons at the LHC are taken into
account using what are called Parton Distribution Function (PDFs). PDFs give the probability
density for finding a particle a parton with a certain longitudinal momentum fraction $x$ at a given
energy scale. The momentum fraction $x$ is the the parton's fractional momentum with respect
to the hadron under consideration, protons for CMS related simulations. 
%Figure~\ref{fig:sim_pdf} 
%shows example PDFs for two given energy scales. 

The specific PDFs which are used in the
following analyses are provided by \texttt{NNPDF3.0} with the exact PDF set being 
\texttt{NNPDF30\_nlo\_as\_0118}~\cite{Ball:2014uwa, Ball:2011uy}. The \texttt{NNPDF3.0} PDFs
use a global dataset including, but not limited to, data from HERA, ZEUS, ATLAS, LHCb, and CMS.
Functional forms derived from theoretical QCD predictions with electroweak corrections are fit
to the available data resulting in the provided PDFs used by CMS for event simulation~\cite{Ball:2014uwa}.
Uncertainties from the PDF estimation method are applied in the following analyses to cover
potential mismodeling of the proton PDFs used in the CMS simulations.



\section{Underlying Event}
In addition to the particles resulting from the hard scattering portion of a proton-proton collision,
there is what is called the underlying event. The underlying event consists of particles coming from
the hadronization of the partons in the colliding protons which are not the two partons associated the
hard scattering event, this is often called the beam-beam remenants. The underlying event also
consists of the particles resulting from multiple-parton interactions. The underlying event in 
event generators predominantly characterize the kinematics and compostion of soft ``jets''.
There are specific ``jet'' related physics observabes which are sensitive to the characteristics of the underlying
event~\cite{Khachatryan:2015pea, Field:cdf2008}. For a description of ``jets'' see Section~\ref{sec:obj_reco_jets}.
The initial underlying event tune for \POWHEG 8 is the Monash Tune. A CMS specific tuning of 
\POWHEG 8 has been constructed using the parameters of the Monash Tune. The tune is based
on data from CDF and 7 \TeV CMS data and is called \texttt{CUETP8M1}. The differences between
\texttt{CUETP8M1} and the Monash Tune related to the treatement of the energy-dependence 
parameters in the fit. Considering \texttt{CUETP8M1} is derived based largely on data from
CMS, there is very good agreement between CMS data and simulations based on the
\texttt{CUETP8M1} tune~\cite{Khachatryan:2015pea}. 



\section{Parton Showers}
The generators are interfaced with \PYTHIA 8.212 ~\cite{Sjostrand:2014zea} to model the parton showering and fragmentation, as well as the decay of the $\Pgt$ leptons.
    Shower evolution is viewed as a probabilistic process which occures with unit total probability
    the cross section is not directly affected, but indirectly it is via the changed event shape
    Resonance Decays
Ordinary decays
    Tau decays
    B jet



\section{Hadronization}
The Lund string model used in Pythia8 vs. cluster model
Color flow
Hadronization - quark confinement



\section{Pileup}
There where 27 proton-proton collisions per bunch crossing on average
in the 2016 data collected by CMS.
For most bunch crossings all of these proton-proton collisions are soft
scattering events and, with no hard process present, the event it unlikely to be stored for
future physics analysis because it will fail the Level-1 or High Level Trigger selections. 
When there is a hard scattering process present in a bunch crossing,
these soft scattering collisions will still be present and must be modeled. 
To emulate this effect in simulated events, in addition to the two protons involved in the 
hard scattering interaction, additional proton-proton interactions are added. 
These additional proton-proton collisions are
dominated by the same type of soft scattering collisions observed in data and are referred to as ``pileup''.
\textsc{Pythia8} is used to generate these soft scattering collision. 
The distribution of the number of soft scattering events added to a simulated 
sample is intended to align with the distribution observed in the 2016 data.
As alignment is never 100\% perfect immediately after the samples are simulated, the simulated events
are weighted to adjust their distribution to the data based on the measured instantaneous
luminosity for each bunch crossing, see Section~\ref{sec:htt_mc_samples}.



\section{Detector Simulation}
At this stage of event simulation, the simulated events model our best approximation
of the proton-proton collisions and resulting decays and hadronization taking place
in the CMS detector. There is one critical last step to complete the event
simulations. The decay products from the collisions must interact with a simulated
model of the CMS detector. The result of this final stage is simulated events that
are stored in the same data format as the data gathered by the detector itself,
raw energy deposits.

The \textsc{Geant4} software toolkit is used to simulate the CMS 
detector~\cite{Agostinelli:2002hh}. At its most basic, \textsc{Geant4} is a 
toolkit for simulating the passage of particles through matter. It contains a 
vast library of functionality allowing the creation of model physics detectors
made out of any desired material and in any desired geometrical configuration.
To validate the \textsc{Geant4} modeling of the CMS detector, test beam data and
collision data are used. Good agreement is seen between the data and the
simulated particles passing through the CMS detector for the energy response
and resolution for pions and protons~\cite{geant4_cms_2017}.

After the decay products of the simulated events have passed through the \textsc{Geant4}
simulation, they are stored in the same format as data is originally gathered. From
this point forward, data and simulated events are reconstructed with the same 
algorithms.




Signal and background processes are modeled with samples of simulated events.
The signal samples with a Higgs boson produced through gluon fusion ($\cPg\cPg\PH$), vector boson fusion (VBF),
or in association with a $\PW$ or $\PZ$ boson ($\PW\PH$ or $\PZ\PH$), are generated at next-to-leading order (NLO) in perturbative quantum chromodynamics (pQCD) with the \POWHEG 2.0~\cite{Nason:2004rx,Frixione:2007vw, Alioli:2010xd, Alioli:2010xa, Alioli:2008tz} generator. The \textsc{minlo hvJ}~\cite{Luisoni:2013kna} extension of \POWHEG 2.0 is used for the $\PW\PH$ and $\PZ\PH$ simulated samples. 
The $\ttbar\PH$ process is negligible.
The various production cross sections and branching fractions for the SM Higgs boson production, and their corresponding uncertainties are taken from Refs.~\cite{deFlorian:2016spz,Denner:2011mq,Ball:2011mu} and references therein.

The \aMCATNLO~\cite{Alwall:2014hca} generator is used for $\PZ+$jets and $\PW+$jets processes. They are simulated at leading order (LO) with the MLM jet matching and merging~\cite{Alwall:2007fs}.
The \aMCATNLO generator is also used for diboson production simulated at next-to-LO (NLO) with the FxFx jet matching and merging~\cite{Frederix:2012ps}, whereas \POWHEG 2.0 and 1.0 are used for $\ttbar$ and single top quark production, respectively.




%\chapter{Object Reconstruction and Selection}

In this chapter I discuss the progression from detector-based signals through to an
overall event reconstruction within CMS. At CMS, physics-object reconstruction draws
on input from all detector subseystems simultaneously to build particle tracks
and cluster together energy deposits, to link together these tracks and energy
deposits to construct basic physics-objects such as electrons and charged
hadrons, and to build composit objects such as ``jets'' and hadronically decaying
$\tau$ leptons. Event based quantities such as the \etvecmiss are also reconstructed.
To achieve all of this, CMS uses its Particle Flow (PF) reconstruction 
algorithm~\cite{Sirunyan:2017ulk}. The particle flow concept has been used in the 
past by other experiments such as ALEPH at LEP~\cite{PF-ALEPH}. CMS is the first
experiment to fully utilize the particle flow techinque in a hadron collider environment.
A discussion of the reconstruction of the two basic PF objects, tracks and energy
clusters, follows in the next section. Afterwards I detail the construction of 
the PF physics-objects then move to the composit objects and full event variables.
The PF approach allows particles from pileup interactions to be identified 
and enables efficient pileup mitigation methods.

\section{Particle Flow Input}
The CMS detector and the PF reconstruction algorithm were specifically designed to
compliment eachother. The CMS detector features: a highly-segmented tracker well
suited to track reconstruction, a fine-grained electromagnetic calorimeter necessary
to separate the individual energy deposits from particles within ``jets'' and
efficient photon and electron identification, a hermetic hadron 
calorimeter for the measurement and identification of charged and neutral hadrons, 
a strong magnetic field for the measurement of the momenta of charged particles and to
separate the calorimeter energy deposits of charged and neutral hadrons within ``jets'', and 
an excellent muon spectrometer for muon identification and to disintangle the muon
tracks from other tracks in the tracker. A schematic of a slice of the CMS detector
and different physics-objects transversing the detector subsystems can be see in
figure~\ref{fig:cms_slice}. The different detector systems all contribute
necessary pieces of information to the PF reconstruction. From the raw detector
signals two classes of PF objects are created, tracks and energy clusters.

\begin{figure*}[htbp]
\centering
     \includegraphics[width=1.0\textwidth]{object_reconstruction_and_selection/plots/cms_slice.pdf}
     \caption{
A schematic of a slice of an x-y cross section of the CMS detector showing different
physics-objects such as electrons, photons, charged and neutral hadroncs, and muons
propagating outwards from the collision region within the detector. The schematic
shows how tracks are linked to energy deposits and in which subdetector regions different
particles deposit most of their energy on average.
     }
     \label{fig:cms_slice}
\end{figure*}


\subsection{Particle Flow Tracks}
Energy deposits often called ``hits'' are recorded by the pixel and strip tracks during
a collision. From these, charged particle tracks are reconstructed in subsequent layers
mapping the progression of charged particles from the beam axis outwards into the detector
volume. Attempting to reconstruct tracks from every possible hit combination quickly
because unreasonable considering the over 70 million tracker pixels and strips which can
each record a hit. PF uses a combinatorial track finder based on Kalman 
Filtering (KF); the algorithim is broken down into three successive steps.
\begin{itemize}
\item Generate seed tracks from a few hits which are compatible with a charged
particle trajectory
\item Gathering other hits along the seed track trajectory when propagated through
the rest of the tracker subsystem
\item Final track fitting to determine track properties such as the origin, transverse
momentum, and direction.
\end{itemize}
Only tracks meeting certain quality standards are kept for analysis. These tracks must
be seeded with two hits in consecutive layers in the pixel detector, and are required 
to be reconstructed with at least eight tracker hits in total, and with at most one 
missing hit along the track trajectory. Tracks must also have a curvature corresponding
to a momentum greater than 0.9\GeV.

There is a balance that must be struck between imposing tight quality cuts on reconstructed
tracks which increases the purity of genuine track within the reconstructed track collection
but also decreases the efficiency for reconstructing genuine tracks, and loosening
quality cuts to reconstruct genuine tracks with a higher efficiency and lower purity.
After an initial pass through what is called the global combinatorial track finder,
which has stringent track quality criteria impossed by the eight hit requirement, 
the efficiency to reconstruct
genuine tracks is roughly 80\% for charged pions with $\pt = 10\GeV$, and 99\%
for isolated muons. This corresponds to a misreconstructed track rate in the range of
about 2.5\% for charged pions with $\pt = 10\GeV$, see figure~\ref{fig:kf_tracking}.
As hits are recorded in a reconstructed track, they are removed from the available,
unused hits which can be combined in subsequent passes through the tracking algorithm.

There are ten passes through the tracking algorithm in total. Each pass loosens the track quality criteria,
such as $\chi^2$ and number of hits requirements, beyond the stingent initial criteria.
This helps to reconstruct difficult to build tracks. Tracks can be difficult to 
reconstruct for multiple reasons such as detector inefficiencies leading to missing
detector hits, or particles originating from elsewhere in the detector 
besides along the beam axis. These tracks could be from hadrons interacting
within the tracker material before reaching the eight-hits threshold, or from the decay
of particles with finite life-times. Additionally, there can be difficulty disintangling
the many tracks within a collimated ``jet'' where many of the tracks are close to or
nearly overlapping with one another.
The misreconstruction rate is suppressed in each iterative step despite the loosening
quality criteria by the removal of the hits which are previously incorporated into
a reconstructed track. This  suppresses the random hit-to-seed association in the
next iteration and allows moderate efficiency gains for only small misreconstruction losses.
The efficiency and misreconstruction rate in figure~\ref{fig:kf_tracking} shows the
results for the initial pass throug the tracking algorithm, the results after all
passes which require the track seed to contain hits in the pixel detector, and the
final results after considering displaced tracks as well.
After all iterations the efficiency is about 90\% for a charged pions with $\pt = 10\GeV$
for a misreconstruction rate of 3\%.

\begin{figure*}[htbp]
\centering
     \includegraphics[width=0.45\textwidth]{object_reconstruction_and_selection/plots/pf_track_eff.pdf}
     \includegraphics[width=0.45\textwidth]{object_reconstruction_and_selection/plots/pf_track_misId.pdf}
     \caption{
Efficiency (left) and misreconstruction rate (right) of the global combinatorial track finder (black squares) 
which is the first pass through the tracking algorithm. The prompt iterations of the tracking method (green 
triangles) show the results after all iterations based on seeds with at least one 
hit in the pixel detector are completed. The final results after all iterations (red circles) includes
iterations with displaced seeds. Efficiency and misreconstruction are plotted as a function 
of the track $\pt$, for charged hadrons in multijet events without pileup interactions. Only tracks with 
$\abs\eta < 2.5$ are considered. The efficiency 
is displayed for tracks originating from within 3.5 cm of the beam axis and $\pm$30 cm of the nominal 
centre of CMS along the beam axis.
     }
     \label{fig:kf_tracking}
\end{figure*}

Tracks which are likely associated with electrons receive special treatment in PF. Electrons
will often emit bremsstrahlung radiation while propagating through the tracker making their
tracks difficult to reconstruct because of the sudden kinks. 
When energetic photons are radiated from an electron, the pattern recognition in the KF algorithm
may have difficulty accommodating the sudden change in electron momentum. This can cause the track to 
be reconstructed with a small number of hits than would be associated with the true electron
path. A new collection of tracks is created based on a preselection on the number of hits and 
the fit $\chi^2$ for the reconstructed KF-based tracks. The new collection which is a subset of the
KF-based tracks are fit again with a Gaussian-sum filter (GSF)~\cite{gsf_electrons}. Instead of
modeling the energy loss of particles as a single gaussian probability density function (PDF) like the KF
algorithm does, the GSF models the energy loss as a mixture of multiple gaussiand PDFs. 
The GSF fitting algorithm is more adapted to electrons than the KF used in the iterative tracking.
It allows for sudden and substantial energy losses along the trajectory. The
additional freedom here allows much better fits for the electron-based tracks and provides
better estimates for the track origin, trajector towards the calorimeters, and $\pt$.

Muon tracks are built from a combination of the pixel and strip tracker information and the
muon spectrometer information. High purity muon hits in the muon spectrometer is granted by the 
calorimeters and the solenoid which absorb the vast majority of non-muon particles, except neutrinos, 
before reaching the muon spectrometer. There are three difference categories of muon tracks
reconstructed:
\begin{itemize}
\item Standalone muons are based on hits within each DT or CSC detector. Hits are clustered to form 
track segments that are used as seeds for the pattern recognition in the muon spectrometer to gather
other hits in the muon systems along the muon trajectory.
\item Global muons are reconstructed from a standalone-muon track which is matched to a track 
in the inner tracker. If the trajectory parameters of the two tracks propagated onto a common surface 
they are considered compatible and the hits from the inner track and from the standalone-muon track are combined and fit to form a global-muon track.
\item Tracker muons are built from an extrapolation of a track from the inner track system to a
single compatibe muon segment within the muon spectrometer system. The inner tracker-based track must have 
a $\pt > 0.5\GeV$ and a total momentum p in excess of 2.5 GeV.
\end{itemize}

About 99\% of the muons produced within the geometrical acceptance of the muon system are 
reconstructed either as a global muon or a tracker muon and very often as both.
For muons with $\pt > 200\GeV$ the momentum resolution, based on the inner track, is 
improved by the inclusion of the track extension to the muon system.
For muons with $\pt < 200\GeV$ the inner tracker already provides a precise measurement of 
their momentum.



\subsection{Particle Flow Energy Clusters}
Energy clusters make up the second basic building block to construct PF physics-objects. The PF energy
clustering algorithm which constructs the energy clusters serves multiple purposes. The energy clustering
is designed to:
\begin{itemize}
\item Detect and measure the energy and direction of stable neutral particles (photons and neutral hadrons)
\item Separate neutral particles from charged hadron energy deposits
\item Reconstruct and identify electrons and all accompanying bremsstrahlung photons
\item Assist the energy measurement of charged hadrons for which the track parameters were not 
determined accurately; this is primairly the case for low-quality and high-$\pt$ tracks
\end{itemize}
The clustering is performem separately in each subdetector, the ECAL barrel and end caps and HCAL barrel
and end caps with an aim of a high detection efficiency even for low-energy particles and the ability to 
separat close energy deposits. Clustering begins with seed hits which have an energy above the seed hit
threshold and an energy larger than the energy of the adjacent hits. In the ECAL barrel, the fine 
granularity of the ECAL allows
for the clustering to consider all eight adjacent hits, four on the sides and four on the corners.
The HCAL barrel has a more coarse granularity, so only the four adjacent sides are considered when
selecting a seed candidate. From the starting seed, topological clusters are grown outward by aggregating
hits with at least a corner in common with a cell already included in the cluster. Hits must have an energy
in excess of two times the subdetector noise level to be considered for clustering. The energy thresholds
for seeding a cluster and cluster inclusion threshold are in table~\ref{tab:pf_cluster_thresholds}.


\begin{table*}[htbp]
\centering
\begin{footnotesize}
%\begin{scriptsize}
\begin{tabular}{|l|cc|cc|}
\hline
        &       \multicolumn{2}{|c|}{ECAL}        &       \multicolumn{2}{|c|}{HCAL}        \\  %&   Preshower \\
        &   barrel  &   endcaps        &       barrel   &   endcaps        \\  %&    \\
\hline
Cell $E$ threshold (\MeV) & 80 & 300    &   800 &   800 \\   %& 0.06 \\
\hline
Seed Number closest cells   &   8 & 8   &   4 & 4 \\    %& 8 \\
Seed $E$ threshold (\MeV)   &  230  &  600  &  800  & 1100 \\    %& 0.12  \\
Seed $E_{\text{T}}$ threshold (\MeV)   &  0  &  150  & 0   & 0 \\    %& 0 \\
%\hline
%Gaussian width (cm)   &  1.5  &  1.5  &  10.0  & 10.0  \\    %& 0.2 \\
\hline
\end{tabular}
\end{footnotesize}
%\end{scriptsize}
\label{tab:pf_cluster_thresholds}
\caption{
The clustering parameters used for ECAL and HCAL energy deposit clustering. The ECAL endcap
requires an additional seed $E_{\text{T}}$ threshold because the detector noise
increases as a function of $\abs\eta$.
}
\end{table*}

Residual energy calibrations are applied to the ECAL and HCAL energy clusters. The calibrations are
designed to account for the effects of the hit energy thresholds which will always result in a 
smaller amount of energy being incorporated into a cluster than was measured by the detector
for a given single object.
In the ECAL, the residual energy calibration is determined from simulated single photon events.
This generic calibration is applied to all ECAL clusters prior to the hadron cluster calibration 
mentioned next. 

ECAL and HCAL energy clusters are linked together as a potential hadronicy decay energy deposit
if their positions in \etaphi overlap. For a hadronic decay, the total calorimeter response (ECAL + HCAL) depends on 
the fraction of the shower energy deposited in the ECAL, and is not linear with energy. The 
ECAL and HCAL cluster energies therefore need to be substantially recalibrated to get an 
estimate of the true hadron energy. Simulated single neutral hadrons, specifically single 
$\text{K}^{0}_{\text{L}}$s, are used for the hadronic decay response calibration seen for the
barrel in figure~\ref{fig:pf_calo_calib}. The applied calibrations in the left plot lead
to excellent agreement in the calorimeter response in the right plot. It is much harder to
make larger improvements in the energy resolution without actually changing the available
input calorimeter hit information granularity.


\begin{figure*}[htbp]
\centering
     \includegraphics[width=0.45\textwidth]{object_reconstruction_and_selection/plots/calo_calibrations.pdf}
     \includegraphics[width=0.45\textwidth]{object_reconstruction_and_selection/plots/calo_response_and_res.pdf}
     \caption{
(left) Calibration coefficients obtained from single $\text{K}^{0}_{\text{L}}$s in the barrel as a 
function of their true energy $E$. The blue triangles show the calibrations for hadrons depositing 
energy only in the HCAL. The red circles (green squares) show the ECAL (HCAL) calibration for hadrons
depositing energy in both the ECAL and HCAL.
(right) Relative raw (blue) and calibrated (red) energy response (dashed curves and triangles) and 
resolution (full curves and circles) for single $\text{K}^{0}_{\text{L}}$s in the barrel, as a 
function of their true energy $E$.
%Here the raw (calibrated) response and resolution are obtained by a Gaussian fit to the distribution of 
%the relative difference between the raw (calibrated) calorimetric energy and the true hadron energy.
     }
     \label{fig:pf_calo_calib}
\end{figure*}


In general, a given particle is expected to result in multiple PF elements (tracks and energy
clusters) in the various CMS subdetectors. The reconstruction of a particle first proceeds 
with a link algorithm that connects the PF elements from different subdetectors. For example,
tracks are linked to energy clusters if the extrapolated position of the track aligns within the
angular acceptance of an energy cluster in \etaphi. Energy clusters can be linked between subdetectors as mentioned
in the case of hadronic energy deposits above which span the ECAL and HCAL. A link is established
when the cluster position in the more granular calorimter, ECAL, is within the cluster envelope
in the less granular calorimeter, HCAL. Once PF elements are linked together they are referred to as a PF block which can contain 
elements associated either by a direct link or by an indirect link through common elements.
The link distance is defined as the distance between the extrapolated track position and the 
cluster position in the \etaphi plane.


\subsection{Particle Flow Candidates}
Particle Flow candidates are selected from the PF blocks based on designated quality cuts.


\subsubsection{Muons}
Isolated global muons are selected from the global muon track colletion by looking at the inner
tracker tracks and calorimeter 
energy deposits within a distance $\dr < 0.3$ to the muon trajectory. 
The sum of the $\pt$ of the tracks and of the ET of the energy deposits is required not to 
exceed 10\% of the muon $\pt$. This isolation criterion alone successfully reject hadrons that would
otherwise be misidentified as muons. No further selection is applied to these muon candidates.
If the muon track $\pt < 200\GeV$, then the momentum assigned to the muon PF candidate is that of 
the inner track. For muon tracks with $\pt > 200\GeV$, the momentum assigned is the momentum associated
with the smallest $\chi^2$ probability from these different track fits: tracker only, tracker and 
first muon detector plane, global, and global without the muon detector planes featuring a high occupancy.
The PF elements and blocks that make up an identified PF muon candidate masked against further processing
to prevent their inclusion in other PF candidates.

In these analyses, there are several additional criteria applied beyond that which is required to
be a PF candidate. These analyses use muons which pass two different PF muon identification working
points: \texttt{PF ID Loose} and \texttt{PF ID Medium}~\cite{sm-htt-2017}. The \texttt{PF ID Loose}
working point is only slightly tighter than the baseline criteria for a PF muon candidate; 
\texttt{PF ID Loose} must be either a global or a tracker PF muon. This selection is highly efficient
for prompt muons.

The \texttt{PF ID Medium} muon working point requires first that muons pass \texttt{PF ID Loose} then
applies additional track-quality and muon-quality requirements on the different muon tracks which are
linked to the PF muon candidate. The number of valide inner tracker
hits must be greater than 80\%. Additionally either one or the other of the following criteria must be met:
\begin{outline}
\1 Option 1 - ``Tight Segment Compatility''
    \2 Candidate has a segment compatility score of at least 0.451 which ensures that the 
track is reasonablly compatible with inner tracker-based track
\1 Option 2 - ``Good Global Muon'':
    \2 PF muon candidate is a global muon
    \2 The normalized track $\chi^2 < 3$
    \2 The compatibility $\chi^2$ between the standalone muon track and the inner tracker muon is
less than 12
    \2 The muon track kink-finder, which is desiged to remove muons produced from in-flight decays, 
must have a value less than 20
    \2 Candidate has a segment compatility score of at least 0.303 which is looser than the value
required in Option 1
\end{outline}
The \texttt{PF ID Medium} muon working is still very efficienct for prompt muon selection but
does bring some additional reduction in fake object selection which is helpful in the high
statistics $htt$ analysis. 

To reject non-prompt or misidentified muons and electrons, a relative lepton isolation is defined as:
\begin{equation}
I^{\ell} \equiv \frac{\sum_{charged}  \pt + \max\left( 0, \sum_{neutral}  \pt
                                         - \frac{1}{2} \sum_{charged, PU} \pt  \right )}{\pt^{\ell}}.
\label{eq:rel_isolation}
\end{equation}

Some of the quantities mentioned here refer to descriptions in the following.
In this expression, $\sum_{charged}  \pt$ is the scalar sum of the transverse energy of the 
charged particles originating from the primary vertex and located in a cone of size
$\dr = \sqrt{\smash[b]{(\Delta \eta)^2 + (\Delta \phi)^2}} = 0.4$\,(0.3)
centered on the muon (electron) direction. The sum, $\sum_{neutral}  \pt$, represents
a similar quantity for neutral particles. The contribution of photons and neutral hadrons 
originating from pileup vertices is estimated from the scalar sum of the transverse
energy of charged hadrons in the cone originating from pileup vertices,
$\sum_{charged, PU} \pt$. This sum is multiplied by a factor of $1/2$, which corresponds 
approximately to the ratio of neutral to charged hadron production in the hadronization process
of inelastic $\Pp\Pp$ collisions, as estimated from simulation. The expression $\pt^{\ell}$ 
stands for the $\pt$ of the lepton. Isolation requirements used in the following analyses, 
based on $I^{\ell}$, range from $I^{\ell} < 0.1$ to $I^{\ell} < 0.25$ depending on the signal efficiency and
background rejection needs of the specific final state. For the $\htt$ analysis, these
working points are listed in the $\htt$ analysis section, table~\ref{tab:htt_obj_selection}.




\subsubsection{Electrons and Prompt Photons}
With the muon related PF elements masked from further processing, electron and prompt photon
identification begins.
Electron identification is based on information from the inner tracker tracks and the calorimeter
energy clusters. Because of bremsstrahlung radiation, electron tracks can be much more kinked
than those for other particles as was discussed above in relation to the GSF algorithm track
fitting. Additionally, because of radiated bremsstrahlung photons, the resulting energy clusters 
from an electron propagating through the detector can be spread out in the
$\phi$ direction. PF electrons are built from the linking of a GSF track to an
ECAL-based energy cluster. To suppress the amount of charged hadrons faking electrons, the 
sum of the energies measured in HCAL hits behind ($\dr < 0.15$) 
the electron-linked ECAL energy cluster must no exceed 10\% of the ECAL-based 
energy cluster energy~\cite{Sirunyan:2017ulk}. The energy assignment for an electron candidate is obtained from a 
combination of the calibrated ECAL energy with the momentum of the GSF track. The electron 
direction is chosen to be that of the GSF track.

Before being saved to the PF electron collection, electron candidates must satisfy additional
identification criteria targeted at reducing electron fakes. Up to fourteen variables are
fed into a Boosted Decision Tree (BDT) which selections the passing PF electrons. The BDT
input variables include track and energy cluster details such as:
\begin{itemize}
\item Amount of energy radiated off the GSF track
\item Distance between the GSF track extrapolation to the ECAL entrance and the position of the ECAL seeding cluster
\item Ratio between the energies gathered in HCAL and ECAL by the track-cluster association process
\item The KF and GSF track $\chi^2$ values, and
\item The numbers of inner tracker hits
\end{itemize}

Photon candidates are seeded by ECAL energy clusters with no matching KF or GSF track.
They are retained as PF photons if they are isolated from other tracks and calorimeter energy clusters, 
and if the ECAL energy distribution and the ratio between the HCAL and ECAL energies, $H/E$, are 
compatible with those expected from a photon shower. Similar to the PF masking after the muon 
reconstruction, tracks and energy clusteres used to
build PF electron and photons are masked from further processing simplifying the task ahead
for charged and neutral hadron identification.

There are additional electron identification requirements used in these analyses which are tighter
than the PF electron criteria. The additional identification criteria relies on a multivariate (MVA) discriminant
which combines many of the same variables used in the PF electron BDT~\cite{Khachatryan:2015hwa} 
and addes some additional energy cluster distribution variables such as: $\sigma_{i\eta i\eta}$ and
$\sigma_{i\phi i\phi}$ cluster shape covariance. The analyses use
two different MVA working points, one with 90\% signal efficiency and one with 80\% efficiency.


\subsubsection{Charged and Neutral Hadrons}
Once muons, electrons, and isolated photons are identified and removed from the available PF blocks
and elements, the remaining particles to be identified are hadrons resulting from jet fragmentation and 
hadronization. The ECAL and HCAL energy clusters which are not linked to any tracks are turned into
PF neutral hadrons while energy deposits successfully linked to a track are turned into PF charged
hadrons. Non-isolated photons are indistinguishable from the neutral hadron group and are considered
as part of that collection. Charged and neutral hadron form the last sets of fundamental physics-objects
which are reconstructed by PF. The next PF steps involve reconstructing composit object such as ``jets''
and hadronically decaying tau leptons and calculating event quantities such as the primary vertex and
\etvecmiss.


\section{Event Level Quantities}
There are multiple event level quantities that require input from all PF physics-objects for their
calculation. The calculation of the primary vertex is specifically needed for the identification of
composit objects such as ``jets'' and hadronically decaying $\tau$ leptons.


\subsection{Primary Vertex Reconstruction}
The original location of the $\pp$ collisions which gave rise to a given event can be found by tracing
the reconstructed physics-object tracks back to the collision region and grouping together tracks that share a common
origin; the origins are called vertices. In any given recorded collision there will be usually be a 
singular hard-scatter $\pp$ collision and multiple soft-scatter collsion. The hard-scatter vertex is 
identified as the vertex with the largest quadratic sum of the $\pt$ of the associated physics-objects 
and is called the primary vertex~\cite{Sirunyan:2017ulk}. The other vertices are referred to as the
pileup vertices.


\subsection{Missing Transverse Energy}
The CMS detector can detect and measure the energy of all standard model particles with the exception
of neutrinos which leave the detector undetected. The neutrino energy contribution to an event can
be estimated using the missing transverse energy, \etvecmiss. The \etvecmiss is calculated from
the raw missing transverse momentum vector which is defined 
to balance the vectorial sum of the transverse momentum of all particles.

\begin{equation}
\vec{p}^{\text{miss}}_{\text{T,PF}}(\text{raw}) = - \sum^{N_{\text{particles}}}_{i=1} \vec{p}_{\text{T},i}
\end{equation}

All particles reconstructed in the event are used to determine the missing transverse energy,
\etvecmiss~\cite{Khachatryan:2014gga}. The specific \etvecmiss used in these analyses is Type-1
\etvecmiss which is adjusted for the effect of jet energy corrections.


\section{Composit Object Identification and Selection}
Composit objects reconstructed by PF attempt to cluster together physics-objects which likely
resulted from the jet fragmentation or hadronization process. The three groups used in these
analyses are discussed below.

 
\subsection{Jets}
Jets are collections of energy deposits and tracks within a defined conical area radiating outward
from the collision region. They are created when a quark or gluon undergoes the hadronization process.
Jets are reconstructed with an anti-\kt clustering algorithm implemented in the \FASTJET 
library~\cite{Cacciari:2008gp, Cacciari:2011ma, Cacciari:fastjet2}. The anti-\kt clustering is based on the grouping
together of neutral and charged PF candidates within a distance parameter $\dr = 0.4$. Charged PF 
candidates not associated with the primary vertex are not considered when building jets.
A correction is applied to jet energies to adjust for the contribution to the jet energy from 
additional $\pp$ interactions within the same or nearby bunch crossings. The energy of a jet is 
corrected via calibrations based on simulation and data~\cite{CMS-JME-10-011}.


\subsection{b-jet ID and Secondary Vertex}
Jets which result from the decay and hadronization of b quarks are used to help classify events
in the $\htt$ analyses. B quarks are not produced from the four leading Higgs boson production
mechanisms. Therefore, in multiple final states, events with jets which are likely caused by the decay and
hadronization of b quarks are a likely sign of a background event.
The combined secondary vertex (CSVv2) algorithm is used to identify jets that likely resulted
from a b quark decay, a ``b-jet''. The CSVv2 algorithm uses track-based lifetime information together with 
the reconstructed secondary vertices associated with the jet to provide a likelihood ratio 
discriminator for b-jet identification. The b jet identification working point chosen in the $\htt$ analyses 
gives an efficiency for identifying genuine b jets of about 70\%, and for misidentifying light flavor jets
as b jets of about 1\%.


\subsection{Taus}
Hadronically decaying $\Pgt$ leptons are reconstructed in PF with the hadron-plus-strips (HPS)
algorithm~\cite{Khachatryan:2015dfa, CMS-PAS-TAU-16-002} which is seeded with the collection of 
anti-\kt jets discussed above. The HPS algorithm reconstructs $\tauh$ candidates based on their
compatibility with one of the primary $\tau\to\tauh$ decay modes listed in table~\ref{tab:tau_dms}.
A large number of decay modes involve decays of $\PGpz  \to  \gamma\gamma$. The $\PGpz$ are
searched for in strips of \etaphi elongated in the $\phi$ direction to catch photon conversions.
Since the start of Run-II, the HPS algorithm has moved from using a fixed width strip,
$0.05 \eta \times 0.20 \phi$, to a dynamic width strip where the width depents on the 
$\pt$ of the electron or photon used to seed the strip. More dynamic strip details in the following paragraph.
Strips containing one or more electron of photon constituent and passing a cut of $\pt > 2.5\GeV$
for the sum of the electrons and photons are kept as $\PGpz$ candidates. Based on the number of
charged hadrons and $\PGpz$s, the $\tauh$ candidate is assigned a decay mode. For the $\tau$
decay modes involving a meson resonance, there is a mass window requirement that necessitates
that the $\tauh$ invariant mass be consistent with the meson resonance. $\tauh$ with only a single
charged hadron are assigned a mass equal to the mass of a $\pi^{\pm}$, 140\MeV. Figure~\ref{fig:tau_mass}
shows the reconstructed mass for hadronically decaying taus in 2012 data using a selection with
high genuine tau purity in the $\Pgm\tauh$ final state.

\begin{table*}[htbp]
\centering
\begin{tabular}{|l|cc|}
\hline
Decay Mode                                             &   Meson Resonanace     & $\mathcal{B}$ (\%) \\
\hline
$\tau^{-}  \to  \Pe^{-}\bar{\nu}_{\text{e}}\nu_{\tau}$ &                        &     17.8  \\
$\tau^{-}  \to  \Pgm^{-}\bar{\nu}_{\mu}\nu_{\tau}$     &                        &     17.4  \\
$\tau^{-}  \to  h^{-}\nu_{\tau}$                       &                        &     11.5  \\
$\tau^{-}  \to  h^{-}\PGpz\nu_{\tau}$                  &      $\rho$(770)       &     26.0  \\
$\tau^{-}  \to  h^{-}\PGpz\PGpz\nu_{\tau}$             &      a1(1260)          &     10.8  \\
$\tau^{-}  \to  h^{-}h^{+}h^{-}\nu_{\tau}$             &      a1(1260)          &      9.8  \\
$\tau^{-}  \to  h^{-}h^{+}h^{-}\PGpz\nu_{\tau}$        &                        &      4.8  \\
Other modes with hadrons                               &                        &      1.8  \\
\hline
Total leptonic modes                                   &                        &     35.2  \\
Total hadronic modes                                   &                        &     64.8  \\
\hline
\end{tabular}
\label{tab:tau_dms}
\caption{
Decay modes for $\tau^{-}$ leptons including leptonic decays and hadronic decays. 
The $h^{\pm}$ stand for $\pi^{\pm}$ or $K^{\pm}$. Inverting all
of the ``-'' for ``+'' will give the decay modes for $\tau^{+}$ leptons. Nearly 65\% of $\tau$
leptons decay hadronically to $\tauh$.
}
\end{table*}

\begin{figure*}[htbp]
\centering
     \includegraphics[width=0.55\textwidth]{object_reconstruction_and_selection/plots/CMS-TAU-14-001_tau_mass.pdf}
     \caption{
The reconstructed invariant mass of the $\tauh$ candidate. A spike is seen at 140\MeV for the 1-prong
$\tauh$ decay mode where the mass is assigned equal to the mass of a $\pi^{\pm}$. The 1-prong+$\PGpz$
decay mode is seen to peak around 770\MeV while the 3-prong decay mode centers around 1260\MeV.
     }
     \label{fig:tau_mass}
\end{figure*}

The dynamic strip reconstruction has been optimized to best group together the $\PGpz$ 
decay products within a $\tauh$. It was found that the fixed width strips used in Run-I
occasionally allowed the $\PGpz$ decay products to escape a strip which would result in
the escaped particle being added to the isolation sums for the $\tauh$ and not contributing
to the energy or momentum of the $\tauh$. Widening the fixed width strips can be used to catch
these escaping particles. However, considering that more boosted $\tauh$ will have a more 
collimated structure, it is also helpful to narrow the strip at higher $\tauh \pt$ to suppress
pileup or other non-$\tau$ contributions when possible. The strip widths are calibrated to
on simulations that target retaining 95\% of the $\PGpz$ decay products, see figure~\ref{fig:tau_dyn_strip}.

\begin{figure*}[htbp]
\centering
     \includegraphics[width=0.45\textwidth]{object_reconstruction_and_selection/plots/tau_dyn_strip_eta.pdf}
     \includegraphics[width=0.45\textwidth]{object_reconstruction_and_selection/plots/tau_dyn_strip_phi.pdf}
     \caption{
Distance in $\eta$ (left) and $\phi$ (right) between $\tauh$ and e/$\gamma$, that are due to 
hadronic tau decay
products, as a function of e/$\gamma \pt$. The size of the window is larger in the $\phi$
direction due to bending in the magnetic field. The dotted line shows the 95\%
quantile while the red line shows the fit to the 95\% quantile. The red line is used
to define the widths of the dynamic strip.
     }
     \label{fig:tau_dyn_strip}
\end{figure*}

A multivariate (MVA) discriminator~\cite{Hocker:2007ht}, including isolation, shape-based variables
and lifetime information, is used to reduce the rate for  quark and gluon initiated jets
to be identified as $\tauh$ candidates. The working point used in the $\htt$ analysis, \texttt{Tight Tau MVA},
has an efficiency of about 60\% for genuine $\tauh$,
with about 1\% misidentification rate for quark- and gluon-initiated jets, for a $\pt$ range typical 
of $\tauh$ originating from a $\PZ$ boson. A looser working point is used in the Higgs associated 
production analysis ZH final states, \texttt{Medium Tau MVA}, which as an efficiency of 65\% for genuine
$\tauh$ with a 2\% misidentification rate.

Electrons and muons can both be reconstructed as $\tauh$ candidates, usually into the 1-prong or
1-prong+$\PGpz$ decay modes. For electrons this can happen easily when an electron emits a bremsstrahlung
photon mimicking a $\PGpz$s in the decay mode reconstruction. MVA discriminants have been developed 
which specifically target the suppression of electrons and muons being misidentified as 
$\tauh$~\cite{Khachatryan:2015dfa, CMS-PAS-TAU-16-002}.
A range of anti-e and anti-$\Pgm$ discriminant working points are used in these analyses. The choice
of which working point to use is tailored towards suppressing the dominant backgrounds in different
final states and is discussed in detail in the analysis sections~\ref{sec:htt_analysis, sec:vh_analysis}.



%\chapter{Analysis Strategy: Higgs $\to \tau\tau$}

This chapter describes a study of Higgs Boson production and subsequent
decay to a pair of $\tau$ leptons using CMS proton-proton collision data gathered in 2016.  This is the first
$\htt$ analysis performed using center-of-mass energy 13 TeV data from the LHC. Combining
these 13 TeV results with 7 TeV and 8 TeV CMS $\htt$ results we produce
the first single experiment observation of the $\htt$ process, observed at the 5.9 $\sigma$
confidence level.  Additionally, this study provides the strongest constraints on VBF Higgs 
production to date for all CMS Higgs Boson analyses.

\section{Overview: Higgs $\to \tau\tau$}

This chapter specifically focuses on studying the Higgs Boson produced via the gluon fusion
or the VBF production mechanisms.  A study of the Higgs Boson produced in associated production with
$\PW/\PZ$ is presented in the following chapter~\ref{sec:vh_analysis}. This study utilizes the
full 2016 $\pp$ dataset collected by CMS corresponding to 35.9$\fbinv$ of integrated luminosity.
In the following pages the symbol $\ell$ refers to electrons and muons and $\tauh$ refers to hadronically
decaying $\tau$ leptons.  We study all possible $\tau\tau$ final state combinations with the
exception of two electron and two muon final states because of the low 
$\tau\tau \to \tau_{e}\tau_{e}/\tau_{\mu}\tau_{\mu}$
branching fractions and high background from $\PZ \to ee/\mu\mu$.  The $\htt$ final states which are
studied are: $\tau_{e}\tauh$ denoted here as $\Pe\tauh$, $\tau_{\mu}\tauh$ denoted as $\Pgm\tauh$,
$\tau_{e}\tau_{\mu}$ denoted here as $\Pe\Pgm$, and lastly, $\tauh\tauh$ denoted as $\tauh\tauh$.
This combination of final states covers about 94\% of all possible $\tau\tau$ final states.
The different $\tau\tau$ final states will be refered to as different channels in the following pages.
We ensure uniqueness between the four studied channels be applying veto criteria to events based
on the number of reconstructed and loosely identified electrons and muons.  This ensures that 
no data or simulated event is double counted in two channels.

Selected events are classified into three different categories targeting different characteristics
of the gluon fusion and VBF production topologies.  The categories are defined according to the
number and kinematics of the associated jets in each event along with the reconstructed $\pt^{Higgs}$.
A number of different control regions are used in the final fit for signal extraction.  This allows
the fit to simultaneously adjust and constrain all processes targeted by a control region.  This is in
contrast to extracting a scale factor which is applied as a fixed value in an analysis and not
allowed to adjust as other background process adjust in the final fit.  The backgrounds which are
targeted with dedicated control regions are: $\PW$+jets, QCD, and $\ttbar$.



\subsection{Event Selection}

There are specific baseline criteria applied to all electrons, muons, $\tauh$, and jets for every
event.  Depending on the final state, additional requirements are placed on these objects based
on combination or trigger requirements and analysis optimization.  The baseline criteria ensure
that each object is well reconstructed and well identified and consistent with the analysis
strategy. 

\subsection{Triggers}
Selected events are required to to have fired a trigger consisten with their categorized final
state channel.  For the $\Pgm\tauh$ channel, events are selected using a combination
of a single isolationed muon trigger as well as a cross trigger firing on an isolated muon and
an isolated $\tauh$.  The $\Pgm\tauh$ cross trigger allows for a lower possible $\pt$ threshold
on the selected muon.  In contrast, due to the HLT menu available in 2016, the $\Pe\tauh$ cross triggers
do not bring a substantial increase in acceptance and were not used in this analysis.
For the $\Pe\tauh$ channel, events are selected using only a trigger firing on a 
single isolated electron.  For the $\Pe\Pgm$ channel, events are selected with electron-muon corss
triggers requiring an isolated online electron and an isolated online muon.  There are two 
different electron-muon corss triggers used with their HLT $\pt$ thresholds detailed following.
For the $\tauh\tauh$ channel, events are selected using an online criteria of two loosely isolated $\tauh$.  
The high level trigger paths used and their online $\pt$ thresholds are detailed in table~\ref{tab:htt_hlt_triggers}.

Due to changing running conditions and a changing HLT menu throughout 2016 data taking, the HLT path
requirements change throughout 2016 and are era dependant.  The study avoids using any prescaled
HLT paths; during latter eras, at higher instantaneous luminsity, some HLT paths became diabled
or prescaled.  When this happens we avoid using that specific trigger for the given prescaled/disabled
era.  For the $\Pgm\tauh$ channel this 
leads to changes in the $\abs\eta$ requirement on the online muon in the single muon triggers.
There are two muon-tau cross triggers available throughout 2016.  One of them requires only the
presence of a single muon at the Level-1 trigger and has ``SingleL1'' appended to its path
name.  The other one requires the presence of both a muon and a $\tauh$ at the Level-1 trigger.
The single electron trigger used in the $\Pe\tauh$ channel remained constant through the year.
Due to changing pileup conditions and increasing luminosity during the final eras, the online
isolation criteria was changed for the $\tauh\tauh$ channel triggers.  The online $\tauh$ 
isolation changed from being based on purely charged energy deposits to being based on a 
combination of charged and neutral based energy deposits.  The triggers useded in the
$\Pe\Pgm$ channel changed during the final two eras, G and H, and had a lepton DZ filter
applied to them.  The DZ filter requires that the electron and muon match to the HLT primary
vertex and is a common method used to reduce rate at the HLT.  In all channels, the electrons, muons, and
$\tauh$ in each event must be matched to within $\Delta R < 0.5$ with the associated
HLT object which triggerd the event.


\begin{table*}[htbp]
\centering
\begin{footnotesize}
%\begin{scriptsize}
\begin{tabular}{|l|l|l|l|}
\hline
  Channel           &         Trigger $\pt$ Req.              &     High Level Trigger Path  &   Eras  \\
\hline
  $\mu\tauh$       &         $\Pgm(22)$                     &  \scriptsize{HLT\_IsoMu22\_v*} & B-F  \\ 
                   &         $\Pgm(22)$                     &  \scriptsize{HLT\_IsoTrkMu22\_v*} & B-F   \\ 
                   &         $\Pgm(22)$                     &  \scriptsize{HLT\_IsoMu22\_eta2p1\_v*} & C-H  \\
                   &         $\Pgm(22)$                     &  \scriptsize{HLT\_IsoTrkMu22\_eta2p1\_v*} & C-H   \\
                   &         $\Pgm(19)\,\&\,\tauh (21)$     &  \scriptsize{HLT\_IsoMu19\_eta2p1\_LooseIsoPFTau20\_SingleL1\_v*} &  All Eras  \\
                   &         $\Pgm(19)\,\&\,\tauh (21)$     &  \scriptsize{HLT\_IsoMu19\_eta2p1\_LooseIsoPFTau20\_v*} &  All Eras \\
\hline
  $\Pe\tauh$       &         $\Pe (25)$                     &  \scriptsize{HLT\_Ele25\_eta2p1\_WPTight\_Gsf\_v*}   & All Eras \\
\hline
 $\tauh\tauh$      &         $\tauh (35)\,\&\,\tauh (35)$   &  \scriptsize{HLT\_DoubleMediumIsoPFTau35\_Trk1\_eta2p1\_Reg\_v*} & B-G   \\ 
                   &         $\tauh (35)\,\&\,\tauh (35)$   &  \scriptsize{HLT\_DoubleMediumCombinedIsoPFTau35\_Trk1\_eta2p1\_Reg\_v*} & H  \\
\hline
  $\Pe\Pgm$        &         $\Pe(12)\,\&\,\Pgm (23)$       &  \scriptsize{HLT\_Mu23\_TrkIsoVVL\_Ele12\_CaloIdL\_TrackIdL\_IsoVL\_v*} & B-F \\
                   &         $\Pe(12)\,\&\,\Pgm (23)$       &  \scriptsize{HLT\_Mu23\_TrkIsoVVL\_Ele12\_CaloIdL\_TrackIdL\_IsoVL\_DZ\_v*} & G-H  \\
                   &         $\Pe(23)\,\&\,\Pgm (8)$        &  \scriptsize{HLT\_Mu8\_TrkIsoVVL\_Ele23\_CaloIdL\_TrackIdL\_IsoVL\_v*} & B-F  \\
                   &         $\Pe(23)\,\&\,\Pgm (8)$        &  \scriptsize{HLT\_Mu8\_TrkIsoVVL\_Ele23\_CaloIdL\_TrackIdL\_IsoVL\_DZ\_v*} & G-H  \\
\hline
\end{tabular}
\end{footnotesize}
%\end{scriptsize}
\caption{For each channel the online HLT $\pt$ threshold is listed along with the specific associated
HLT paths.  Through out 2016 data taking, the HLT menu changed to respond to changing online
conditions at CMS and the LHC.  This is reflected in progressivly tighter triggers
being available towards the end of the 2016 run.
\label{tab:htt_hlt_triggers}
}
\end{table*}



\subsection{Baseline Object Selection}
All electrons and muons must meet the minimum requirement
that the distance of closest approach to the primary vertex satisfies $\abs{d_z}<0.2$ cm
along the beam direction, and $\abs{d_{xy}}<0.045$ cm in the transverse plane. Ensuring
compatibility with the primary vertex is consistent with the predicted infinitesimal life-time of
a Higgs Boson. The HPS reconstruction of $\tauh$ detailed in section~\ref{sec:XXX} can involve
combining together multiple tracks and $\PGpz$s coming from intermediary $\tau$ decay products.
Due to these intermediary products, the reconstructed $d_{xy}$ for $\tauh$ are often
larger than those for electrons and muons.  Because of this, the primary vertex matching
criteria are relaxed for $\tauh$ and only require $\abs{d_z}<0.2$ cm.

The offline selection criteria for all electrons, muons and $\tauh$ are motivated and constrained
by the High Level Trigger requirements of their path.  Specifically, the offline $\pt$ criteria
applied are always higher than the HLT $\pt$ threshold to ensure a stable measurement and application
of trigger efficiencies.  An offline $\pt$ threshold applied right at the HLT $\pt$ threshold
makes measurement of the steeply rising efficiency at the HLT threshold absolutly critical.
This is very difficult to do perfectly and would lead to very large trigger systematics
at low $\pt$ in the turn-on region.  Additionally, there are offline $\eta$ restrictions
enforced which align with those at the HLT.

All selected electrons, muons, and $\tauh$ must be well identified and isolated from overlapping
energy deposits and reconstructed objects.  This study uses identification criteria
provided centrally by the CMS Physics Object Groups.  All identification and isolation
working points following have been selected through an optimization process selecting
for increased analysis sensitivity.  The optimimum working points strike a balance
between signal efficiency and background rejection.  For electrons an MVA-based ID
is used in both the $\Pe\tauh$ and $\Pe\Pgm$ channels which has been tuned to provide 
80\% electron selection efficiency.  Muons in both the $\mu\tauh$ and $\Pe\Pgm$ channels
the Particle Flow Medium ID is required.  The selection of $\tauh$ relies on MVA-based working
points.  The $\tauh$ MVA-based working points combine both object identification and object
isolation together into a single set of working points.  Table~\ref{tab:htt_obj_selection}
details the $\pt$, $\abs\eta$, identification and isolation criteria for all electrons, muons,
and $\tauh$ selected in the study.


\begin{table*}[htbp]
\centering
\begin{small}
\begin{tabular}{l|l|l|l|l}
  Channel       & $\pt$ ($\GeV$) & $\eta$ & Identification & Isolation \\
\hline
  $\mu\tauh$       &   $\pt^\Pgm>20$     &  $\abs{\eta^\Pgm}<2.1$    &   PF ID Medium &  $I^{\Pgm}<0.15$       \\
                   &   $\pt^{\tauh}>30$  &  $\abs{\eta^{\tauh}}<2.3$ &   MVA $\tauh$ ID  & MVA $\tauh$ ID \\
\hline
 $\tauh\tauh$      &   Leading $\pt^{\tauh}>50$ & $\abs{\eta^{\tauh}}<2.1$  &    MVA $\tauh$ ID    & MVA $\tauh$ ID    \\
                   &   Subleading $\pt^{\tauh}>40$ & $\abs{\eta^{\tauh}}<2.1$  &    MVA $\tauh$ ID & MVA $\tauh$ ID    \\
\hline
  $\Pe\tauh$       &   $\pt^\Pe>26$      & $\abs{\eta^\Pe}<2.1$       &   MVA 80\% WP  &  $I^{\Pe}<0.1$  \\
                   &   $\pt^{\tauh}>30$  &  $\abs{\eta^{\tauh}}<2.3$  &   MVA $\tauh$ ID & MVA $\tauh$ ID \\
\hline
  $\Pe\Pgm$        &   $\pt^{\Pe}>13$    & $\abs{\eta^\Pe}<2.5$   &   MVA 80\% WP   & $I^{\Pe}<0.15$   \\
                   &   $\pt^{\Pgm}>15$   & $\abs{\eta^\Pgm}<2.4$  & PF ID Medium &  $I^{\Pgm}<0.2$    \\
\hline
\end{tabular}
\end{small}
\caption{Kinematic, identification and isolation selection requirements for the four di-$\Pgt$ channels.
\label{tab:htt_obj_selection}
}
\end{table*}


The three different signal extraction categories rely on the details of reconstructed jets, or
lack there of, in each event.  Jets are reconstructed using the anti-$k_{\text{T}}$ algorithm with distance
parameter $\Delta\text{R}=0.4$~\cite{Cacciari:2008gp}.  Charged hardons that are not consistent with
the primary vertex are removed from the anti-$k_{\text{T}}$ clustering.  Jets are only considered
if they pass they pass the loose working point of the PF Jet ID discriminator~\cite{jetID}.
Jets must have $\pt > 30 \GeV$ and $\abs\eta<4.7$.  The $\pt$ and $\eta$ requirements are altered
for cases where the jet is identfied as having likely originated from a b-quark.  Jets likey originating
from a b-quark are considered if they pass the CISVv2 Medium working point are then tagged as b-tagged jets.
B-tagged jets have a relaxed $\pt$ requirement but much tighter $\eta$ requirement of $\pt > 20 \GeV$ 
and $\abs\eta<2.4$.  The tightened $\eta$ requirment necessitates that b-tagged jets are located within
the detector volumne fully covered by the CMS pixel and strip tracker.
Lastly, all jets must be separated from the selected electrons, muons, and $\tauh$ by $\Delta\text{R}>0.5$.

Depending on the di-$\tauh$ channel, there are specific topological cuts targeted at significantly
reducing the contribution of certain background processes in the signal region.  The large $\PW+\text{jets}$
cross section combined with a non-negligible jet $\to \tauh$ fake rate leads to a large $\PW+\text{jets}$
contribution in the $\ell\tauh$ channels.  This contribution is significantly reduced at the cost of
minimal signal events by cutting on the transverse mass, $\MT$.  Where the $\MT$ selection is defined as

\begin{equation}
\MT \equiv \sqrt{\smash[b]{2 \pt^\ell \ptmiss [1-\cos(\Delta\phi)]}} < 50\GeV,
\end{equation}

where $\pt^\ell$ is the transverse momentum of the lepton $\ell$,
and $\Delta\phi$ is the azimuthal angle between its direction and the \etvecmiss.

FIXME - Add MT plot from ZTT paper?

In the $\Pe\Pgm$ channel, the large \ttbar background is reduced by requiring 
$p_\zeta - 0.85 \, p_\zeta^{\text{vis}} > -35$ or $-10$\GeV depending on which category the event is
classified within.  $p_\zeta$ is the component of the \etvecmiss projected along the bisector 
of the transverse momenta of the two leptons and $p_\zeta^{\text{vis}}$ is the sum of the components 
of the lepton transverse momenta along the same direction~\cite{Khachatryan:2014wca}, also see
figure~\ref{fig:htt_pZeta} for visual reference.
This $p_\zeta$ selection criteria has a high signal efficiency because the \etvecmiss is typically oriented
in the same direction as the visible di-$\Pgt$ system in signal events because the \etvecmiss is 
the results of neutrinos from the signal $\tau$s.  The orientation of the \etvecmiss with respect
to the di-$\tau$ system is much less predictable in \ttbar events.  In addition, events with a b-tagged 
jet are discarded to further suppress the \ttbar background in the $\Pe\Pgm$ channel.

\begin{figure*}[htbp]
\centering
     \includegraphics[width=0.4\textwidth]{higgs_to_taus/plots/pZeta_def.pdf}
     \includegraphics[width=0.4\textwidth]{higgs_to_taus/plots/htt_em_pZeta.pdf}\\
     \caption{
(Left) Diagram showing the construction of the $p_\zeta$ and $p_\zeta^{\text{vis}}$ projections.
(Right) An example $p_\zeta - 0.85 \, p_\zeta^{\text{vis}}$ distribution is shown for a similar but not
overlapping selection in the $\Pe\Pgm$ channel.  This distribution is from an analysis focusing on specifically studying the
$\PZ\to\tau\tau$ process~\ref{HIG-15-007}.
In the $\htt$ analysis, the Higgs Boson $p_\zeta - 0.85 \, p_\zeta^{\text{vis}}$
spectrum aligns very closely with the $Z\to\tau\tau$ distribution shown here.
     }
     \label{fig:htt_pZeta}
\end{figure*}

\subsection{Categorization}

Selected events are split into three mutually exclusive categories per decay channel.
The categories are designed to target different aspects of the gluon fusion and VBF Higgs Bosont production mechanisms.
In each category the two variables that maximize the $\PH\to\Pgt\Pgt$ sensitivity are chosen to build 
two-dimensional (2D) distributions.

The three categories are defined as:
\begin{itemize}
\item {0-jet}: This category targets Higgs Boson events produced via gluon fusion.
The two variables chosen to extract the results are $\mvis$ and
the reconstructed $\tauh$ decay mode in the $\Pgm\tauh$ and $\Pe\tauh$ decay channels.
In the $\Pe\Pgm$ channel, $\mvis$ and the $\pt$ of the muon are used.  The $\PZ\to\ell\ell$ background 
is large in the 1-prong and 1-prong + $\PGpz$(s) $\tauh$ decay modes in the
$\Pgm\tauh$ and $\Pe\tauh$ channels.  By using only $\mvis$ instead of $\mtt$ the \etvecmiss
does not smear the mass reconstruction for $\PZ\to\ell\ell$ and retains a nice sharp peak.
The shape Z $\mvis$ peak of $\PZ\to\ell\ell$ provides 
an excellent handel to distinguish $\PZ\to\ell\ell$ from the other background process and
helps constrain the associated uncertainties for electrons and muons faking $\tauh$.
As electrons and muons almost exclusively fake 1-prong 1-prong + $\PGpz$(s) $\tauh$,
the reconstructed $\tauh$ decay mode is used as the second of the 2D variables in the
$\Pgm\tauh$ and $\Pe\tauh$ channels.  Additionally, the lack of $\PZ\to\ell\ell$ in the
3-prong decay mode results in increased signal significance.
Examples of the 2D distributions for the signal and $\PZ\to\ell\ell$ background
in the 0-jet category of the $\Pgm\tauh$ decay channel are shown in Fig.~\ref{fig:htt_2Dcategories} (top).
In the $\tauh\tauh$ decay channel, only one observable, $\mtt$, is considered because of the low i
event yields due to the relatively high $\pt$ thresholds on the $\tauh$ at trigger level, and 
because of the sharply falling $\tauh$ $\pt$ distribution.  Simulations indicate that about 98\% 
of signal events in the 0-jet category correspond to Higgs Bosons produced via the gluon 
fusion production mechanism.\\

\item {VBF}: This category targets Higgs boson events produced via the VBF process.
The presence of jets from the hard scattering process in VBF production leads the study to heavily
utilize jet kinematics and the jet topology in the VBF category.
Events are selected with at least two (exactly two) jets with $\pt>30$\GeV in the
$\tauh\tauh$, $\Pgm\tauh$, and $\Pe\tauh$ ($\Pe\Pgm$) channels.
In the $\Pgm\tauh$, $\Pe\tauh$, and $\Pe\Pgm$ channels, the two leading jets are required to have 
an invariant mass, $\mjj$, larger than 300\GeV. The variable $\pth$, defined as the magnitude 
of the vectorial sum of the $\ptvec$ of the visible decay products of the $\Pgt$ leptons 
and $\etvecmiss$, is required to be greater than 50 (100)\GeV in the $\Pgm\tauh$
 and $\Pe\tauh$ ($\tauh\tauh$) channels to reduce the contribution from $\PW+\text{jets}$ 
backgrounds. This selection criterion also suppresses the background from quantum 
chromodynamics (QCD) multijet events. In addition, the $\pt$ threshold on the $\tauh$ 
candidate is raised to 40\GeV in the $\Pgm\tauh$ channel, and the two leading jets in the 
$\tauh\tauh$ channel should be separated in pseudorapidity by $\Delta\eta>2.5$. The $\Delta\eta>2.5$
cut in the $\tauh\tauh$ channel significantly reduces the contributions QCD events at the cost
of very few signal events because of the large jet $\Delta\eta$ in true VBF events.
The two observables used in the VBF category are $\mtt$ and $\mjj$ for all channels. Example 2D 
distributions for the signal and $\PZ\to\Pgt\Pgt$ background
in the VBF category of the $\Pgm\tauh$ decay channel are shown in Fig.~\ref{fig:htt_2Dcategories} (center). 
Integrating over the whole $\mjj$ phase space, up to 57\% of the signal events in the VBF 
category are produced in the VBF production mode, but this proportion increases with $\mjj$ allowing
for signal production process discrimination in the highest $\mjj$ ranges.\\

\item {Boosted}: This category contains all selected events that do not enter one of the previous 
categories, namely events with one jet and events with several jets that fail the specific requirements of the VBF category.
The Boosted category contains a mix of gluon fusion events produced in association with one or more jets (78--80\% of signal events),
VBF events where one of the jets escaped detection or has low $\mjj$ (11--13\%), as well as
Higgs bosons produced in association with a $\PW$ or a $\PZ$ boson decaying hadronically (4--8\%).
Because these gluon fusion events failed the 0-jet category, the Higgs Boson will be recoiling 
off of one or more jets making $\pth$ a natural choice for the second distribution variable with
$\mtt$ as the other of the 2D variables. 
Most background processes, including $\PW+\text{jets}$ and QCD multijet events, typically have low $\pth$. 
Example 2D distributions for the signal and $\PW+\text{jets}$ background in the boosted category of 
the $\Pgm\tauh$ decay channel are shown in Fig.~\ref{fig:htt_2Dcategories} (bottom).
\end{itemize}

In figure~\ref{fig:htt_2Dcategories}, the background processes are chosen for illustrative 
purpose for their separation from 
the signal. The $\PZ\to\Pgm\Pgm$ background in the 0-jet category is concentrated in 
the regions where the visible mass is close to 90\GeV and is negligible when the $\tauh$ 
candidate is reconstructed in the 3-prong decay mode. The $\PZ\to\Pgt\Pgt$ background in 
the VBF category mostly lies at low $\mjj$ values whereas the distribution of VBF signal 
events extends to high $\mjj$ values. In the boosted category, the W+jets background, 
which behaves similarly to the QCD multijet background, is rather flat with respect to $\mtt$, and 
is concentrated at low $\pth$ values.

The three categories and the variables used to build the 2D distributions are summarized in
Table~\ref{tab:htt_categories}. 

\begin{figure*}[htbp]
\label{fig:htt_2Dcategories}
\centering
     \includegraphics[width=0.4\textwidth]{higgs_to_taus/plots/Figure_001-a.pdf}
     \includegraphics[width=0.4\textwidth]{higgs_to_taus/plots/Figure_001-b.pdf}\\
     \includegraphics[width=0.4\textwidth]{higgs_to_taus/plots/Figure_001-c.pdf}
     \includegraphics[width=0.4\textwidth]{higgs_to_taus/plots/Figure_001-d.pdf}\\
     \includegraphics[width=0.4\textwidth]{higgs_to_taus/plots/Figure_001-e.pdf}
     \includegraphics[width=0.4\textwidth]{higgs_to_taus/plots/Figure_001-f.pdf}
     \caption{Two dimensional unit normalized distributions for Higgs Boson events passing event selection 
for each category are shown (left). The specific Higgs Boson process shown is the one 
specifically targeted for each category. Distributions for select dominant background 
processes are shown (right). The rows correspond to the three categories: 0-jet (top), 
VBF (center), and boosted (bottom).  All distributions are from the $\Pgm\tauh$ decay channel.} 
\end{figure*}


\begin{table*}
\centering
\begin{small}
\begin{tabular}{llll}
 & 0-jet & VBF & Boosted \\
\hline
 & \multicolumn{3}{c}{Selection} \\ \cline{2-4}
$\tauh\tauh$ & No jet &  \scriptsize{$\geq$2 jets, $\pth>100\GeV$, $\Delta\eta_{\mathrm{jj}}>2.5$} & Others\\
$\Pgm\tauh$ & No jet &  \scriptsize{$\geq$2 jets, $\mjj>300\GeV$, $\pth>50\GeV$, $\pt^{\tauh}>40\GeV$} & Others\\
$\Pe\tauh$ & No jet &  \scriptsize{$\geq$2 jets, $\mjj>300\GeV$, $\pth>50\GeV$} & Others\\
$\Pe\Pgm$ & No jet & \scriptsize{2 jets, $\mjj>300\GeV$} & Others \\
\hline
 & \multicolumn{3}{c}{Observables}\\ \cline{2-4}
$\tauh\tauh$ & $\mtt$                 &    $\mjj$, $\mtt$  &   $\pth$, $\mtt$  \\
$\Pgm\tauh$ & $\tauh$ decay mode, $\mvis$   &    $\mjj$, $\mtt$  &  $\pth$, $\mtt$  \\
$\Pe\tauh$ & $\tauh$ decay mode, $\mvis$   &    $\mjj$, $\mtt$  &  $\pth$, $\mtt$ \\
$\Pe\Pgm$ & $\pt^{\Pgm}$, $\mvis$   &     $\mjj$, $\mtt$  &   $\pth$, $\mtt$  \\
\hline
\end{tabular}
\caption{ Category selection and observables used to build the 2D kinematic distributions. 
The events failing the 0-jet and VBF selection are included in the boosted category and are
denoted by ``Others''.
\label{tab:htt_categories}
}
\end{small}
\end{table*}

The results of the analysis are extracted with a global maximum likelihood fit based on  the 2D distributions in the various signal regions, and on some control regions, detailed in Section~\ref{sec:background_estimation}, that constrain the normalizations of the main backgrounds.

\section{Data Set}
The $\htt$ study utilizes the full 2016 $\pp$ dataset collected by CMS corresponding to 35.9$\fbinv$ 
of integrated luminosity.  The data were gathered at center-of-mass energy 13 TeV data from the LHC.
As data are gathered at CMS, the condition of the CMS detector are recorded simultaneously.
The CMS experiment uses an offline validation process to ensure that only high quality
data is used in future analyses.  Data collected while the CMS detector is in a faulty state
is flagged as such and, when an analyst processes the data, the faulty data is skipped.
When only considering the data considered good and usable for physics, the CMS experiment
gathered 35.9$\fbinv$ of integrated lumiosity.

In this analysis only a subset of the total data marked as good is used.  During data
gathering, data is filtered into primary datasets (PDs) corresponding to which HLT trigger
made the determination to save a given event.  The HLT triggers using in this analysis,
detailed in table~\ref{tab:htt_hlt_triggers}, correspond to: the Single Electron/Photon
PD, the Single Muon PD, the Muon and Electron PD, and the Tau PD.



\section{Monte Carlo Samples}
Signal and background processes are modeled with samples of simulated events.
For details on the production for simulated events, see section~\ref{sec:simulation}.

The signal samples with a Higgs Boson produced vai the gluon fusion production
mechanism ($\cPg\cPg\PH$), vector boson fusion (VBF), or in association with a $\PW$ or $\PZ$ boson ($\PW\PH$ or $\PZ\PH$), 
are generated at next-to-leading order (NLO) in perturbative quantum chromodynamics (pQCD) 
with the \POWHEG 2.0~\cite{Nason:2004rx,Frixione:2007vw, Alioli:2010xd, Alioli:2010xa, Alioli:2008tz} 
generator. For the $\PW\PH$ and $\PZ\PH$ simulated samples, the \textsc{minlo hvJ}~\cite{Luisoni:2013kna} 
extension of \POWHEG 2.0 is used. The set of parton distribution functions (PDFs) used for all
simulates samples is NNPDF30\_nlo\_as\_0118~\cite{Ball:2011uy}. It is found that the $\ttbar\PH$ 
process is negligible in comparison with the other signal processes and is therefor excluded.
Mass dependant production cross sections and $\htt$ branching fractions for the SM Higgs Boson production, 
and their corresponding uncertainties are taken from 
Refs.~\cite{deFlorian:2016spz,Denner:2011mq,Ball:2011mu} and references therein.

The $\text{MG5}_\aMCATNLO$~\cite{Alwall:2014hca} generator is used for $\PZ+$jets and $\PW+$jets processes. 
They are simulated at leading order (LO) with MLM jet matching and merging~\cite{Alwall:2007fs}.
The \aMCATNLO generator is also used for diboson production simulated at next-to-LO (NLO) with the 
FxFx jet matching and merging~\cite{Frederix:2012ps}. In contrast, \POWHEG 2.0 and 1.0 are used for \ttbar
and single top quark production, respectively. The generators are interfaced with \PYTHIA 
8.212 ~\cite{Sjostrand:2014zea} to model the parton showering and fragmentation, as well as 
the decay of the $\Pgt$ leptons. The \PYTHIA parameters affecting the description of the 
underlying event are set to the {CUETP8M1} tune~\cite{Khachatryan:2015pea}.

A event-by-event weight is applied to the simulated events such that the distribution of the 
number of additional pileup interactions in the simulated sample more closely aligns with that in data.
The distribution of additional pileup interactions in data is estimated from the measured instantaneous 
luminosity for each bunch crossing. The average number of additional pileup interactions in
data is approximately 27 interactions per bunch crossing.



\section{SVFit Algorithm}



\section{Background Estimation}
The largest irreducible source of background is the Drell--Yan production
of $\PZ/\Pgg^*\to\Pgt\Pgt, \ell\ell$. Proper shape and normalization for this
leading background is critical is critical to the success of the analysis.
A dedicated, high purity, $\PZ/\Pgg^*\to\Pgm\Pgm$ control region was
used to measure reweighting factors for application to all Drell--Yan
simulated events and is detailed in the Monte Carlo Corrections 
section~\ref{sec:mc_corrections}.

The $\PW+\text{jets}$ process is modelled using simulation.
In the $\tauh\tauh$ and $\Pe\Pgm$ channels the $\PW+\text{jets}$ contribution 
is small compared to other backgrounds. In these two channels both the shape and 
yield are taken from simulation.
The background from $\PW+\text{jets}$ production contributes significantly to the
$\Pgm\tauh$ and $\Pe\tauh$ channels, when the $\PW$ boson decays leptonically and
a jet is misidentified as a $\tauh$ candidate. In the $\ell\tauh$ channels
the shape of the background is simulation based while the yields are estimated 
using data in a $\PW+\text{jets}$ enriched dedicated side-band region defined
based on transverse mass, $\MT > 80\GeV$. In this high-$\MT$ side-band region, the $\PW+\text{jets}$
process is scaled so that the sum total of expected background events is equivalent
to the sum total of observed data events. The scaling factor is then applied
in the low transverse mass signale regions $\MT < 50\GeV$. A scaling factor
is measured for the 0-jet and Boosted categories for both $\Pe\Pgt$ and $\Pgm\Pgt$ resulting
in four $\PW+\text{jets}$ scaling values. The scaling factor measured in the Boosted
category is extrapolated to the VBF category. The $\PW+\text{jets}$ purity in these
side-band regions varies from about 50\% in the boosted category to 85\% in the 0-jet category.

The high-$\MT$ side-bands, described above, are included as a control regions in the final fit.
These high-$\MT$ $\PW+\text{jets}$ control regions can be seed in Figure~\ref{fig:htt_wj_CR1}.
The control regions are only a singular bin because they are used solely to constrain 
the normalization of the $\PW+\text{jets}$ process.

\begin{figure*}[!htbp]
\centering
     \includegraphics[width=0.19\textwidth]{higgs_to_taus/plots/Figure_002-a.pdf}
     \includegraphics[width=0.19\textwidth]{higgs_to_taus/plots/Figure_002-b.pdf}
     \includegraphics[width=0.19\textwidth]{higgs_to_taus/plots/Figure_002-c.pdf}
     \includegraphics[width=0.19\textwidth]{higgs_to_taus/plots/Figure_002-d.pdf}
     \includegraphics[width=0.19\textwidth]{higgs_to_taus/plots/Figure_002-e.pdf}
     \caption{the high-$\MT$ control regions enriched in the $\PW+\text{jets}$ background used in 
the maximum likelihood fit, together with the signal regions, to extract the results. 
The normalization of the predicted background distributions corresponds to the result of 
the global fit. These regions, defined with $\MT>80$\GeV, control the yields of the 
$\PW+\text{jets}$ background in the $\Pgm\tauh$ and $\Pe\tauh$ channels.  
The constraints obtained in the boosted categories are propagated to the VBF categories 
of the corresponding channels.}
     \label{fig:htt_wj_CR1}
\end{figure*}


The QCD multijet process constitutes an important source of reducible background 
in the $\tauh\tauh$ and $\ell\tauh$ channels. The QCD multijet process is subdominant
in the $\Pe\Pgm$ channel. For all four channels, the QCD multijet, is entirely estimated from data.  
Side-band control regions are constructed to estimate the shape and the yield of the QCD multijet 
background in these channels.

QCD multijet estimation for the $\ell\tauh$ channels follows:
\begin{enumerate}

\item Raw yield is extracted using a sample where the
$\ell$ and the $\tauh$ candidates have the same-sign charge. Using this sample, the QCD multijet 
process is estimated from data by subtracting the contribution of the Drell--Yan, \ttbar, diboson,
and $\PW+\text{jets}$ processes as in equation~\ref{eqn:qcd_eqn}. The $\PW+\text{jets}$ processes has been scaled by the
$\PW+\text{jets}$ high-$\MT$ to low-$\MT$ scale factor described previously.
\begin{equation}
\label{eqn:qcd_eqn}
\text{QCD} = \text{Data} - \text{Other Background Processes}
\end{equation}

\item The raw yield obtained above is corrected to account for observed differences between the background 
composition in the same-sign and opposite-sign regions. The extrapolation factor between the same-sign 
and opposite-sign regions is determined by comparing the yield of the estimated QCD multijet background for 
events with $\ell$ candidates passing inverted isolation criteria, in the same-sign and opposite-sign 
regions. It is constrained and measured by adding to the global fit the opposite-sign region where 
the $\ell$ candidates pass inverted isolation criteria, using the QCD multijet background estimate 
from the same-sign region with $\ell$ candidates passing inverted isolation criteria. For the same 
reasons as in the case of the W+jets background, the constraints are also extrapolated to the VBF 
signal region. 
%Figure~\ref{fig:htt_qcd_CR3} shows these control regions for the 0-jet and boosted categories 
%of the $\Pgm\tauh$ and $\Pe\tauh$ channels; the observable is $\mvis$ or $\mtt$ to provide 
%discrimination between the QCD multijet and the $\PZ\to\Pgt\Pgt$ processes.

\item The 2D distributions of the QCD multijet background are estimated from a region with 
same-sign leptons, same as the yield estimate, except the isolation of the $\ell$ and $\tauh$ 
candidates is additionally relaxed to reduce the statistical fluctuations in the distributions. 
The contributions from other background process are subtracted from data, eqn.~\ref{eqn:qcd_eqn},
to extract the QCD multijet shape template in this region.
\end{enumerate}

The same technique is used in the $\Pe\Pgm$ channel, except no control region is included in the 
fit because QCD multijet events contribute little to the total background in this decay channel.

%\begin{figure*}[!htbp]
%\centering
%     \includegraphics[width=0.3\textwidth]{higgs_to_taus/plots/Figure_003-a.pdf}
%     \includegraphics[width=0.3\textwidth]{higgs_to_taus/plots/Figure_003-b.pdf}
%     \includegraphics[width=0.3\textwidth]{higgs_to_taus/plots/Figure_003-c.pdf}
%     \includegraphics[width=0.3\textwidth]{higgs_to_taus/plots/Figure_003-d.pdf}
%     \includegraphics[width=0.3\textwidth]{higgs_to_taus/plots/Figure_003-e.pdf}
%     \caption{Control regions enriched in the QCD multijet background used in the maximum likelihood fit, together with the signal regions, to extract the results. The normalization of the predicted background distributions corresponds to the result of the global fit. These regions, defined by selecting events with opposite-sign $\ell$ and $\tauh$ candidates with $\ell$ passing inverted isolation conditions,  control the
%yields of the QCD multijet background in the $\Pgm\tauh$ and $\Pe\tauh$ channels.  The constraints obtained in the boosted categories are propagated to the VBF categories of the corresponding channels.}
%     \label{fig:htt_qcd_CR3}
%\end{figure*}

In the $\tauh\tauh$ channel, the large QCD multijet background is estimated differently than
the other channels. Comparing the QCD multijet shape templates derived using equation~\ref{eqn:qcd_eqn} 
in the same-sign region to the opposite-sign region shows inconsistencies between the expected shapes.
The Kolmogorov-Smirnov (KS) test was used to assess the level of agreement or disagreement
between the two compared QCD multijet templates. The KS test is a useful tool for comparing
the compatibility of two shape templates and is a nonparametric test of the compatibility
of continuous, one-dimensional probability distributions. Specifically, this study relied
on a two sample, binned KS test which tested the compatibility of the QCD multijet
estimated background in the same-sign region with that estimated in the opposite-sign region.
In addition, QCD multijet templates templates where tested for compatibility with both templates
defined in the same-sign region, but having differeing isolation requirement. This same check
was also done in the opposite-sign region. Table~\ref{tab:htt_qcd_ks} summarizes the results of
these KS test comparisons. It can be seen that QCD multijet shapes derived in the same-sign region
are not compatible with QCD multijet shapes derived in the opposite-sign region. And, QCD multijet
shapes derived using differing isolation requirements in the opposite-sign region show agreement
with other shapes derived in the opposite-sign region. For this reason we estimate the QCD multijet
shape template using a relaxed isolation opposite-sign side-band region.

\begin{table}[h!]
\begin{center}
{\footnotesize
\begin{tabular}{|c|c|c|}
\hline
Sign Configuration & $\tau_{h,2}$ Isolation Cuts & KS Test Value \\
\hline
%SS vs. OS & $\tau_{h,2}$!=VTight \&\& $\tau_{h,2}$==Tight & 0.033 \\
%SS vs. OS & $\tau_{h,2}$!=Tight \&\& $\tau_{h,2}$==Medium & 0.003 \\
%SS vs. OS & $\tau_{h,2}$!=Medium \&\& $\tau_{h,2}$==Loose & $<$0.001 \\
%SS vs. OS & $\tau_{h,2}$!=VTight \&\& $\tau_{h,2}$==Loose & 0.001 \\
SS vs. OS & Not VTight and Passes Tight & 0.033 \\
SS vs. OS & Not Tight and Passes Medium & 0.003 \\
SS vs. OS & Not Medium and Passes Loose & $<$0.001 \\
SS vs. OS & Not VTight and Passes Loose & 0.001 \\
\hline
\end{tabular}
\begin{tabular}{|c|c|c|c|}
\hline
Charge Config. & $\tau_{h,2}$ Iso. Cuts Shape 1 & $\tau_{h,2}$ Iso. Cuts Shape 2 & KS Test Value \\
\hline
%OS & $\tau_{h,2}$!=Tight \&\& $\tau_{h,2}$==Medium & $\tau_{h,2}$!=VTight \&\& $\tau_{h,2}$==Tight & 0.168 \\
%OS & $\tau_{h,2}$!=Medium \&\& $\tau_{h,2}$==Loose & $\tau_{h,2}$!=Tight \&\& $\tau_{h,2}$==Medium & 0.104 \\
%OS & $\tau_{h,2}$!=Medium \&\& $\tau_{h,2}$==Loose & $\tau_{h,2}$!=VTight \&\& $\tau_{h,2}$==Tight & 0.543 \\
OS & Not Tight and Passes Medium & Not VTight and Passes Tight & 0.168 \\
OS & Not Medium and Passes Loose & Not Tight and Passes Medium & 0.104 \\
OS & Not Medium and Passes Loose & Not VTight and Passes Tight & 0.543 \\
\hline
%SS & $\tau_{h,2}$!=Tight \&\& $\tau_{h,2}$==Medium & $\tau_{h,2}$!=VTight \&\& $\tau_{h,2}$==Tight & 0.587 \\
%SS & $\tau_{h,2}$!=Medium \&\& $\tau_{h,2}$==Loose & $\tau_{h,2}$!=Tight \&\& $\tau_{h,2}$==Medium & 0.036 \\
%SS & $\tau_{h,2}$!=Medium \&\& $\tau_{h,2}$==Loose & $\tau_{h,2}$!=VTight \&\& $\tau_{h,2}$==Tight & 0.029 \\
SS & Not Tight and Passes Medium & Not VTight and Passes Tight & 0.587 \\
SS & Not Medium and Passes Loose & Not Tight and Passes Medium & 0.036 \\
SS & Not Medium and Passes Loose & Not VTight and Passes Tight & 0.029 \\
\hline
\end{tabular}
}
\end{center}
\caption{
Kolmogrov-Smirnov test results comparing estimated QCD multijet shape templates to test shape compatibility
between different possible QCD multijet shape estimation regions. The $\tauh$ MVA isolation selection
for each comparison is listed where ``VTight'' means \texttt{Very Tight Tau MVA}, ``Tight'' means \texttt{Tight Tau MVA},
``Medium'' means \texttt{Medium Tau MVA}, and ``Loose'' means \texttt{Loose Tau MVA}.
In these comparisons the isolation of the highest $\pt$ $\tauh$, $\tau_{h,1}$,
is kept at \texttt{Medium Tau MVA} isolation.  
The upper table shows comparisons between same-sign (SS) and opposite-sign (OS) selections for the same
exact isolation selection. The lower table shows KS test results for comparisons
within the same charge configuration across different isolation requirements.
}
\label{tab:htt_qcd_ks}
\end{table} 

The selection for estimating the QCD multijet background shape template and raw yeild requires 
that at least one of the $\tauh$
must fail the \texttt{Tight Tau MVA} requirement which is the isolation selection used to 
define the signal region. This selection
ensures that the side-band sample is disjoin from the signal region.
In this region, the QCD multijet background shape and raw yield are obtained
by subtracting the contribution of the Drell--Yan, \ttbar, and $\PW+\text{jets}$ processes, 
estimated from simulation as explained above, from the data as in eqn~\ref{eqn:qcd_eqn}.

A scaling factor is required to adjust the raw yield estimated above to the expected
QCD multijet yield in the signal extraction region. This extrapolation factor
is estimated in the same-sign charge region. Two same-sign QCD multijet samples are constructed
using 1) the exact same isolation requirements as the signal region; all selections are identical
here except for the $\tauh$ charge configuration. And, 2) a second region with the exact 
same relaxed isolation requirements as the region used to estimate the opposite-sign
QCD multijet shape template and raw yield. From these two samples a ``Relaxed-Region-to-Signal-Region''
scaling factor is calculated as,
\[ \text{Relaxed-Region-to-Signal-Region SF} = \frac{\text{Signal Region Yield}}{\text{Relaxed Region Yield}} \]
Multiplying the raw yield estimated above by this scaling factor results in the estimated
QCD multijet contribution in the signal region. The ``Relaxed-Region-to-Signal-Region'' scale factors
for the three $\tauh\tauh$ categories are listed in table~\ref{tab:htt_qcd_sf}. The uncertainties associated
with these scale factors are propagated through the QCD multijet estimation process and combined
with results from closure tests to estimate a final category dependant systematic and statistic uncertainty
for the QCD multijet estimation process.

\begin{table*}[htbp]
\centering
\begin{tabular}{|l|c|}
\hline
Category   &   ``Relaxed-Region-to-Signal-Region''   \\
           &   Scale Factor   \\
\hline
0-jet      &    0.25 $\pm$ 1.0\%  \\
VBF        &    0.18 $\pm$ 10\%  \\
Boosted    &    0.23 $\pm$ 1.4\%  \\  
\hline
\end{tabular}
\label{tab:htt_qcd_sf}
\caption{
The category dependant scale factors used to adjust the QCD multijet yield to correspond 
to the expected yield in the signal region. The large uncertainty on the VBF scale factor show the
limited amount of QCD multijet events in the VBF same-sign region. 
}
\end{table*}

The events selected with opposite-sign $\tauh$ candidates passing relaxed isolation requirements 
form control regions, shown in Fig.~\ref{fig:htt_qcd_CR4}, and are used in the global fit to extract the results.

\begin{figure*}[!htbp]
\centering
     \includegraphics[width=0.24\textwidth]{higgs_to_taus/plots/Figure_004-a.pdf}
     \includegraphics[width=0.24\textwidth]{higgs_to_taus/plots/Figure_004-b.pdf}
     \includegraphics[width=0.24\textwidth]{higgs_to_taus/plots/Figure_004-c.pdf}
     \includegraphics[width=0.24\textwidth]{higgs_to_taus/plots/Figure_004-d.pdf}
     \caption{Control regions enriched in the QCD multijet background used in the maximum likelihood fit, 
together with the signal regions, to extract the results. The normalization of the predicted background 
distributions corresponds to the result of the global fit. These regions, a formed by selecting events with 
opposite-sign $\tauh$ candidates passing relaxed isolation requirements with at least one of them
failing \texttt{Tight Tau MVA} isolation. These regions control the yields of the QCD multijet background 
in the $\tauh\tauh$ channel.}
     \label{fig:htt_qcd_CR4}
\end{figure*}



The \ttbar production process is one of the main backgrounds in the $\Pe\Pgm$ channel.
In all channels \ttbar is predicted from simulation. The normalization is adjusted in
A \ttbar-enriched sample orthogonal to the signal region provides a control region which is included
in the global fit. The \ttbar control region is defined from the $\Pe\Pgm$ channel. The yield
of \ttbar in all channels and categories is adjusted by the \ttbar control region.
This control region, shown in Fig.~\ref{fig:htt_ttbar_CR2},
The selection enforming orthagonality from the $\Pe\Pgm$ signal region is inverting the $p_\zeta$ requirement
and the events should contain at least one jet.

\begin{figure}[htb]
\centering
     \includegraphics[width=0.23\textwidth]{higgs_to_taus/plots/Figure_005-a.pdf}
     \includegraphics[width=0.23\textwidth]{higgs_to_taus/plots/Figure_005-b.pdf}
     \caption{Control region enriched in \ttbar background, used in the maximum likelihood fit, 
together with the signal regions, to extract the results. The normalization of the predicted background 
distributions corresponds to the result of the global fit. This region, defined by inverting the 
$p_\zeta$ requirement and rejecting events with no jet in the $\Pe\Pgm$ final state, is used to estimate the
yields of the \ttbar background in all channels.}
     \label{fig:htt_ttbar_CR2}
\end{figure}

Other background processes, such as, contributions from diboson and single top quark production, are estimated 
from simulation.

\pagebreak

\section{Monte Carlo Corrections}
\label{sec:mc_corrections}

Corrections are applied to the simulated Monte Carlo samples to help correct for measured differences
between observed data and expectations based on simulation. Many of these corrections are designed
to correct differences in reconstruction and identification efficiencies for objects between data
and simulation. These corrections are derived in
fully orthogonal regions from the analysis signal regions when ever possible. There are however many
instances where the $\Pgm\tauh$ channel is used to derive corrections. In these cases the measurement
region used to derive corrections will overlap with a subset of the analysis signal regions. This
overlap is discussed in the Systematic Uncertainties section~\ref{sec:htt_systematics}.


\subsection{Pileup Reweighting}
During 2016 data taking, the LHC delivered \pp collisions to CMS with an average of approximately 27
interactions per bunch crossing. Monte Carlo samples are generated with additional pileup interactions
added to the hard-scattering process for each event. A reweighting technique is used which improves
the aligment between the pileup interactions in data and those in Monte Carlo. This is critical because
there are some reconstructed event characteristics such as \etvecmiss which can be influenced by the
number of pileup interactions per event.


\subsection{Tau Identification Efficiency}
The reconstruction efficiency for hadronically decaying taus is observed to be different in data and in simulated samples.
Correction factors are derived by the Tau POG using the $\Pgm\tauh$ channel using a tag-and-probe method. Essentially,
the Tau ID Efficiency measurement selects $\PZ/\Pgg^*\to\Pgt\Pgt\to\Pgm\tauh$ events on the $\PZ$
mass peak and performs a fit with the $\PZ\to\Pgm\tauh$ process treated as a signal. The degree
to which the $\PZ\to\Pgm\tauh$ process is scaled up to down is the resulting Tau ID Efficiency
scale factor. The measured scale factor is 0.95 $\pm$ 0.05 for all $\tauh$ passing \texttt{Tight Tau MVA}.
This correction factor is applied to all simulated $\tauh$ which are matched at the generator
level to $\tau$ leptons.


\subsection{Tau Energy Correction}
The energy of each simulated $\tauh$, which matches to a $\tau$ at the generator level, is corrected
to better represent the observed energy of $\tauh$ in data. The correction is measured and applied
as a function of $\tauh$ decay mode with each correction listed in table~\ref{tab:htt_tec}. The measurement
of this correction is performed in the $\Pgm\tauh$ final state by the Tau POG. The effect of
the shifted $\tauh$ energy is fully propagated through to the \etvecmiss for each event.

\begin{table*}[htbp]
\centering
\begin{tabular}{|l|c|}
\hline
$\tauh$ Decay Mode   &   Energy Correction   \\
\hline
1-prong            &   -1.8\%  $\pm$ 1.2\%  \\
1-prong+$\PGpz$    &   +1.0\%  $\pm$ 1.2\%  \\
3-prong            &   +0.4\%  $\pm$ 1.2\%  \\  
\hline
\end{tabular}
\label{tab:htt_tec}
\caption{
Energy corrections applied to simulated true $\tauh$. The energy corrections are measured
and applied depending on the reconstructed decay mode of the $\tauh$.
}
\end{table*}



\subsection{Lepton Identification and Isolation Efficiencies}
Similar to the Tau ID Efficiency measured above, the electron and muon reconstruction, identification,
and isolation efficiencies are measured in both data and simulation. Correction factors equal to
$\epsilon_{data}/\epsilon_{MC}$ are derived as a function of both lepton $\pt$ and lepton $\eta$.
The electron (muon) efficiencies are measured in $\PZ\to\Pe\Pe$ ($\PZ\to\Pgm\Pgm$) events using a tag-and-probe
method.


\subsection{Trigger Efficiencies}
Trigger efficiencies are measured in data and in simulation for all of the channels. For the channels
which trigger on $\tauh$, the $\Pgm\tauh$ and $\tauh\tauh$ channels, the $\tauh$ efficiency is measured
using tag-and-probe in the $\Pgm\tauh$ final state. The efficiencies for the $\tauh\tauh$ channel
are measured with dedicated muon+$\tauh$ monitoring triggers. The $\tauh$ trigger requirements
are identical between the monitoring trigger and the $\tauh\tauh$ trigger paths. The tag-and-probe
is conducted using the Single Muon PD. The selection for the $\tauh$ trigger efficiency measurements 
requires one well identified and isolated muon per event. The muon is required to fire a single muon trigger.
Events with at least one $\tauh$ which passes \texttt{decay mode finding} and \texttt{Tight Tau MVA} isolation
constitute the denominator selection for the efficiency measurement. Events where the $\tauh$
fires the trigger under study are recorded as passing events. The ratio of passing events to
denominator events defines the trigger efficiency. Trigger efficiencies for are measured and applied as 
a function of $\tauh$: $\pt$, decay mode, and generator matching status. Figure~\ref{fig:htt_tt_trig}
shows the trigger efficiency versus $\pt$ for the double-$\tauh$ trigger.

\begin{figure*}[!htbp]
\centering
     \includegraphics[width=0.65\textwidth]{higgs_to_taus/plots/htt_tautau_trigger_efficiency.pdf}
     \caption{
Comparison of the trigger efficiencies for the \texttt{HLT\_DoubleMediumIsoPFTau35\_Trk1\_eta2p1\_Reg}
which was active in data during eras B-G. The efficiency was measured using tag-and-probe in the
$\Pgm\tauh$ final state. The efficiency is measured for a single leg of the double-$\tauh$ at a time
with a muon+$\tauh$ monitoring trigger. The efficiencies shown here are used in the $\tauh\tauh$ channel.
The specific distribution shown is for real $\tau$ leptons and includes all used decay modes.
}
     \label{fig:htt_tt_trig}
\end{figure*}

The trigger efficiencies for electrons and muons are also measured with a tag-and-probe method.
Similar to the electron and muon identification and isolation efficiencies, the 
electron (muon) efficiencies trigger are measured in $\PZ\to\Pe\Pe$ ($\PZ\to\Pgm\Pgm$) events.
The denominator selection for the probe lepton requires that the same identification and
isolation criteria used in the analysis be met. The efficiencies are measured as a function of
lepton $\pt$ and $\eta$.



\subsection{Drell--Yan Reweighting}

The simulated Monte Carlo Drell--Yan sample used in the analysis is simulated at leading order (LO)
using the \aMCATNLO generator discussed in section~\ref{sec:simulation}. To assess Monte Carlo to
data agreement we use a dedicated $\PZ/\Pgg^*\to\Pgm\Pgm$ control region with high Drell--Yan purity.
The $\PZ/\Pgg^*\to\Pgm\Pgm$ control region uses events from the Single Muon PD and requires the 
presence of two well-identified muons with $\pt > 25 \GeV$ and an invariant mass between 70 and 110\GeV.
With this selection criteria over 99\% of the events are attributed to $\PZ/\Pgg^*\to\Pgm\Pgm$ decays.

Differences in the distributions of $m_{\ell\ell/\Pgt\Pgt}$ and $\pt(\ell\ell/\Pgt\Pgt)$ between data and 
in simulations are observed in this control region. To correct for this, 2D weights based on these variables 
are derived and applied to simulated $\PZ/\Pgg^*\to\Pgt\Pgt, \ell\ell$ events in the signal region of the analysis. 
In addition, corrections depending on $\mjj$ are derived from the $\PZ/\Pgg^*\to\Pgm\Pgm$ region and 
applied to the $\PZ/\Pgg^*\to\Pgt\Pgt, \ell\ell$ simulation for events with at least two jets passing the 
VBF category selection criteria. After this reweighting, good agreement between data in the 
$\PZ/\Pgg^*\to\Pgm\Pgm$ region and simulation is found for all other variables used in the analysis.
These derived corrections are also applied to the simulated samples of electroweak production of $\PZ$ 
bosons. Electroweak produced $\PZ$ contributes up to 8\% of the $\PZ$ boson production in the VBF category.



\subsection{Hadronic Recoil Corrections}
Recoil corrections are applied to the Drell--Yan and $\PW+\text{jets}$ simulated background events 
as well as the simulated signal events for 
Higgs Bosons produced via the gluon fusion or VBF mechanisms. Recoil corrections correct for the
mismodeling of \etvecmiss in the simulated samples. A variable, $\vec{U}$, is defined as the 
vectorial difference of the measured \etvecmiss and the total transverse momentum of neurtrinos
originating from the decay of the boson.
\begin{equation}
\vec{U} = \etvecmiss - \vec{p}_{\text{T},\nu}
\end{equation}

For $\PZ$ bosons decaying leptonically to $\ell$, $\vec{U}$ can be expressed as,

\begin{equation}
\vec{U} = \vec{p}_{\text{T,B}} - \vec{H}_{\text{T}}
\end{equation}

where $\vec{H}_{\text{T}}$ is the transverse momentum of hadronic recoil and $\vec{p}_{\text{T,B}}$
is the leptonic recoil and is the transverse momentum of leptonically decaying $\PZ$, $\PW$, or Higgs Boson.

Like the Drell--Yan Reweighting mentioned above, $\vec{U}$ is measured in $\PZ\to\Pgm\Pgm$ events.
It is observed that the $\vec{H}_{\text{T}}$ from simulation in $\PZ\to\Pgm\Pgm$ does not align with
the $\vec{H}_{\text{T}}$ from data. A reweighting method is derived to align the simulated distribution with 
data. The $\PZ/\Pgg^*\to\Pgm\Pgm$
final state is used because of the absence of neutrinos in the $\PZ$ decay making the comparison of
the simulation to data and derivation of the reweighting technique far more simple. For the measurement
and application of the corrections $\vec{U}$ is decomposed into two components, one which is
paralled to $\vec{p}_{\text{T,B}}$ ($\vec{U_{1}}$) and one which is orthogonal ($\vec{U_{2}}$).


\subsection{Generator Event Weights}
Generator weights are applied on a per event basis. Samples produced with the \aMCATNLO generator
contain both positive and negative event wieghts, as discussed in the Simulation chapter~\ref{sec:simulation}.
Negative generator weights in a sample reduces the effective statists of that sample. This reduction
in statistical power is often the trade off for increased precision in the modeling of a choosen
background or signal process.


\subsection{Luminosity}
The event weights for simulated events are scaled to the expected yields for each sample. A per-sample
weight (W) is defined based on the number of generated events (N), the best theory predicted cross section
for the process ($\sigma$), and the integrated luminosity of the dataset being modeled 
($\mathcal{L}$), 35.9\fbinv.

\begin{equation}
\text{W} \, = \mathcal{L} \, \,  \frac{\sigma}{\text{N}}
\end{equation}

For samples with negative generator weights, N is redefined as the sum total of the event weights to
properly account for the negative event weights.

\pagebreak

\section{Systematic Uncertainties}
\label{sec:htt_systematics}

\subsection{Uncertainties related to object reconstruction and identification}

The overall uncertainty in the $\tauh$ identification efficiency for genuine $\tauh$ leptons is 5\%, 
which has been measured with a tag-and-probe method in $\PZ\to\Pgt\Pgt$ events.
This number is not fully correlated among the di-$\Pgt$ channels because the $\tauh$ candidates are required to pass
different working points of the discriminators that reduce the misidentification rate of electrons and muons as $\tauh$ candidates.
The trigger efficiency uncertainty per $\tauh$ candidate amounts to an additional 5\%, which leads to a total 
trigger uncertainty of 10\% for processes estimated from simulation in the $\tauh\tauh$ decay channel. This 
uncertainty has also been measured with a tag-and-probe method in $\PZ\to\Pgt\Pgt$ events.

An uncertainty of 1.2\% in the visible energy scale of genuine $\tauh$ leptons affects both the distributions and the
signal and background yields. It is uncorrelated among the 1-prong, 1-prong + $\PGpz$, and
3-prong decay modes.
The magnitude of the uncertainty was determined in $\PZ\to\Pgt\Pgt$ events with one $\Pgt$ lepton decaying hadronically 
and the other one to a muon, by performing maximum likelihood fits for different values of the visible energy scale of 
genuine $\tauh$ leptons. Among these events, less than half overlap with the events selected in the $\Pgm\tauh$ 
channel of this analysis. The fit constrains the visible $\tauh$ energy scale uncertainty to about
0.3\% for all decay modes. The constraint mostly comes from highly populated regions with a high $\tauh$ purity, namely 
the 0-jet and boosted categories of the $\Pgm\tauh$ and $\tauh\tauh$ channels. The decrease in the size of the 
uncertainty is explained by the addition of two other decay channels
with $\tauh$ candidates ($\tauh\tauh$ and $\Pe\tauh$), by the higher number of events in the MC simulations, and by the 
finer categorization that leads to regions with a high $\PZ\to\Pgt\Pgt$ event purity.
Even in the most boosted categories, reconstructed $\tauh$ candidates typically have moderate $\pt$ ($\pt$ less than 100\GeV) and are
found in the barrel region of the detector. As tracks are well measured in the CMS detector for this range of $\pt$,
the visible energy scale of genuine $\tauh$ leptons is fully correlated for all $\tauh$ leptons reconstructed in the 
same decay mode, irrespective of their $\pt$ and $\eta$. The uncertainties in the visible energy scale for genuine 
$\tauh$ leptons together contribute an uncertainty of 5\% to the measurement of the signal strength.

In the 0-jet category of the $\Pgm\tauh$ and $\Pe\tauh$ channels, the relative contribution of $\tauh$ in a given
reconstructed decay mode is allowed to fluctuate by 3\% to account for the possibility that the reconstruction and
identification efficiencies are different for each decay mode. This uncertainty has been measured in a region enriched 
in $\PZ\to\Pgt\Pgt$ events with one $\Pgt$ lepton decaying hadronically and the other one decaying to a muon, by 
comparing the level of agreement in exclusive bins of the reconstructed $\tauh$ decay mode, after adjusting the inclusive 
normalization of the $\PZ\to\Pgt\Pgt$ simulation to its best-fit value. The effect of migration between the reconstructed 
$\tauh$ decay modes is negligible in other categories, where
all decay modes are treated together.

For events where muons or electrons are misidentified as $\tauh$ candidates, essentially $\PZ\to \Pgm\Pgm$ events in 
the $\Pgm\tauh$ decay channel and $Z\to \Pe\Pe$ events in the $\Pe\tauh$ decay channel, the $\tauh$ identification leads 
to rate uncertainties of 25 and 12\%, respectively, per reconstructed $\tauh$ decay mode. Using $\mvis$ and the reconstructed 
$\tauh$ decay mode as the observables in the 0-jet category of the $\Pgm\tauh$ and $\Pe\tauh$ channels helps reduce the 
uncertainty after the signal extraction fit: the uncertainty in the rate of muons or electrons misidentified as $\tauh$ 
becomes of the order of 5\%. The energy scale uncertainty for muons or electrons
 misidentified as $\tauh$ candidates is 1.5 or 3\%, respectively, and is uncorrelated between reconstructed $\tauh$ decay
modes. The fit constrains these uncertainties to about one third of their initial values. For events where quark- or 
gluon-initiated jets are misidentified as $\tauh$ candidates, a linear uncertainty that increases by 20\% per 100\GeV 
in $\tauh$ $\pt$ accounts for a potential mismodeling of the jet$\to\tauh$ misidentification rate as a
function of the $\tauh$ $\pt$ in simulations. The uncertainty has been determined from a region enriched in $\PW+\text{jets}$ 
events, using events with a muon and a $\tauh$ candidate in the final state, characterized by a large transverse mass 
between the $\ptmiss$ and the muon~\cite{Khachatryan:2015dfa,CMS-PAS-TAU-16-002}.

In the decay channels with muons or electrons, the uncertainties in the muon and electron identification, isolation, and 
trigger efficiencies lead to the rate uncertainty of 2\% for both muons and electrons.
The uncertainty in the electron energy scale, which amounts to 2.5\% in the endcaps and 1\% in the barrel of the detector, 
is relevant only in the $\Pe\Pgm$ decay channel, where it affects the final distributions.
In all channels, the effect of the uncertainty in the muon energy scale is negligible.

The uncertainties in the jet energy scale depend on the \pt and $\eta$ of the jet~\cite{CMS-JME-10-011}.
They are propagated to the computation of the number of jets, which affects the repartition of events between the 0-jet, 
VBF, and boosted categories, and to the computation of $\mjj$, which is one of the observables in the VBF category.

The rate uncertainty related to discarding events with a b-tagged jet in the $\Pe\Pgm$ decay channel is up to
5\% for the $\ttbar$ background. The uncertainty in the mistagging rate of gluon and light-flavor jets is negligible.

The \etvecmiss scale uncertainties~\cite{CMS-JME-12-002}, which are computed event-by-event, affect the normalization of 
various processes through the event selection, as well as their distributions through the propagation of these uncertainties 
to the di-$\Pgt$ mass $\mtt$. The \etvecmiss scale uncertainties arising from unclustered energy deposits in the detector 
come from four independent sources related to the tracker, ECAL, HCAL, and forward calorimeters subdetectors. Additionally, 
\etvecmiss scale uncertainties related to the uncertainties in the jet energy scale measurement, which lead to 
uncertainties in the \etvecmiss calculation, are taken into account. The combination of both sources of uncertainties in 
the \etvecmiss scale leads to an uncertainty of about 10\% in the measured signal strength.

\subsection{Background estimation uncertainties}

The $\PZ\to\Pgt\Pgt$ background yield and distribution are corrected based on the agreement between data and the 
background prediction in a control region enriched in the $\PZ\to\Pgm\Pgm$ events, as explained in 
Section~\ref{sec:background_estimation}. The extrapolation uncertainty related to kinematic differences in the selections 
in the signal and control regions ranges between 3 and 10\%, depending
on the category. In addition, shape uncertainties related to the uncertainties in the applied corrections are considered; 
they reach 20\% for some ranges of $\mjj$ in the VBF category. These uncertainties arise from the different level of 
agreement between data and simulation in the $\PZ\to\Pgm\Pgm$ control region obtained when varying the threshold on the muon $\pt$.

The uncertainties in the $\PW+ \text{jets}$ event yield determined from the control regions in the $\Pgm\tauh$ and 
$\Pe\tauh$ channels account for the statistical uncertainty of the observed data, the statistical uncertainty of the 
$\PW+ \text{jets}$ simulated sample, and the systematic uncertainties associated with background processes in these 
control regions. Additionally, an uncertainty in the extrapolation  of the constraints from the high-\MT 
($\MT>80\GeV$) control regions to the low-\MT ($\MT<50\GeV$) signal regions is additionally taken into account. 
The latter ranges from 5 to 10\%, and is obtained by comparing the \MT distributions of simulated and observed 
$\PZ\to\Pgm\Pgm$ events where one of the muons is removed and the \etvecmiss adjusted accordingly, to mimic 
$\PW+\text{jets}$ events. The reconstructed invariant mass of the parent boson in the rest frame is multiplied by the 
ratio of the $\PW$ and $\PZ$ boson masses before removing the muon.
In the $\tauh\tauh$ and $\Pe\Pgm$ channels, where the $\PW+\text{jets}$ background is estimated from simulation, the 
uncertainty in the yield of this small background is equal to 4 and 20\%, respectively. The larger value for the 
$\Pe\Pgm$ channel includes uncertainties in the misidentification rates of jets as electrons and muons, whereas the 
uncertainty in the misidentification rate of jets as $\tauh$ candidates in the $\tauh\tauh$ channel is accounted for 
by the linear uncertainty as a function of the $\tauh$ $\pt$ described earlier.

The uncertainty in the QCD multijet background yield in the $\Pe\Pgm$ decay channel ranges from 10 to 20\%, depending 
on the category. It corresponds to the uncertainty in the extrapolation factor from the same-sign to opposite-sign 
region, measured in events with anti-isolated leptons. In the $\Pgm\tauh$ and $\Pe\tauh$ decay channels, uncertainties from 
the fit of the control regions with leptons passing relaxed isolation conditions are
considered, together with an additional 20\% uncertainty that accounts for the extrapolation from the relaxed-isolation 
control region to the isolated signal region.
In the $\tauh\tauh$ decay channel, the uncertainty in the QCD mutlijet background yield is a combination of the 
uncertainties obtained from fitting the dedicated control regions with $\tauh$ candidates passing relaxed isolation 
criteria, and of extrapolation uncertainties to the signal region ranging from 3 to 15\% and accounting for limited 
disagreement between prediction and data in signal-free regions with various loose isolation criteria.

The yield of events in a $\ttbar$-enriched region is added to the maximum likelihood fit to control the normalization of this process
in the signal region, as explained in Section~\ref{sec:background_estimation}. The uncertainty from the fit in the 
control region is automatically propagated to the signal regions, resulting in an uncertainty of about 5\% on the 
$\ttbar$ cross section. Per-channel uncertainties related to the object reconstruction and identification are 
considered when extrapolating from the $\Pe\Pgm$ final state to the others. The $\ttbar$ simulation is corrected 
for differences in the top quark $\pt$ distributions observed between data and simulation, and an uncertainty 
in the correction is taken into account.

The combined systematic uncertainty in the background yield arising from diboson and single top quark production 
processes is estimated to be 5\%
on the basis of recent CMS measurements~\cite{Khachatryan:2016tgp,Sirunyan:2016cdg}.


\subsection{Signal prediction uncertainties}

The rate and acceptance uncertainties for the signal processes related to the theoretical calculations are 
due to uncertainties in the PDFs, variations of the QCD renormalization and factorization scales,
and uncertainties in the modelling of parton showers.
The magnitude of the rate uncertainty depends on the production process and on the event category.

The inclusive uncertainty related to the PDFs amounts to 3.2, 2.1, 1.9, and 1.6\%, respectively, for 
the $ \cPg\cPg\PH $, VBF, $\PW\PH$, and $\PZ\PH$ production modes~\cite{deFlorian:2016spz}. The
corresponding uncertainty for the variation of the renormalization and factorization scales is 
3.9, 0.4, 0.7, and 3.8\%, respectively~\cite{deFlorian:2016spz}.
The acceptance uncertainties related to the particular selection criteria used in this analysis are less 
than 1\% for the $ \cPg\cPg\PH $ and VBF productions for the PDF
uncertainties. The acceptance uncertainties for the VBF production in the renormalization and factorization 
scale uncertainties are also less than 1\%, while the corresponding uncertainties for the $ \cPg\cPg\PH $ 
process are treated as shape uncertainties as the uncertainty increases
linearly with $\pth$ and $\mjj$.

The \pt distribution of the Higgs boson in the {\POWHEG2.0} simulations is tuned to match more closely
the next-to-NLO (NNLO) plus
next-to-next-to-leading-logarithmic (NNLL) prediction in the
\textsc{HRes2.1} generator~\cite{deFlorian:2012mx,Grazzini:2013mca}.
The acceptance changes with the variation of the parton shower tune in \HERWIG++ 2.6 samples~\cite{Bellm:2013hwb} 
are considered as additional uncertainties, and amount to up to 7\% in the boosted category. The theoretical 
uncertainty in the branching fraction of the Higgs boson to $\Pgt$ leptons is equal to 2.1\%~\cite{deFlorian:2016spz}.

The theoretical uncertainties in the signal production depend on the jet multiplicity; this effect is included 
by following the prescriptions in Ref.~\cite{Stewart:2011cf}. This effect needs to be taken into account because 
the definitions of the three categories used in the analysis are based partially on the number of reconstructed 
jets. Additional uncertainties for boosted Higgs bosons, related to the treatment of the top quark mass in 
the calculations, are considered for signal events with $\pth>150\GeV$.

Theory uncertainties in the signal prediction contribute an uncertainty of 10\% to the measurement of the signal strength.





\subsection{Other uncertainties}

The uncertainty in the integrated luminosity amounts to 2.5\%~\cite{CMS-PAS-LUM-17-001}.

Uncertainties related to the finite number of simulated events, or to the limited number of events in data control 
regions, are taken into account. They are considered for all bins of the distributions used to extract the results 
if the uncertainty is larger than 5\%. They are uncorrelated across different samples, and across bins of a single 
distribution. Taken together, they contribute an uncertainty of about 12\% to the signal strength measurement, 
coming essentially from the VBF category, where the background templates are less
populated than in the other categories.

The systematic uncertainties considered in the analysis are summarized in Table~\ref{tab:uncertainties}.

\begin{table*}[!ht]
\centering
\newcolumntype{x}{D{,}{\text{--}}{2.2}}
\begin{small}
\begin{tabular}{llx}
Source of uncertainty & Prefit & \multicolumn{1}{c}{Postfit (\%) }\\
\hline
 $\tauh$ energy scale                & 1.2\% in energy scale & 0.2,0.3 \\
 $\Pe$ energy scale               & 1--2.5\%  in energy scale & 0.2,0.5\\
 $\Pe$ misidentified as $\tauh$ energy scale & 3\% in energy scale & 0.6,0.8 \\
 $\Pgm$ misidentified as $\tauh$ energy scale & 1.5\% in energy scale &  0.3,1.0\\
 Jet energy scale               & Dependent upon $\pt$ and $\eta$ & \multicolumn{1}{c}{\NA} \\
 \etvecmiss energy scale              & Dependent upon $\pt$ and $\eta$ & \multicolumn{1}{c}{\NA}  \\[\cmsTabSkip]
 $\tauh$ ID \& isolation & 5\% per $\tauh$ & \multicolumn{1}{c}{3.5} \\
 $\tauh$ trigger & 5\% per $\tauh$ & \multicolumn{1}{c}{3} \\
 $\tauh$ reconstruction per decay mode & 3\% migration between decay modes & \multicolumn{1}{c}{2} \\
 $\Pe$ ID \& isolation \& trigger  &   2\% & \multicolumn{1}{c}{\NA} \\
 $\Pgm$ ID \& isolation \& trigger & 2\% & \multicolumn{1}{c}{\NA} \\
 $\Pe$ misidentified as $\tauh$ rate   & 12\%  & \multicolumn{1}{c}{5} \\
 $\Pgm$ misidentified as $\tauh$ rate  & 25\%  & 3,8 \\
 Jet misidentified as $\tauh$ rate     & 20\% per 100\GeV $\tauh$ $\pt$ & \multicolumn{1}{c}{15}  \\[\cmsTabSkip]
 $\PZ\to\Pgt\Pgt/\ell\ell$ estimation & Normalization: 7--15\% & 3,15 \\
                             & Uncertainty in $m_{\ell\ell/\Pgt\Pgt}$, $\pt(\ell\ell/\Pgt\Pgt)$,  & \multicolumn{1}{c}{\NA} \\
                             & and $\mjj$ corrections & \\[\cmsTabSkip]
 $\PW+ \text{jets}$ estimation & Normalization ($\Pe\Pgm$, $\tauh\tauh$): 4--20\% &  \multicolumn{1}{c}{\NA} \\
                               & Unc. from CR ($\Pe\tauh$, $\Pgm\tauh$): $\simeq$5--15& \multicolumn{1}{c}{\NA} \\
                               & Extrap. from high-$m_T$ CR ($\Pe\tauh$, $\Pgm\tauh$): 5--10\% & \multicolumn{1}{c}{\NA}  \\[\cmsTabSkip]
QCD multijet estimation        & Normalization ($\Pe\Pgm$): 10--20\% & 5,20\% \\
                               & Unc. from CR ($\Pe\tauh$, $\tauh\tauh$, $\Pgm\tauh$): $\simeq$5--15\% & \multicolumn{1}{c}{\NA} \\
                               & Extrap. from anti-iso. CR ($\Pe\tauh$, $\Pgm\tauh$): 20\% & 7,10 \\
                               & Extrap. from anti-iso. CR ($\tauh\tauh$): 3--15\% & 3,10 \\[\cmsTabSkip]
 Diboson normalization & 5\% & \multicolumn{1}{c}{\NA}  \\[\cmsTabSkip]
 Single top quark normalization  & 5\% & \multicolumn{1}{c}{\NA} \\[\cmsTabSkip]
 $\ttbar$ estimation & Normalization from CR: $\simeq$5\% & \multicolumn{1}{c}{\NA} \\
                     & Uncertainty on top quark $\pt$ reweighting & \multicolumn{1}{c}{\NA}  \\[\cmsTabSkip]
 Integrated luminosity     & 2.5\% & \multicolumn{1}{c}{\NA} \\
 b-tagged jet rejection ($\Pe\Pgm$) & 3.5--5.0\% & \multicolumn{1}{c}{\NA} \\
 Limited number of events                & Statistical uncertainty in individual bins & \multicolumn{1}{c}{\NA}  \\[\cmsTabSkip]
 Signal theoretical uncertainty  & Up to 20\% & \multicolumn{1}{c}{\NA} \\
\hline
\end{tabular}
\end{small}
\label{tab:uncertainties}
\caption{Sources of systematic uncertainty. If the global fit to the signal and control regions, described 
in the next section, significantly constrains these uncertainties, the values of the uncertainties after 
the global fit are indicated in the third column. The acronyms CR and ID stand for control region and 
identification, respectively.}
\end{table*}
\subsection{Luminsity}
\subsection{Lepton ID and Isolation}
\subsection{QCD Estimation}
\subsection{ttbar Estimation}
\subsection{Tau Take Rate}
\subsection{Energy Scales}
\subsubsection{Tau Energy Scale}
\subsubsection{Jet Energy Scale}
\subsubsection{MET Energy Scale}
\subsection{Drell--Yan Reweighting}
\subsection{Theoretical Uncertainties for Higgs Boson}




%
\chapter{Higgs $\to \tau\tau$: Results}


\begin{figure*}[htbp]
\centering
     \includegraphics[width=0.45\textwidth]{higgs_to_taus/plots/cms_output_freeze_All_Theory_bbb}\\
     \caption{
     }
     \label{fig:htt_systematic_parabola}
\end{figure*}






%\chapter{Higgs $\to \Pgt\Pgt$: $\PW\PH$ and $\PZ\PH$ Associated Production}
\label{sec:vh_analysis}

This chapter describes a study of Higgs boson produced via the $\PW\PH$ and
$\PZ\PH$ associated production processes. The Higgs boson subsequently
decays to a pair of $\tau$ leptons.
This study is complementary to the previously described $ggH$ and VBF targeted
$\htt$ analysis in Chapter~\ref{sec:htt_analysis} and is performed 
using data corresponding to 35.9\fbinv of integrated luminosity 
collected at center-of-mass energy 13\TeV. Combining
the results from this associated production targeted analysis with the results 
of the $ggH$ and VBF targeted analysis~\ref{HIG-16-043}
we produce the first single experiment observation of the $\htt$ process at 13\TeV, 
measured at the 5.5 $\sigma$ confidence level. 

For the $\PZ\PH$ targeted final states, $\PZ\rightarrow \Pe\Pe$
and $\PZ \to \Pgm\Pgm$ decays are considered combined with four possible $\htt$ 
final states: $\Pe\tauh$, $\Pgm\tauh$,
$\Pe\Pgm$ and $\tauh\tauh$. For the $\PW\PH$ targeted final states, four final states are considered with
the $\PW$ boson decaying leptonically to an electron or muon plus a neutrino, 
and at least one $\tauh$ from the Higgs boson decay:
$\Pgm\Pgm\tauh$, $\Pe\Pgm\tauh$, $\Pe\tauh\tauh$ and $\Pgm\tauh\tauh$. 

There are many similarities in the treatment of simulated samples, Monte Carlo
corrections, and uncertainties between the $ggH$ and VBF targeted $\htt$ analysis
and the associated production targeted $\htt$ analysis. When appropriate, instead of repeating
what has already been documented in other chapters, I refer back to the previous sections for
full details.



\section{Overview}
This chapter specifically focuses on studying the Higgs boson produced via the associated
production mechanisms, $\PW\PH$/$\PZ\PH$.
In the following pages the symbol $\ell$ refers to light leptons (electrons and muons) and $\tauh$ refers to hadronically
decaying $\tau$ leptons. Leptons refers inclusively to electrons, muons, and $\Pgt$ including their decay products:
$\tau_{e}$, $\tau_{\mu}$, $\tauh$.
We study all possible $\tau\tau$ final state combinations with the
exception of two electron and two muon final states because of the low 
$\tau\tau \to \tau_{e}\tau_{e}/\tau_{\mu}\tau_{\mu}$
branching fractions. The $\htt$ final states which are
studied are the same as those used in the $ggH$ and VBF analysis and are: 
$\tau_{e}\tauh$ ($\Pe\tauh$), $\tau_{\mu}\tauh$ ($\Pgm\tauh$),
$\tau_{e}\tau_{\mu}$ ($\Pe\Pgm$), and lastly, $\tauh\tauh$ ($\tauh\tauh$).
This combination of $\Pgt\Pgt$ final states covers about 94\% of all possibilities.
We ensure uniqueness between the studied final states be applying veto criteria to events based
on the number of reconstructed loosely identified electrons and muons.



\subsection{Triggers}
The $\PW$ and $\PZ$ bosons in the $\PW\PH$ and $\PZ\PH$ final states ensure the presence of one 
or two well-isolated leptons with sufficiently high $\pt$. Because of this,
we use single or double lepton triggers to select events.
The $\PW\PH$ targeted final states rely on a set of single lepton triggers
which must be fired by the $\PW$ boson lepton. These single
lepton triggers are the same single electron and single muon triggers which are 
used in the $ggH$ and VBF targeted analysis described in 
Section~\ref{sec:htt_triggers}. In general, the $\PW$ boson lepton
will have a higher $\pt$ than the Higgs boson leptons, providing
higher efficiency for trigger selection. Requiring that the $\PW$
boson lepton fires the trigger ensures that there is negligible
isolation or identification
bias in the selection of the Higgs boson leptons from the trigger
requirements. The lack of bias in the Higgs boson lepton selection 
allows us to use specific background estimation techniques which are described
in the background estimation Section~\ref{sec:vh_background_estimation}.

Double electron and double muon triggers are used in 
the $\PZ\PH$ targeted final states to trigger on the $\PZ$ decay products.
The presence of two leptons in the double lepton
triggers allows for lower $\pt$ thresholds online, which increases
the acceptance of $\PZ\PH$ events after offline selections are applied. 
Similar to the $\PW\PH$ final states, triggering on the vector boson associated
leptons removes selection bias from the
Higgs boson associate leptons and enables the use of specific background
estimation methods (Section~\ref{sec:vh_background_estimation}). Additionally, $\PZ\PH$ events can be selected using
single lepton triggers applied to either of the $\PZ$ leptons. 
This helps increase the overall trigger selection efficiency.

The trigger selection criteria for the $\PW\PH$ and $\PZ\PH$ targeted final 
states is detailed in Table~\ref{tab:vh_triggers}.


\begin{table*}[htbp]
\centering
\begin{small}
\begin{tabular}{lll}
     \multicolumn{3}{c}{$\PW\PH$ trigger selection requirements}                 \\ 
\hline
  Final State           &         Trigger ($\pt/\eta$)         & Lepton Selection: $\pt$     \\
\hline
 $\Pe\Pgm\tauh$      &  $\Pgm (22/2.1)$ or $\Pe (25/2.1)$  &     $\pt^\Pe>15, \pt^\Pgm>23$ or $\pt^\Pe>26, \pt^\Pgm>15$  \\ 
 $\Pgm\Pgm\tauh$     &  $\Pgm (22/2.1)$                    &     $\pt^\Pgm>23,\pt^\Pgm>15$                               \\ 
 $\Pe\tauh\tauh$     &  $\Pe (25/2.1)$                     &     $\pt^\Pe>26$                                            \\ 
 $\Pgm\tauh\tauh$    &  $\Pgm (22/2.1)$                    &     $\pt^\Pgm>23$                                           \\ 
\hline \\

\\
     \multicolumn{3}{c}{$\PZ\PH$ trigger selection requirements}                 \\ 
\hline
  Final State           &         Trigger ($\pt/\eta$)         & Lepton Selection: $\pt$             \\
\hline
  $\Pe\Pe\Pgm\tauh$     &                                    &                                   \\ 
  $\Pe\Pe\Pe\tauh$      & $\Pe(23/2.5)\,\&\,\Pe(12/2.5)$,    &  $\pt^\Pe>24\,\&\,\pt^\Pe>13$,    \\ 
  $\Pe\Pe\tauh\tauh$    & or $\Pe(27/2.5)$                   &  or $\pt^\Pe>28$                  \\ 
  $\Pe\Pe\Pe\Pgm$       &                                    &                                   \\ 
\hline
  $\Pgm\Pgm\Pgm\tauh$   &                                    &                                   \\ 
  $\Pgm\Pgm\Pe\tauh$    &  $\Pgm(17/2.4)\,\&\,\Pgm(8/2.4)$,  &  $\pt^\Pgm>18\,\&\,\pt^\Pgm>10$,  \\ 
  $\Pgm\Pgm\tauh\tauh$  &   or $\Pgm(24/2.4)$                &  or $\pt^\Pgm>25$                 \\ 
  $\Pgm\Pgm\Pe\Pgm$     &                                    &                                   \\ 
\hline
\end{tabular}
\end{small}
\caption{Kinematic selection requirements for $\PW\PH$ and $\PZ\PH$ events.
The trigger requirement is defined by a combination of trigger candidates with 
$\pt$ over a given threshold (in \GeV), indicated inside parentheses. The 
pseudorapidity thresholds come from trigger and object reconstruction constraints.
The trigger requirements for the $\PZ\PH$ events are defined by the $\PZ$ boson
decay products, either $\PZ\to\Pe\Pe$ or $\PZ\to\Pgm\Pgm$.
\label{tab:vh_triggers}
}
\end{table*}



\subsection{Event Selection}
\label{sec:vh_evt_selection}
In the semileptonic $\PW\PH$ associated production final states, $\Pe\Pgm\tauh$ and 
$\Pgm\Pgm\tauh$,
the two light leptons are required to have the same charge to reduce the $\ttbar$ 
and $\PZ+\textrm{jets}$ backgrounds where one or more jets is misidentified as a $\tauh$ 
candidate. The $\tauh$ candidate has opposite charge to the light leptons. The leading (highest $\pt$)
light lepton is considered as coming from the $\PW$ boson, while the Higgs boson 
candidate is formed from the $\tauh$ and the subleading (lowest $\pt$) light lepton. The 
correct pairing is achieved in about 75\% of events. The leading light lepton is required 
to fire the single lepton triggers and to have a $\pt$ that is 1\GeV above the online 
thresholds, whereas the subleading light lepton has $\pt>15\GeV$; this $\pt$ threshold
is the result of optimizing the selection for maximum signal significance. 
Based on optimizing for signal sensitivity, selection criteria based 
on three variables improve the sensitivity of the results in both final states:
\begin{itemize}
\item $L_{\text{T}}>100\GeV$, where $L_{\text{T}}$ is the scalar $\pt$ sum of the three leptons in the final state
\item $\abs{\Delta\phi(\ell_1,\PH)}>2.0$, where $\ell_1$ is the leading light lepton, and 
$\PH$ is the system formed by the subleading light lepton and the $\tauh$ candidate
\item $\abs{\Delta\eta(\ell_1,\PH)}<2.0$.
\end{itemize}


In the hadronic $\PW\PH$ associated production final states, $\Pe\tauh\tauh$ and 
$\Pgm\tauh\tauh$,
the $\tauh$ candidates are both assumed to be from the Higgs boson decay, and thus are 
required to have opposite charge. Based on optimizing for signal significance,
the $\tauh$ that has the same charge as the light lepton must 
have $\pt > 35\GeV$. This is driven 
by the fact that the $\tauh$ that has the same charge as the light lepton is almost 
always a jet misidentified as a $\tauh$ candidate, and the jet misidentification 
rate strongly decreases with $\pt$. 
The subleading $\tauh$ must have $\pt > 20\GeV$. 
Selection criteria based on three variables 
have been found to increase the sensitivity of the results in both final states:
\begin{itemize}
\item $L_{\text{T}}>130\GeV$, where $L_{\text{T}}$ is the scalar $\pt$ sum of the three leptons in the final state
\item $\abs{\vec{S_{\text{T}}}}<70\GeV$, where $\vec{S_{\text{T}}}$ is the vectorial $\pt$ sum of the three leptons in the final state and of $\etvecmiss$
\item $\abs{\Delta\eta(\tauh,\tauh)}<2.0$.
\end{itemize}



In the $\PZ\PH$ final states, the $\PZ$ boson is reconstructed from the opposite charge, same-flavor
light lepton combination which has a mass closest to the $\PZ$ boson mass. Different 
electron and muon identification and isolation criteria are used for the leptons 
assigned to the $\PZ$ boson than those assigned to the Higgs boson. A looser
selection is applied for the $\PZ$ boson leptons to increase signal acceptance. The
rate of misidentification for these leptons is relatively low because of the required $\PZ$ mass
window cut, $60\GeV < m_{\ell\ell} < 120\GeV$, and the opposite charge criteria.
In comparison, a tighter selection is applied to the leptons assigned to the Higgs boson to
decrease the background contributions from $\PZ+\textrm{jets}$ and other reducible
backgrounds. The specific selections are detailed in Table~\ref{tab:vh_inclusive_selection},
including those for the $\tauh$ candidates. All identification and isolation criteria
where chosen based on optimizing for the best signal sensitivity.

The signal region is split into a High-$L_{\text{T}}^{\textrm{Higgs}}$ and Low-$L_{\text{T}}^{\textrm{Higgs}}$
region; $L_{\text{T}}^{\textrm{Higgs}}$ is defined as the scalar $\pt$ sum of the decay 
products of the Higgs boson. This splitting helps separate the $\PZ+\text{jets}$
background from the $\PZ\PH$ signal which is concentrated in the
High-$L_{\text{T}}^{\textrm{Higgs}}$ region.
The different $\Pgt$ decay processes define the kinematics of the Higgs boson and
the $L_{\text{T}}^{\textrm{Higgs}}$. Therefore, the $L_{\text{T}}^{\textrm{Higgs}}$
regions are defined based on the specific $\htt$ final states of an event. 
The split between the High- and Low-$L_{\text{T}}^{\textrm{Higgs}}$
regions are:
\begin{itemize}
\item $\ell\ell\Pe\tauh$: $L_{\text{T}}^{\textrm{Higgs}} = 60\GeV$
\item $\ell\ell\Pgm\tauh$: $L_{\text{T}}^{\textrm{Higgs}} = 60\GeV$
\item $\ell\ell\tauh\tauh$: $L_{\text{T}}^{\textrm{Higgs}} = 75\GeV$
\item $\ell\ell\Pe\Pgm$: $L_{\text{T}}^{\textrm{Higgs}} = 50\GeV$
\end{itemize}
For convenience, the High- and Low-$L_{\text{T}}^{\textrm{Higgs}}$ regions are plotted
side-by-side, see Figures~\ref{fig:zh_all_eight1}, ~\ref{fig:zh_all_eight2}, 
and ~\ref{fig:zh_results_svFitAll}.



\subsection{Baseline Object Selection}
\label{sec:vh_obj_selection}

The expression $\pt^{\ell}$ stands for the $\pt$ of the lepton. The electron and muon isolation 
requirements used in this analysis, based on $I^{\ell}$ (Section~\ref{eqn:rel_isolation}), 
are listed in Table~\ref{tab:vh_inclusive_selection}. As stated above, all 
identification and isolation criteria where chosen based on optimizing for the best signal sensitivity.

The $\tauh$ candidates are identified using the MVA discriminant discussed in
Section~\ref{sec:obj_reco_tau}.
Three $\tauh$ MVA working points are used in this analysis,
\texttt{Very Tight Tau MVA}, \texttt{Tight Tau MVA}, and \texttt{Medium Tau MVA} ID. Each working point was selected based on
optimizing the analysis for highest sensitivity to the associated production
$\htt$ processes. In the lower statistics $\PZ\PH$ final states the higher
efficiency \texttt{Medium Tau MVA} is used. In the hadronic $\PW\PH$ final states
a combination of two working points is used. The 
$\tauh$ that has same charge as the light lepton, which 
is likely to be a fake, has to pass the \texttt{Very Tight Tau MVA} working 
point. The other $\tauh$ that has opposite charge to the light lepton
is less likely to be a fake and must only pass \texttt{Medium Tau MVA}.
In the semileptonic final states the \texttt{Tight Tau MVA} working points
is used.

For each final state there are additional $\tauh$ requirements
which help suppress electron to $\tauh$ and muon to $\tauh$ misidentification.
The exact working points used are tuned for each final state to suppress dominant
misidentified backgrounds. These discriminants are denoted in this analysis as
anti-e and anti-$\Pgm$ criteria and are discussed in Section~\ref{sec:obj_reco_tau}.
The discriminants have a range of thresholds with working points being
referred to as ``Very Loose'' (VL), ``Loose'' (L), ``Medium'' (M), and ``Tight'' (T).

A summary of the lepton selection details for each final state in the associated
production analysis is in Table~\ref{tab:vh_inclusive_selection}.
All reconstructed leptons in the events are required to be separated from each 
other by $\Delta R > 0.3$. In the case of $\tauh$, they are required to be 
separated from all other leptons by $\Delta R > 0.5$. The resulting event samples are made mutually 
exclusive by discarding events that have additional loosely identified 
and isolated muons or electrons.

We reject events which have been tagged as likely including heavy flavor jet decays from b-quarks.
The working point chosen gives an efficiency for identifying real b jets of about 70\% for 
about 1\% of light flavor or quark jets being misidentified, Section~\ref{sec:reco_b_jet}.
In the $\PZ\PH$ final states, this selection removes roughly 13\% of the background events
at a cost of only 2\% of the signal events increasing the purity of the signal region.
%For negligible signal loss, this requirement significantly decreases the $\ttbar$,
%$\ttbar\PW$, and $\ttbar\PZ$ background contributions.

\begin{table*}[htbp]
\centering
\begin{small}
\begin{tabular}{ll}
     \multicolumn{2}{c}{ \textbf{$\PW\PH$ selection requirements} }                \\ 
     \multicolumn{2}{c}{$\tauh$ baseline req: $\pt^{\tauh}>20$, $|\eta|<2.3$, anti-e VL, anti-$\mu$ L}    \\
     \multicolumn{2}{c}{$\Pe$ baseline req: $\pt^\Pe>10$, $|\eta|<2.5$, $I^\Pe<0.10$, $\Pe$ MVA Tight ID}                 \\ 
     \multicolumn{2}{c}{$\Pgm$ baseline req: $\pt^\Pgm>10$, $|\eta|<2.4$, $I^\Pgm<0.15$, Medium ID}          \\
\hline
  Final State            &      Additional $\tauh$ Criteria  \\
\hline
 $\Pe\Pgm\tauh$      &   \texttt{Tight Tau MVA}, anti-e VL/T, anti-$\mu$ T/L             \\
 $\Pgm\Pgm\tauh$     &   \texttt{Tight Tau MVA}, anti-$\mu$ T             \\
 $\Pe\tauh\tauh$     &   \texttt{Medium/Very Tight Tau MVA}, anti-e T \\
 $\Pgm\tauh\tauh$    &   \texttt{Medium/Very Tight Tau MVA}, anti-$\mu$ T \\
\hline \\

\\
     \multicolumn{2}{c}{ \textbf{$\PZ\PH$ selection requirements} }                \\ 
     \multicolumn{2}{c}{$\PZ$ boson: opposite charge, same-flavor light leptons, $60\GeV < m_{\ell\ell} < 120\GeV$}  \\ 
     \multicolumn{2}{c}{$\tauh$ baseline req: $\pt^{\tauh}>20$, $|\eta|<2.3$, \texttt{Medium Tau MVA}}   \\ 
     \multicolumn{2}{c}{$\Pe$ baseline req: $\pt^\Pe>10$, $|\eta|<2.5$, MVA Loose ID }   \\ 
     \multicolumn{2}{c}{$\Pgm$ baseline req: $\pt^\Pgm>10$, $|\eta|<2.4$, Loose ID, $I^\Pgm<0.25$ }   \\ 
\hline
  Final State           &        Additional Higgs Boson Lepton Criteria  \\
\hline
  $\Pe\Pe\Pgm\tauh$     &   $I^\Pgm<0.15$       \\
  $\Pe\Pe\Pe\tauh$      &   $\Pe$ Tight MVA ID, $I^\Pe<0.15$ \\
  $\Pe\Pe\tauh\tauh$    &   baseline selection       \\
  $\Pe\Pe\Pe\Pgm$       &   $\Pe$ Tight MVA ID, $I^\Pe<0.15$, $I^\Pgm<0.15$ \\
\hline
  $\Pgm\Pgm\Pgm\tauh$   &   $I^\Pgm<0.15$       \\
  $\Pgm\Pgm\Pe\tauh$    &   $\Pe$ Tight MVA ID, $I^\Pe<0.15$ \\
  $\Pgm\Pgm\tauh\tauh$  &   baseline selection       \\
  $\Pgm\Pgm\Pe\Pgm$     &   $\Pe$ Tight MVA ID, $I^\Pe<0.15$, $I^\Pgm<0.15$ \\
\hline
\end{tabular}
\end{small}
\caption{
Electron, muon, and $\tauh$ selection criteria for each final state in the
associated production $\htt$ analysis. In the $\Pe\Pgm\tauh$ final state there
are two different working points listed for electron and muon rejection.
Anti-e VL, anti-$\mu$ T applies for events where the electron and $\tauh$
are same charge and anti-e T, anti-$\mu$ L applies for events where the muon
and $\tauh$ are same charge.
}
\label{tab:vh_inclusive_selection}
\end{table*}



\section{Data Set}
\label{sec:vh_dataset}
The associated production targeted $\htt$ study utilizes the same dataset as the
$ggH$ and VBF targeted study, the full 2016 $\pp$ 
dataset collected by CMS corresponding to 35.9$\fbinv$ 
of integrated luminosity. The data were gathered at center-of-mass energy 13\TeV.
For further details see Section~\ref{sec:htt_dataset}.



\section{Monte Carlo Samples}
\label{sec:vh_mc_samples}
Signal and background processes are modeled with samples of simulated events.
For details on the production of simulated events, see Section~\ref{sec:simulation}.
%Because the analysis focuses on
%measuring the $\PW\PH$ and $\PZ\PH$ processes, the $\ttbar\PH$ process is 
%included as background.
For each simulated event, a number of additional pileup interactions is simulated and added. 
The number of pileup interactions added is based on best efforts to match the simulated
events to the pileup in data which is estimated from the measured instantaneous
luminosity for each bunch crossing. The average number of additional pileup interactions in
the 2016 CMS data is approximately 27 interactions per bunch crossing.



\section{Mass Reconstruction}
The visible mass of the $\Pgt\Pgt$ system, $\mvis$, can be used to separate
the $\htt$ signal events
from the large contribution of irreducible $\PZ \to \Pgt \Pgt$ events.
However, the neutrinos from the $\Pgt$ lepton decays carry a large fraction of
the $\Pgt$ lepton energy and reduce the discriminating power of this variable.
The \textsc{svfit} algorithm used in the $ggH$ and VBF targeted analysis,
discussed in Section~\ref{sec:svfit}, is used in the $\PZ\PH$ final states.
It combines the \etvecmiss 
with the four-vectors of both $\Pgt$ candidates to calculate a more accurate 
estimate of the mass of the parent boson and is denoted as $\mtt$. The
$\mtt$ variable is used in the $\PZ\PH$ final state where the majority of 
\etvecmiss is attributable to the $\Pgt$ decays. The $\mvis$ is used in the 
$\PW\PH$ final state because the \textsc{svfit} algorithm is not designed to account for the 
additional \etvecmiss associated with the neutrino from the $\PW$ boson
decay~\cite{Bianchini:2014vza}. 



\section{Background Estimation}
\label{sec:vh_background_estimation}
The simulated background processes are all scaled to their NLO cross section in Table~\ref{tab:vh_bkg_xsec}.

\begin{table}[htpb]
\begin{center}
\begin{footnotesize}
\begin{tabular}{lc}
Background Process& Cross section (pb) \\
\hline
Z+jets Inclusive Jet Production & 5747 \\
$t\bar{t}$ & 831.8\\
EWK $WZ \rightarrow 3\ell\nu$& 4.708\\
$ZZ \rightarrow 4\ell$& 1.212\\
$gg \rightarrow ZZ \rightarrow 2\ell2\ell$ & 0.005423 \\
$gg \rightarrow ZZ \rightarrow 4\ell$ & 0.002703 \\
$\ttbar\PZ + \text{jets}$ & 0.2529 \\
$WWW$   & 0.2086  \\
$WWZ$   & 0.1651  \\
$WZZ$   & 0.05565  \\
$ZZZ$   & 0.01398  \\
$\PZ\PH$, $\PH \rightarrow \PW\PW$  & 0.02005  \\
$\PW^{-}\PH$, $\PH \rightarrow \PW\PW$  & 0.01209  \\
$\PW^{+}\PH$, $\PH \rightarrow \PW\PW$  & 0.01905  \\
$gg \rightarrow \PH \rightarrow WW \rightarrow 2\ell 2\nu$& 1.001\\
VBF $H \rightarrow WW \rightarrow 2\ell 2\nu$ & 0.08900\\
$gg \rightarrow \PH \rightarrow \PZ\PZ$  & 0.0121  \\
$\ttbar\PH \rightarrow \text{non-}\bbbar$  & 0.215  \\
\hline
\end{tabular} 
\end{footnotesize}
\end{center}
\caption{
    NLO cross sections for considered backgrounds. In this table, $\ell$ represents all three generations of charged leptons, $\Pe$,$\Pgm$,$\Pgt$. In some cases the production mechanism is listed: quarks ($qq$) versus gluons ($gg$).
}
\label{tab:vh_bkg_xsec}
\end{table}


The irreducible backgrounds for the associated production analysis can
be split into those with four lepton final states for $\PZ\PH$ and those
with three lepton final states for $\PW\PH$. When a lepton escapes the fiducial
volume of the detector, or is otherwise poorly reconstructed, the four
lepton irreducible backgrounds can populate the $\PW\PH$ three
lepton final states.
Backgrounds typically composing
the irreducible background for the $\PZ\PH$ final states are: $\PZ\PZ$, 
$\ttbar\PZ$, $\PW\PW\PZ$, $\PW\PZ\PZ$, and $\PZ\PZ\PZ$. The dominant
irreducible backgrounds for the $\PW\PH$ final states are:
$\PW\PZ$ and $\ttbar\PW$. The irreducible backgrounds are 
estimated from simulation and scaled to their theoretical cross section. Higgs 
boson decays to pairs of $\PW$ or $\PZ$ bosons 
are also estimated from simulation and considered as background processes. 
Additionally, the $\ttbar\PH$ production process with all Higgs boson decay
paths is estimated from simulation and considered a background processes.

The reducible backgrounds, which have at least one jet misidentified as an electron, 
muon, or $\tauh$ lepton, are estimated from data. 
Data events meeting specific requirements detailed below are reweighted 
as a function of a misidentification rate to estimate the 
contribution of these processes in the signal region. 

In the $\PW\PH$ final states, the misidentification rate of jets as electrons, muons, 
or $\tauh$ candidates is measured in $\PZ+\textrm{jets}$ events. The $\PZ$ boson is reconstructed 
in its dielectron decay mode for measuring the jet to muon misidentification
rate, and is reconstructed in its dimuon decay mode for measuring the jet to electron
or $\tauh$ misidentification rate.
The rates are measured in bins of the lepton $\pt$, and are 
split between reconstructed decay mode for the $\tauh$ candidates. 

In the semileptonic $\Pe\Pgm\tauh$ and $\Pgm\Pgm\tauh$ final states, 
events are selected for reweighting if they pass the full signal region 
selection except that the subleading light lepton or the $\tauh$ do not 
pass the isolation or identification conditions.
To remove the overlap between this method and the simulated samples, events in simulation that have a jet that is 
misidentified as the $\tauh$ or as the subleading lepton, are discarded. Simulated 
events that have a jet misidentified as the leading $\PW$ boson lepton, but two real leptons 
for the Higgs boson leptons, are estimated from simulation as their 
contribution is not taken into account with the misidentification rate method 
described above. These events mostly arise from $\ttbar$ and $\PZ+\textrm{jets}$ processes, 
and account for a small fraction of the total expected background in the signal region. 
In the hadronic $\Pe\tauh\tauh$ and $\Pgm\tauh\tauh$ final states, 
the method is essentially the same, except that the lepton most susceptible to being misidentified,
thus having the misidentification estimation method applied to it, 
is the $\tauh$ candidate that has the same charge as the light lepton.  

In the $\PZ\PH$ final states, a similar misidentification rate is used
to estimate the contribution of jets misidentified as electrons, muons, or $\tauh$
in signal region events. The misidentification rate is measured in four lepton final states which
is dominated by $\PZ+\textrm{jets}$ events with a small contribution from 
$\ttbar$ events. Identical to the $\PW\PH$ final states, the rates are measured 
in bins of the lepton $\pt$, and are split between reconstructed decay modes for 
the $\tauh$ candidates.

In these four lepton final states, the leptons most susceptible 
to being misidentified jets are the Higgs boson leptons.
In the $\PZ\PH$ final states, data events which pass the full
signal region selection, except either or both of the Higgs boson leptons 
fail identification or isolation criteria, are weighted by the misidentification
rate for the failing lepton.
To avoid double counting events with misidentified leptons, 
events with both Higgs boson leptons failing have their weight
subtracted from the events which only have a single lepton failing. 
To remove the overlap between this method and the simulated samples, events in simulation that have a jet that is 
misidentified as an electron, muon, or $\tauh$ are discarded.
This misidentification rate method is used to estimate the yield of the reducible
backgrounds.

The shape of the reducible background contribution is taken from
data in a signal-free region with same charge Higgs boson leptons. The
contribution of irreducible backgrounds in the same charge region used to 
derive the shape template is less than 1\%. This means that the data derived template is
a very pure reducible background selection.
A high statistics, relaxed identification and isolation selection is used to reduce
statistical uncertainties for the shape template. Kolmogorov-Smirnov (KS) tests have been performed to validate
shape compatibility between this relaxed same charge selection and the signal region
reducible background distributions. The KS tests indicate there is likely no
shape bias.

The events estimated from these misidentification methods are labeled
as ``jet fakes'' in the following distributions.



\section{Monte Carlo Corrections}
\label{sec:vh_mc_corrections}
Corrections are applied to the simulated Monte Carlo samples to help correct for measured differences
between observed data and expectations based on simulation. Many of these corrections are designed
to correct differences in reconstruction and identification efficiencies for leptons between data
and simulation. These corrections are derived in
fully orthogonal regions from the associated production analysis signal regions.
The Monte Carlo corrections applied in the associated production analysis are identical to those
applied in the $ggH$ and VBF targeted analysis when applicable, see 
Section~\ref{sec:mc_corrections} for details.



\section{Systematic Uncertainties}
\label{sec:vh_systematics}
The systematic uncertainty model used for the associated production analysis share many
similarities with the uncertainty model used for the $ggH$ and VBF targeted analysis
considering the many shared object definitions, simulated backgrounds, and simulated $\htt$
signal samples.
Specific nuisances with identical treatment are:
\begin{itemize}
\item Uncertainty of the $\tauh$ identification efficiency for genuine $\tauh$
\item Uncertainty on the visible energy scale of genuine $\tauh$ leptons
\item Uncertainties in the muon and electron identification, isolation, and trigger efficiencies
\item Uncertainty related to discarding events with a b-tagged jet
\item \etvecmiss scale uncertainties; these uncertainties are skipped for the
$\PW\PH$ semileptonic final states where \etvecmiss is not used
\item Uncertainty on the finite number of simulated events
\item Uncertainty in the integrated luminosity
\end{itemize}
The full details of how these identical nuisances are treated can be found in
Section~\ref{sec:htt_systematics}.


\subsection{Simulated Background Estimation Uncertainties}
Uncertainties from the renormalization and the factorization scales, and from the 
choice of the PDF set (Section~\ref{sec:sim_pdf}), are taken into account for the $\PZ\PZ$ and $\PW\PZ$ 
background processes. The uncertainty from the renormalization and factorization 
scales is determined by varying these scales between 0.5 and 2 times their nominal 
value and computing the change in acceptance. This leads to yield uncertainties 
of $^{+3.2\%}_{-4.2\%}$ for the $\Pq\Pq\rightarrow \PZ\PZ$ background, and $\pm 3.2\%$ 
for the $\PW\PZ$ process. The uncertainty from the PDF set is determined following 
the PDF4LHC recommendations~\cite{Butterworth:2015oua}, and leads to yield uncertainties of $^{+3.1\%}_{-4.2\%}$ for 
the $\Pq\Pq\rightarrow \PZ\PZ$ background, and $\pm 4.5\%$ for the $\PW\PZ$ process. 
In addition, a 10\% uncertainty in the k-factor used for the $\Pg\Pg \rightarrow 
\PZ\PZ$ prediction is considered. The uncertainty in the cross section of the 
rare $\ttbar\PW$ and $\ttbar\PZ$ processes amounts to 25\%.

\subsection{Reducible Background Estimation Uncertainties}
The reducible backgrounds are estimated using the measured rates for jets to be 
misidentified as electrons, muons, or $\tauh$ discussed in 
Section~\ref{sec:vh_background_estimation}. The misidentification rates are 
measured in different bins of lepton $\pt$, and are further split between 
reconstructed decay modes for the $\tauh$. In the $\PW\PH$ final states where
the shape of the reducible background is estimated using the misidentification rate
method, the statistical uncertainty in every 
bin is considered as an independent uncertainty, which is propagated to the mass 
distributions and to the yields of the reducible background estimate. Rate
uncertainties applied in the $\PZ\PH$ final states cover the possible fluctuation
in rate from these types of uncertainties. Additionally, the shape is taken from the same charge
region and has previously been validated as compatible (Section~\ref{sec:vh_background_estimation}) 
thus no specific uncertainty is applied to the shape selection.

In both the $\PW\PH$ and $\PZ\PH$ final states, an additional
uncertainty on the misidentification rates due to mismodeling of simulated
samples is incorporated. The yields of the simulated prompt contributions, which are subtracted from
data, are adjusted within their uncertainty and the effect is propagated forward. This
creates a set of misidentification rates corresponding to an upwards shift in the
normalization of the prompt simulated events as well as a downwards shifted set. These
shifted misidentification rates are then used to estimate the reducible background yield and 
mass distributions corresponding to this 1$\sigma$ shift in the prompt simulated 
background normalization.

In the $\PW\PH$ final states, an 
additional uncertainty comes from potentially different misidentification rates 
in $\PZ+\textrm{jets}$ events, where the rates are measured, and in $\PW+\textrm{jets}$ or 
$\ttbar$ events, which constitute a large fraction of the reducible background 
in the signal region. To cover this, a 20\% yield uncertainty for the reducible 
background is applied in each $\PW\PH$ final state. In the $\PZ\PH$ final
states a similar uncertainty is applied based on potential differences in the
measurement region versus the application region. These uncertainties
range from 26\% in the $\ell\ell\Pgm\tauh$ final states to 100\% in the
$\ell\ell\Pe\Pgm$ final states. The large uncertainty in the $\ell\ell\Pe\Pgm$ 
final states results from the very low expected reducible background yields, 
which makes any comparison of the method susceptible to large statistical fluctuations.


\subsection{Theoretical Uncertainties for Higgs Boson}
The rate and acceptance uncertainties for the signal processes related to the 
theoretical calculations are due to uncertainties in the PDFs, variations of 
the QCD renormalization and factorization scales, and uncertainties in the 
modeling of parton showers. 
The magnitude of the rate uncertainty depends on the production process.
The rate uncertainties are found to be statistically insignificant when
measuring the rate of Higgs boson production.
The inclusive uncertainty related to the PDFs amounts to 1.9 and 1.6\%, 
respectively, for the $\PW\PH$, and $\PZ\PH$ production modes~\cite{deFlorian:2016spz}. The
corresponding uncertainty for the variation of the renormalization and 
factorization scales is 0.7 and 3.8\%, respectively~\cite{deFlorian:2016spz}.

The systematic uncertainties considered in the associated production
targeted analysis are summarized in Table~\ref{tab:vh_uncertainties}.

\begin{table*}[!ht]
\centering
\newcolumntype{x}{D{,}{\text{--}}{2.2}}
\begin{tabular}{ll}
Source of uncertainty & Magnitude \\
\hline
 $\tauh$ energy scale                & 1.2\% in energy scale\\
 $\Pe$ energy scale               & 1--2.5\%  in energy scale \\
 $\etvecmiss$ energy scale              & Dependent upon $\pt$ and $\eta$ \\
 $\tauh$ ID \& isolation & 5\% per $\tauh$  \\
 $\Pe$ ID \& isolation \& trigger  &   2\%  \\
 $\Pgm$ ID \& isolation \& trigger & 2\%  \\
 Diboson normalization & 5\% \\
 Integrated luminosity     & 2.5\%  \\
 b-tagged jet rejection & 4.5\% heavy flavor, 0.15\% light flavor \\
 Limited number of events                & Statistical uncertainty in individual bins  \\
 Signal theoretical uncertainty  & Up to 20\% \\
 Reducible background uncertainties & $\PW\PH$: shape and yield based \\
                                    & $\PW\PH$: 20\% yield \\
                                    & $\PZ\PH$: 26--100\% yield \\
\hline
\end{tabular}
\caption{Sources of systematic uncertainty}
\label{tab:vh_uncertainties}
\end{table*}



\section{Results}
\label{sec:vh_results}

The extraction of the results uses a global maximum likelihood fit based on a 
simultaneously fit of all the $\PW\PH$ and $\PZ\PH$ final state signal regions. 
Section~\ref{sec:vh_sr} shows the twelve signal region distributions
used in the global maximum likelihood fit. 
The global maximum likelihood fit results in a best fit signal
strength for this dedicated $\PW\PH$ and $\PZ\PH$ associated production analysis of
$\mu = 2.5 ^{+1.4} _{-1.3}$. This corresponds to 
a significance of 2.3 standard deviations while a significance of 1.0 standard deviation was expected.
Chapter~\ref{sec:cmb_results} discusses results for the combination of the
$ggH$ and VBF targeted analysis with the associated production targeted
analysis.

\subsection{Signal Region Details}
\label{sec:vh_sr}

In the $\PZ\PH$ final states, the $\mtt$ distribution is
used for signal extraction. The Low-$L_{\text{T}}^{\textrm{Higgs}}$ and
High-$L_{\text{T}}^{\textrm{Higgs}}$ regions are plotted side-by-side
in the following distributions. Figures~\ref{fig:zh_all_eight1}, ~\ref{fig:zh_all_eight2}, 
and \ref{fig:zh_results_svFitAll} show the $\mtt$ distributions 
for each of the $\PZ\PH$ final states and the combined distribution for 
all eight $\PZ\PH$ final states summed together. The eight $\PZ\PH$ final states are each 
fit separately in the global fit; combining them together helps reduce statistical
fluctuations for visualization purposes only.
The distributions are post-fit and show full uncertainties.
The $\PW\PH$ and $\PZ\PH$ signals are shown as 5x larger than their best-fit
value $\mu = 2.5$.

\begin{figure}[h!]
 \begin{center}
  \includegraphics[width=0.49\textwidth]{higgs_to_taus_vh/plots/zh/eeet_postfit.pdf}
  \includegraphics[width=0.49\textwidth]{higgs_to_taus_vh/plots/zh/emmt_postfit.pdf}
  \includegraphics[width=0.49\textwidth]{higgs_to_taus_vh/plots/zh/eemt_postfit.pdf}
  \includegraphics[width=0.49\textwidth]{higgs_to_taus_vh/plots/zh/mmmt_postfit.pdf}
 \end{center}
 \caption{The postfit $\mtt$ distributions used to extract the signal shown
  for the (top left) $\Pe\Pe\Pe\tauh$, (top right) $\Pgm\Pgm\Pe\tauh$, 
  (bottom left) $\Pe\Pe\Pgm\tauh$, and (bottom right) $\Pgm\Pgm\Pgm\tauh$
  final states. The final state is listed in the
  top left corner of each distribution.
  The distributions show full uncertainties.
  The $\PW\PH$ and $\PZ\PH$ signal are shown as 5x larger than their best-fit
  value $\mu = 2.5$.
 }
 \label{fig:zh_all_eight1}
\end{figure}

\begin{figure}[h!]
 \begin{center}
  \includegraphics[width=0.49\textwidth]{higgs_to_taus_vh/plots/zh/eett_postfit.pdf}
  \includegraphics[width=0.49\textwidth]{higgs_to_taus_vh/plots/zh/mmtt_postfit.pdf}
  \includegraphics[width=0.49\textwidth]{higgs_to_taus_vh/plots/zh/eeem_postfit.pdf}
  \includegraphics[width=0.49\textwidth]{higgs_to_taus_vh/plots/zh/emmm_postfit.pdf}
 \end{center}
 \caption{The postfit $\mtt$ distributions used to extract the signal shown
  for the (top left) $\Pe\Pe\tauh\tauh$, (top right) $\Pgm\Pgm\tauh\tauh$, 
  (bottom left) $\Pe\Pe\Pe\Pgm$, and (bottom right) $\Pgm\Pgm\Pe\Pgm$
  final states. The final state is listed in the
  top left corner of each distribution.
  The distributions show full uncertainties.
  The $\PW\PH$ and $\PZ\PH$ signal are shown as $5\times$ larger than their best-fit
  value $\mu = 2.5$.
 }
 \label{fig:zh_all_eight2}
\end{figure}

%\begin{figure}[h!]
% \begin{center}
%  \includegraphics[width=0.45\textwidth]{higgs_to_taus_vh/plots/zh/llet_postfit.pdf}
%  \includegraphics[width=0.45\textwidth]{higgs_to_taus_vh/plots/zh/llmt_postfit.pdf}
%  \includegraphics[width=0.45\textwidth]{higgs_to_taus_vh/plots/zh/lltt_postfit.pdf}
%  \includegraphics[width=0.45\textwidth]{higgs_to_taus_vh/plots/zh/llem_postfit.pdf}
% \end{center}
% \caption{The postfit $\mtt$ distributions used to extract the signal shown
%  for (top left) $\ell\ell\Pe\tauh$, (top right) $\ell\ell\Pgm\tauh$, 
%  (bottom left) $\ell\ell\tauh\tauh$, and (bottom right) $\ell\ell\Pe\Pgm$.
%  The left half of each distribution is the Low-$L_{\text{T}}^{\textrm{Higgs}}$ region
%  while the right half of each distribution is the High-$L_{\text{T}}^{\textrm{Higgs}}$ region.
%  $\ell\ell$ covers both $\PZ \to \Pgm\Pgm$ and $\PZ \to \Pe\Pe$ events.
%  The distributions show full uncertainties.
%  The $\PW\PH$ and $\PZ\PH$ signals are shown as 5x larger than their best-fit
%  value $\mu = 2.5$.
% }
% \label{fig:zh_results_svFitLLXX}
%\end{figure}


\begin{figure}[h!]
 \begin{center}
  \includegraphics[width=0.65\textwidth]{higgs_to_taus_vh/plots/zh/zh_postfit.pdf}
 \end{center}
 \caption{The postfit $\mtt$ distributions used to extract the signal shown
  for all 8 $\PZ\PH$ final states combined.
  The distribution shows full uncertainties.
  The left half of the distribution is the Low-$L_{\text{T}}^{\textrm{Higgs}}$ region
  while the right half corresponds to the High-$L_{\text{T}}^{\textrm{Higgs}}$ region.
  The $\PW\PH$ and $\PZ\PH$ signals are shown as 5x larger than their best-fit
  value $\mu = 2.5$.
 }
 \label{fig:zh_results_svFitAll}
\end{figure}



The results in the $\PW\PH$ final states are obtained from the distributions of the 
visible mass of the $\tauh$ leptons in the hadronic $\ell\tauh\tauh$ final states, 
and of the visible mass of the $\tauh$ and subleading light lepton in the 
semileptonic $\ell\ell\tauh$ final states. The mass distributions
are shown in Figure~\ref{fig:mass_llt} for the semileptonic final states 
and Figure~\ref{fig:mass_ltt} for the hadronic final states. 
Figure~\ref{fig:mass_wh} shows all four $\PW\PH$ final states combined together
for visualization purposes only.

\begin{figure}[h!]
 \begin{center}
  \includegraphics[width=0.49\textwidth]{higgs_to_taus_vh/plots/wh/emt_postfit.pdf}
  \includegraphics[width=0.49\textwidth]{higgs_to_taus_vh/plots/wh/mmt_postfit.pdf}
 \end{center}
 \caption{Postfit mass distributions in the $\Pe\Pgm\tauh$ (left) and 
 $\Pgm\Pgm\tauh$ (right) final states.
 The distributions show full uncertainties.
 The $\PW\PH$ and $\PZ\PH$ signals are shown as 5x larger than their best-fit
 value $\mu = 2.5$.
 }
 \label{fig:mass_llt}
\end{figure}

\begin{figure}[h!]
 \begin{center}
  \includegraphics[width=0.49\textwidth]{higgs_to_taus_vh/plots/wh/ett_postfit.pdf}
  \includegraphics[width=0.49\textwidth]{higgs_to_taus_vh/plots/wh/mtt_postfit.pdf}
 \end{center}
 \caption{Postfit mass distributions in the $\Pe\tauh\tauh$ (left) 
 and $\Pgm\tauh\tauh$ (right) final states.
 The distributions show full uncertainties.
 The $\PW\PH$ and $\PZ\PH$ signals are shown as 5x larger than their best-fit
 value $\mu = 2.5$.
 }
 \label{fig:mass_ltt}
\end{figure}

\begin{figure}[h!]
 \begin{center}
  \includegraphics[width=0.65\textwidth]{higgs_to_taus_vh/plots/wh/wh_postfit.pdf}
 \end{center}
 \caption{Postfit mass distributions of the four $\PW\PH$ final states
 combined together. 
 The distributions show full uncertainties.
 The $\PW\PH$ and $\PZ\PH$ signals are shown as 5x larger than their best-fit
 value $\mu = 2.5$.
 }
 \label{fig:mass_wh}
\end{figure}


\subsection{Analysis Sensitivity Details}
An excess of observed events with respect to the SM background expectation is 
visible in the most sensitive bins of the analysis, Figure~\ref{fig:sb}.
This distribution is created by grouping events in the signal regions 
by their decimal logarithm of the ratio of the 
signal ($S$) to signal-plus-background ($S+B$) in each bin

The postfit background and signal yields and the observed yields for the
$\PW\PH$ final states are shown in Table~\ref{tab:sb_wh} while those
for the $\PZ\PH$ final states are shown in Table~\ref{tab:sb_zh}. 
The $\PZ\PH$ final states are grouped according to the Higgs boson decay.

\begin{figure}[!ht]
 \begin{center}
  \includegraphics[width=0.55\textwidth]{higgs_to_taus_vh/plots/combined/wh_vs_zh_sbweight.pdf}
 \end{center}
 \caption{
 Distribution of the decimal logarithm of the ratio between the expected signal, 
 corresponding to the best fit value $\mu=2.5$, and the 
 sum of expected signal and expected background in each bin of the mass distributions 
 used to extract the results, in all signal regions. The background contributions are 
 separated based on the final states, $\PW\PH$ versus $\PZ\PH$. The inset 
 shows the corresponding difference between the 
 observed data and expected background distributions divided by the background expectation, 
 as well as the signal expectation divided by the background expectation.
 }
 \label{fig:sb}
\end{figure}

\begin{table*}
\centering
\begin{small}
\newcolumntype{x}{D{,}{\,\pm\,}{5.5}}
\begin{tabular}{lxxxx}
Process & \multicolumn{1}{c}{$\PW\PH, \Pe\Pgm\tauh$ } & \multicolumn{1}{c}{$\PW\PH, \Pgm\Pgm\tauh$ } & \multicolumn{1}{c}{$\PW\PH, \Pe\tauh\tauh$} & \multicolumn{1}{c}{$\PW\PH, \Pgm\tauh\tauh$}  \\
\hline
$\PZ\PZ$                  & 1.56, 0.05    & 0.93, 0.03  & 0.82, 0.04  & 1.18, 0.05   \\
$\PW\PZ$                  & 7.92, 0.28    & 6.69, 0.24  & 4.83, 0.25  & 8.38, 0.42   \\
Jet Fakes                 & 10.09, 1.61   & 12.19, 1.72 & 10.68, 1.27 & 19.80, 1.87  \\
Rare                      & 2.28, 0.61    & 3.77, 0.84  & 1.71, 1.08  & 1.76, 0.90   \\
Total backgrounds         & 21.85, 1.75   & 23.58, 1.92 & 18.04, 1.67 & 31.12, 2.12  \\
\hline
$\PW\PH, \PH \to\Pgt\Pgt$ & 4.28, 0.72    & 4.25, 0.73  & 3.51, 0.62  &  5.45, 0.97  \\
$\PZ\PH, \PH \to\Pgt\Pgt$ & 0.42, 0.07    & 0.40, 0.08  & 0.33, 0.07  &  0.44, 0.10  \\
Total signal              & 4.70, 0.72    & 4.65, 0.73  & 3.84, 0.92  &  5.98, 0.98  \\
\hline
%Observed &  \multicolumn{1}{c}{28 $\pm$ 5.3} &  \multicolumn{1}{c}{29 $\pm$ 5.4} &  \multicolumn{1}{c}{23 $\pm$ 4.8} &  \multicolumn{1}{c}{38 $\pm$ 6.2}  \\
Observed &  \multicolumn{1}{c}{28} &  \multicolumn{1}{c}{29} &  \multicolumn{1}{c}{23} &  \multicolumn{1}{c}{38}  \\
\hline
\end{tabular}
\end{small}
\caption{Background and signal expectations for the $\PW\PH$ final states, 
together with the number of observed 
events, for the post-fit signal region distributions.
The background uncertainty accounts for all sources of background uncertainty, 
systematic as well as statistical, after the global fit. The contribution from 
``Rare'' includes events from triboson, $\ttbar\PW$, $\ttbar\PZ$, $\ttbar\PH$ production,
and other rare processes.
}
\label{tab:sb_wh}
\end{table*}


\begin{table*}
\centering
\begin{small}
\newcolumntype{x}{D{,}{\,\pm\,}{5.5}}
\begin{tabular}{lxxxx}
Process & \multicolumn{1}{c}{$\ell\ell\Pe\tauh$} &  \multicolumn{1}{c}{$\ell\ell\Pgm\tauh$} &  \multicolumn{1}{c}{$\ell\ell\tauh\tauh$} &  \multicolumn{1}{c}{$\ell\ell\Pe\Pgm$} \\
\hline
$\PZ\PZ$                        & 14.40, 0.36 & 26.91, 0.55 & 25.58, 1.05 & 9.33, 0.18 \\   
Jet Fakes                       & 14.01, 1.55 & 17.58, 1.17 & 58.05, 2.87 & 3.66, 4.60 \\
Rare                            & 0.62, 0.08  & 1.54, 0.61  & 0.81, 0.42  & 3.02, 0.23 \\
Total backgrounds               & 29.03, 1.59 & 46.03, 1.43 & 84.44, 3.08 & 16.01, 4.61\\             
\hline
$\PW\PH, \PH \to\Pgt\Pgt$       & 0.008, 0.002  & 0.01, 0.003  & 0.016, 0.005  & 0.002, 0.001 \\
$\PZ\PH, \PH \to\Pgt\Pgt$       & 2.83, 0.39  & 5.31, 1.30  & 5.29, 1.17  & 1.62, 0.20 \\
Total signal                    & 2.84, 0.39  & 5.32, 0.70  & 5.31, 1.17  & 1.62, 0.20 \\
\hline
%Observed &  \multicolumn{1}{c}{33 $\pm$ 5.75} &  \multicolumn{1}{c}{53 $\pm$ 7.28} &  \multicolumn{1}{c}{87 $\pm$ 9.33} &  \multicolumn{1}{c}{20 $\pm$ 4.47}  \\
Observed &  \multicolumn{1}{c}{33} &  \multicolumn{1}{c}{53} &  \multicolumn{1}{c}{87} &  \multicolumn{1}{c}{20}  \\
\hline
\end{tabular}
\end{small}
\caption{Background and signal expectations for the $\PZ\PH$ final states, 
together with the number of observed 
events, for the post-fit signal region distributions. The $\PZ\PH$ final states
are each grouped according to the Higgs boson decay products. 
$\ell\ell$ covers both $\PZ \to \Pgm\Pgm$ and $\PZ \to \Pe\Pe$ events.
The background uncertainty accounts for all sources of background uncertainty, 
systematic as well as statistical, after the global fit. The contribution from 
``Rare'' includes events from triboson, $\ttbar\PZ$, $\ttbar\PH$ production,
and other rare processes.
}
\label{tab:sb_zh}
\end{table*}



\clearpage

%\chapter{Combined $\htt$ Results}
\label{sec:cmb_results}

The previous two Chapters, ~\ref{sec:htt_analysis} and ~\ref{sec:vh_analysis}, discussed
analyses targeted at specific Higgs boson production mechanisms. In this chapter
I present results combining the 
$ggH$ and VBF targeted analysis~\cite{cms_13TeV_htt_jhep_2017}
with the $\PW\PH$ and $\PZ\PH$ associated production targeted analysis~\cite{HIG-18-007}.
By combining these two $\htt$ analyses we have signal regions targeting the leading Higgs 
boson production processes: $ggH$, VBF, and $\PW\PH$ and $\PZ\PH$ associated production. 
Combining the results, signal strengths, the $\htt$ significance and, Higgs
boson couplings can be probed with greater precision than either analysis alone.

Changes in the $ggH$ signal modeling and uncertainties were made between
publication of the $ggH$ and VBF targeted analysis~\cite{cms_13TeV_htt_jhep_2017} 
and the combination here to take advantage of the most accurate, available
simulations~\cite{CMS-PAS-HIG-17-031}.
The $ggH$ and VBF results presented in Chapters~\ref{sec:htt_analysis} uses $ggH$ simulated with NLO 
accuracy (Chapter~\ref{sec:simulation}). In this combination, 
the NLO $ggH$ samples were reweighted using the \textsc{NNLOPS} generator which is accurate at
the next-to-next-to-leading order in the strong coupling~\cite{Hamilton2013}.
The reweighting matches the Higgs boson
$\pt$ spectrum and the quantity of jets from the hard-scattering process between these generators
and is defined to preserve the normalization
of the inclusive $ggH$, $\htt$ simulated process~\cite{CMS-PAS-HIG-17-031}. 
Additionally, the
$ggH$ cross section uncertainty scheme has been updated to align 
with the one proposed in Reference~\cite{deFlorian:2016spz}.
This uncertainty scheme includes 9 nuisance parameters accounting for uncertainties in the cross 
section prediction for exclusive jet bins, the 2 jet and 3 jet VBF phase spaces, different Higgs boson $\pt$ 
regions, and the uncertainty in the Higgs boson $\pt$ distribution due to missing higher order finite 
top quark mass corrections.


\section{Signal Strength and Significance}
The best fit signal strength ($\mu$) and significance can be computed for the combination and
leads to a decrease in the relative uncertainty on $\mu$ and an increase
in significance. The $\mu$ and significance
for each analysis and the combined values are presented in Table~\ref{tab:cmb_mu_and_sig}.
The slight excess in signal strength for the associated production analysis is tempered
when combined with the $ggH$ and VBF analysis, resulting in a $\mu$ fully consistent 
with the SM. The combination leads to an 
observed significance of 5.5 standard deviations (4.8 expected), surpassing the threshold for a
purely 13\TeV based CMS observation of the $\htt$ process.

\begin{table*}[htbp]
\renewcommand{\arraystretch}{1.3}
\centering
\begin{tabular}{lcc}
Analysis         &   Best Fit Signal Strength    &   Observed Significance    \\
\hline
$ggH$ and VBF             &   $\mu = 1.09 ^{+0.27} _{-0.26}$   &  4.9 $\sigma$     \\
Associated production     &   $\mu = 2.5  ^{+1.4}  _{-1.3}$    &  2.3 $\sigma$     \\
$\htt$ combination        &   $\mu = 1.24 ^{+0.29} _{-0.27}$   &  5.5 $\sigma$     \\
\hline
\end{tabular}
\caption{
Best fit signal strength and significance for three fit scenarios: $ggH$ and VBF,
associated production, and the combination with the updated $ggH$ modeling.
}
\label{tab:cmb_mu_and_sig}
\end{table*}


The signal strength can be decomposed into the four leading Higgs boson production 
mechanisms. Figure~\ref{fig:cmb_mu_higgs_processes} shows this decomposition for
the combined results. 

\begin{figure}[!ht]
 \begin{center}
  \includegraphics[width=0.65\textwidth]{higgs_to_taus_vh/plots/combined/mu_higgs_procs.pdf}
 \end{center}
 \caption{
 Best fit signal strength per Higgs boson production process, for $\mH = 125.09\GeV$.
 The constraints from the combined global fit are used to extract each of the 
 individual best fit signal strengths. The combined best fit signal strength 
 is $\mu = 1.24 ^{+0.29} _{-0.27}$.
 }
 \label{fig:cmb_mu_higgs_processes}
\end{figure}



\section{Higgs Boson Couplings}
The Higgs boson couplings to different particles can be measured in different
ways. In this thesis the couplings are split into two groups, couplings to
fermions ($\kappa_\text{f}$) and couplings to vector bosons ($\kappa_\text{V}$) 
and are measured using couplings on both the production and decay processes. 
$\kappa_\text{V}$ and $\kappa_\text{f}$ quantify
the ratio between the measured and the SM value for the couplings of the Higgs boson
using the methods described in Reference~\cite{Khachatryan:2016vau}.

The $\htt$ decay process
provides direct access to the Higgs boson fermion couplings. While all
$\htt$ events provide access to fermionic couplings on the decay side,
the different production mechanisms provide access to different couplings on that side.
The $ggH$ process provides indirect access to fermionic couplings through
the top-quark loop. The VBF and associated production processes provide access to the
vector boson couplings via Higgs boson production.
The same combination of the dedicated $ggH$ and VBF analysis, including
the updates to the $ggH$ modeling, with
the dedicated $\PW\PH$ and $\PZ\PH$ analysis can place the tightest
$\htt$ analysis limits in the $(\kappa_\text{V}$,$\kappa_\text{f})$ parameter space
because there is simultaneous access to processes which can constrain both
couplings.

To measure the couplings, a likelihood scan is performed for $\mH=125.09\GeV$ in 
the ($\kappa_\text{V}$,$\kappa_\text{f}$) parameter space.
For this scan only, Higgs boson decays to pairs of $\PW$ or $\PZ$ bosons, $\hww$ or $\hzz$,
are considered as part of the signal. They both contributed to $\kappa_\text{V}$ on the
decay side and are treated accordingly based on their production processes.
%The $\ttbar\PH$ production process is still treated as background because
%the MC sample we use is not split by Higgs boson decay mode. 
All nuisance 
parameters are profiled for each point of the scan. As shown in 
Figure~\ref{fig:cmb_kFkV}, the observed likelihood contour is consistent with the SM expectation 
of $\kappa_\text{V}$ and $\kappa_\text{f}$ both equal to unity showing
agreement with the SM Higgs boson couplings to fermions and vector bosons.

\begin{figure}[!ht]
 \begin{center}
  \includegraphics[width=0.60\textwidth]{higgs_to_taus_vh/plots/combined/kFkV_HIG-18-007_plus_HIG-16-043_comp_up.pdf}
 \end{center}
 \caption{Scan of the negative 
 log-likelihood difference as a function of $\kappa_V$ and $\kappa_f$, for 
 $\mH = 125.09$\GeV.  All nuisance parameters are profiled for each point. 
 This scan is a combination of the $ggH$ and VBF targeted analysis with the 
 $\PW\PH$ and $\PZ\PH$ targeted analysis. For reference, the results for just
 the $ggH$ and VBF targeted analysis are also presented and correspond
 to the updated $ggH$ modeling.
 For this scan, all $\hww$ and $\hzz$ processes 
 are treated as signal.
 }
 \label{fig:cmb_kFkV}
\end{figure}



\clearpage

%\chapter{Conclusions}
\label{sec:conclusion}

In summary...

\section{HTT Summary}
The 13\TeV $\htt$ analysis using data gathered by the CMS experiment in 2016 discussed here provides the first 
single-experiment observation of the SM Higgs boson decaying to fermions. This is a very important benchmark for the CMS
experiment and the high energy particle physics commuinty as a whole. There is still room to improve this
measurement using the 2016 13\TeV center-of-mass energy data collected by the CMS experiment. The analysis
detailed here did not include any channels dedicated to targeting Higgs bosons produced in associated production.
The following chapter discusses the $\PW\PH$ and $\PZ\PH$ associated production $\htt$ analysis.



\section{VH Summary}
A search for the standard model Higgs boson based on data collected in proton-proton collisions by the
CMS detector in 2016 at a center-of-mass energy of 13\TeV focusing on the
two $\PW\PH$ and $\PZ\PH$ associated production processes has been presented. Event
categories have been split into three lepton final states targeting $\PW\PH$ production
and four lepton final states targeting $\PZ\PH$ production. The results are extracted
via maximum likelihood fits using the visible di-$\Pgt$ mass for the $\PW\PH$
final states and full di-$\Pgt$ mass for the $\PZ\PH$ final states. 
%Observed limits of 4.7 
%(expected 2.0) are placed on the Higgs boson associated production processes 
%times the SM prediction for a Higgs boson mass of 125.09\GeV. 
The best fit signal
strength is $\mu = 2.54 ^{+1.35} _{-1.26}$ ($\mu = 1.00 ^{+1.08} _{-0.97}$ expected) 
for a significance of 2.3 standard deviations (1.0 expected).

Combining this analysis with the previous 13\TeV $ggH$ and VFB targeted $\htt$ 
analysis~\cite{cms_13TeV_htt_jhep_2017}, we place the tightest constraints
on the $\htt$ process. 
The best fit signal strength is $\mu = 1.24 ^{+0.29} _{-0.27}$ leading to an
observed significance of 5.5 standard deviations (4.8 expected). 
The combination leads to a significant increase in constraint for the coupling
of the Higgs boson to vector bosons, the coupling to fermions is not greatly
affected. The resulting measured couplings are consistent with SM predictions
within one standard deviation.


\bibliographystyle{unsrt}
\bibliography{thesis_bibliography}

\end{document}
