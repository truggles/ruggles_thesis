\chapter{Introduction}
\label{sec:introduction}

In this thesis, I discuss the observation of the Higgs boson decaying to a pair
of tau leptons ($\Pgt^{+}\Pgt^{-}$). The thesis includes two analyses which target different
ways in which the Higgs boson is produced. The analyses a performed using
13\TeV center-of-mass energy proton-proton collision data from the CERN LHC.
The data is gathered by the CMS experiment and corresponds to an integrated
luminosity of 35.9\fbinv. The observation here is the first observation of the
Higgs boson decay to $\Pgt^{+}\Pgt^{-}$ at 13\TeV and constitutes an important
milestone in the effort to better understand the Higgs boson, one of the fundamental
particles of the Standard Model (SM) of physics, and the fundamental
properties of nature.

The Standard Model of physics is a mathematical framework for explaining
the interactions and behavior of the fundamental particles observed in nature.
It has been built up and defined through the 1950s and 60s culminating in the
theoretical prediction of the existance of a neutral scalar boson, now called the
Higgs boson.
The SM incorporates descriptions of three of the four fundamental forces of nature:
the strong force, the electromagnetic force, and the weak force.
The force which is not described in the SM is the gravitational force. This omission is one of the reasons
physicists are working to both develop and experimentally test more comprehensive 
theories of nature.

The Higgs boson eluded observation by experimental particle physicsts
for about 40 years after the establishement of its theoretical prediction. 
The Higgs boson was discovered in 2012
by the CMS and ATLAS collaborations at CERN~\cite{Aad:2012tfa, Chatrchyan:2012xdj, Chatrchyan:2013lba}.
With this discovery, all particles predicted and described in the SM have been observed.
Based on research leading up to today, the SM is the most well tested theory of nature at the fundamental level.
Over all, the SM shows remarkable continuity between the theoretical predictions and
the experimental observations. 

Reflecting on the discovery of the Higgs boson, the focus of the experimental particle physics community has transitioned from
Higgs boson ``discovery'' mode to Higgs boson ``measurement'' mode. The leading high energy particle physics experiments
are dedicating a vast portion of their research effort and person power towards
efforts to measure the Higgs boson properties as precisely as possible. Many details of the properties of the Higgs
boson are firmly predicted by theory. Affirmation or negation of these predictions,
such as how often a Higgs boson decays into a pair of $\Pgt$ leptons,
are critical to further testing the merits of the SM. Affirmation of the
predicted Higgs boson properties would further support the SM along with the myriad previous
experimental results. Significant discrepancies between the SM
theoretical predictions and the observed Higgs boson properties could point to
flaws in the SM and would lead to a more full and complete understanding of nature.

This thesis is arranged to build up a theoretical understand of the SM and the
Higgs boson in Chapter~\ref{sec:pheno}. A description of the LHC machine and
CMS detector follows in Chapter~\ref{sec:cms_lhc}. The next two Chapters, ~\ref{sec:simulation} and
~\ref{sec:obj_reconstruction}, discuss the foundations of physics analyses at CMS, the physics
simulations and details of how detector information is processed to created tau
leptons and other particles. Chapter~\ref{sec:htt_analysis} covers the analysis
of the Higgs boson when produced via the gluon fusion or vector boson fusion
processes and Chapter~\ref{sec:vh_analysis} covers the Higgs boson associated
production processes. In Chapter~\ref{sec:cmb_results}, the results of
both analysis are combined together to achieve the strongest 13\TeV Higgs
boson decaying to $\Pgt^{+}\Pgt^{-}$ results possible. Finally, the thesis
is drawn to a conclusion in Chapter~\ref{sec:conclusion}.
The mentioned Higgs boson decay process to a pair of $\Pgt$ leptons will be denoted
$\htt$ through out this thesis where the $\pm$ are dropped from the $\Pgt$ superscript
for convenience.


\section{The Standard Model of Particle Physics}
The Standard Model of particle physics~\cite{Glashow:1961tr,SM1,SM3} is currently
the best framework for mathematical predictions and explanations of the behavior
of the fundamental particles of nature. These fundamental particles can be grouped
together by behavoir and common characteristis and arranged diagramatically as in
Figure~\ref{fig:sm_particles}. The particles are split vertically into two categories:
fermions which have spin--$\frac{1}{2}$, and the bosons which have integer
spin of either 0 or 1. In general, the fermions constitute what we are familiar of
as matter while the bosons are the mediators of the fundamental forces.
The rest of this chapter introduces the particles and forces of the SM as well
as preivous experimental results. For a more detailed mathematical treatement
of the phenomology of SM see the following Chapter~\ref{sec:pheno}.

\begin{figure*}[htbp]
\centering
     \includegraphics[width=0.7\textwidth]{introduction/plots/sm_particles.pdf}
     \caption{
The fundamental particles of the SM and some of their properties including their:
mass, electric charge, and spin. The units for mass are reported as electron volts divided by
the speed of light ($c$) squared and use scientific notation prefixes.
M for million, G for billion.
     }
     \label{fig:sm_particles}
\end{figure*}

The fermions can be further grouped into either quarks or leptons based on whether
they carry ``color'' charge or not. 
Quarks carry a color charge and have an electric charge of either $-\frac{1}{3}e$ or $\frac{2}{3}e$
where $e$ is the elementary charge ($1.602 \times 10^{-19}$ Coulombs).
Where as the leptons are colorless (carry no color charge) and
have integer electric charge of 0$e$ or -1$e$. In Figure~\ref{fig:sm_particles} the fermions are
arranged according to what is called their mass ``generation'' with more massive
particles appearing to the right in the third mass generation column.

The first
mass generation column composes the fundamental particles which we interact with
every day. Up-quarks and down-quarks are the fundmanetal piecies within
the protons and neutrons building the atoms which contribute to the molecules which
make up the paper pages of this thesis or your computer screen. High energy protons
are a little different and are discussed later in the thesis. Electrons are the
remaining fundamental particles we are familiar with and are also part of the basic 
structure of atoms. The electron neutrino is less familiar because it only
interacts with the other particles through the weak force and does not directly
contribute to the basic atoms which compose the matter from which we are built.

The bosons are split into two groups, the gauge bosons which mediate the three fundamental
forces covered by the SM, and the solo Higgs boson which behaves differently
and will be discussed in detail later. 
The fundamental forces, their mediator particles, and the relative strength of the force
are listed in Table~\ref{tab:sm_forces}. The relatively weak strength of the gravitational
force is what allows the SM to still successfully predice the most basic behaviors
of particles despite not including the gravitational force.

\begin{table*}[htbp]
\centering
\begin{tabular}{lcc}
Fundamental Force        &    Force Mediator             & Relative Strength   \\
\hline
Strong                   &    gluon ($\Pg$)              &   1                 \\ 
Electromagnetic          &    photon ($\Pgg$)            &   $10^{-3}$         \\ 
Weak                     &    $\PW$ and $\PZ$ bosons     &   $10^{-14}$        \\ 
Gravitational            &    unknown                    &   $10^{-43}$        \\ 
\hline
\end{tabular}
\caption{
The fundamental forces, their mediator particles, and the relative strength of the force.
There has been no observed mediator for the gravitational force.
}
\label{tab:sm_forces}
\end{table*}

The strong force has the largest relative strength of the fundamental forces but
the reach or distance over which the force can be felt is very limited and is 
confined to the sub-atomic scale, $10^{-15}$m. The strong force is experienced
between particles with a color charge, exclusively gluons and quarks.
Figure~\ref{fig:sm_particles} shows a diagram of the SM particles with blue
lines linking particles which experience a force with the mediating boson particle.
For example, a blue line links the quarks with the gluons representing the
strong force.

The electromagnetic force follows after the strong force in order of largest relative
strength. The reach of the electromagnetic force is infinite and decreases with
distance as $\frac{1}{r^{2}}$. Despite its infinite reach, the electromagnetic 
force is not experienced on the macroscopic scale because all stable matter is composed
of both electrically positively charged and negatively charged matter and leads to
an overall neutral electrical charge for the universe. The electromagnetic force is experienced by all
electrically charged particles: quarks, the non-neutrino leptons, and the $\PW^{\pm}$ bosons.
This force is mediated by the photon which has neutral electric charged.
% XXX Add example of electrons holding you up in your chair, preventing you from
% falling through it?

The next force in descending order of relative strength is the weak force.
The weak force is experienced by
all of the leptons and the quarks and is mediated by the $\PW^{\pm}$ and $\PZ$ 
bosons. It is responsible for familiar phenomena such as the radioactive decay of atoms. 
Beta decay is one example of radioactive decay where, within an atomic nucleus,
a neutron is transformed into a proton and an electron and an electron antineutrino
(more on antiparticles following). Fundamentally, what happens
is the quark composition within the proton changes thereby changing the proton to a
neutron. This process is mediated by a $\PW^{-}$ boson which subsequently decays
to an electron and the antielecton neutrino.

The final and weakest force is the one we are most familiar with, the gravitational
force. Just like the electromagnetic force, the reach of the gravitational force 
is infinite and decreases with distance as $\frac{1}{r^{2}}$. Yet, unlike the
electromagnetic force, gravity is felt over extremly large distances. This is because
gravity is a purely attractive force which acts on all massive particles.
It is amazing that the force we are most colloquially familiar with is actually the
weakest of the four fundamental forces. Because the gravitational force is so weak
when talking about the effect on a single particle, it can be ignored for
all of the particle physics calculations throughout this thesis.

\begin{figure*}[htbp]
\centering
     \includegraphics[width=0.7\textwidth]{introduction/plots/elementary_particle_interactions_SM.pdf}
     \caption{
Diagram showing the bosons arranged into a central column with the fermions in the
upper corners. The blue lines linking particles and groups of particles together
indicate that those fermions can be influenced by force associated to that mediator
boson. The Higgs boson in the center is discussed in Chapter~\ref{sec:pheno}.
     }
     \label{fig:sm_particles}
\end{figure*}

In addition to the particles shown in Figure~\ref{fig:sm_particles}, there exist antiparticles.
which were mentioned in the context of an example of the weak force, beta decay.
Each of the SM particles has an antiparticle, though some particles, such as the photon
are their own antiparticle. Antiparticles have the same mass as their
``normal'' particle pair except they have opposite electric charge. The antiparticle
partner of the electron is the positron which is sometimes called an antielectron.
Antiparticles can be created in many types of interactions in particle physics experiments
and are very common. A particle and its antiparticle can annihilate each other resulting in,
for example, a photo. The photon is a possible result because of the neutral overall charge
of a particle--antiparticle pair, and total charge must be conserved.
Antiparticles are denoted in this thesis with a ``bar'' over the top of a particle symbol. For example
a top-quark is $t$ while an antitop-quark is $\bar{t}$.

For a more thorough treatment of the SM, its particles, and its forces see
Chapter~\ref{sec:pheno}.



\section{The Standard Model: Experimental Context}
The gap between the prediction of the Higgs boson and its discovery was a long 40 years.
Many of the particles making up the SM were not discovered when the Higgs boson
was originally being theorized. In fact, the existance of quarks or the discovery 
of gluons, the mediator of the strong force, or the $\PW$ and the $\PZ$ bosons,
the mediators of the weak force, we all still to be found. The decades after the
1960s saw discovery after discovery, slowly piecing together and validating
the SM.

The internal structure of protons was illuminated by
deep inelastic scatter experiments carried out at SLAC which eventually led to 
the observation of the three least massive quarks: up ($u$), down ($d$), and strange ($s$)
~\cite{PhysRevLett.23.930,Breidenbach:1969kd}. In 1974, the $J/\Psi$ particle, a composit 
particle made from a charm quark ($c$) and a charm anti-quark ($\bar{c}$) was 
discovered~\cite{PhysRevLett.33.1404,PhysRevLett.33.1406}. 
The bottom quark ($b$) was discovered in 1977 via the decays of a new particle, the Upsilon
meson~\cite{PhysRevLett.39.252}. The top quark ($t$) was the last quark of the three
known generations discovered
and was not found until 1995 at Fermilab~\cite{PhysRevLett.74.2626,PhysRevLett.74.2632}.
The gluon which mediates the strong force for all of the quarks was not discovered until 
1979 at DESY~\cite{PhysRevLett.43.830}.

Beyond the partons there were discoveries of new bosons, speciically the mediators
of the weak force.
In 1983 the $\PW$ and the $\PZ$ bosons were discovered~\cite{AUBERT1983275,1983398}. 
These two bosons were the
most massive fundamental particles by far at the time of their discovery with values of
84.4\GeV and 91.2\GeV respectively.
A very important discovery laying the foundation for the analyses in this thesis was
the discovery of the third generation charged lepton. The $\Pgt$ lepton was discovered
in 1975 by Martin Perl~\cite{PhysRevLett.35.1489}. While far from an exhaustive list,
these many discoveries give an indication of the very strong background of experimental
research which supports the SM.

\subsection{Higgs Boson Searches: Experimental Results}
As more piecies of the SM were validated and particle accelerators became more powerful
searches for the Higgs boson were conducted at multiple experiments.
Searches for a Higgs boson were performed at the 
LEP at CERN~\cite{Barate:2000ts,Abdallah:2003ip,Achard:2001pj,Abbiendi:2000ac}.
In the datasets corresponding to these searches, there were few enough potential Higgs
boson events that no discoveries could be made.
Instead, these searchs all resulted in placing limits on the possible mass of the Higgs boson.
The Tevatron at Fermilab was active in Higgs boson searches through the early 2000s
with multiple analyses targeting the same decay process studied in this thesis, the Higgs
boson decaying to a pair of $\Pgt$ leptons. Similar to the LEP results, these analyses
placed limits on the possible mass and cross section of the Higgs boson~\cite{Aaltonen:2012jh, Abazov:2012zj}.

%FIXME XXX XXX

Using $\Pp\Pp$ collision data at $\sqrt{s}=7$ and $8\TeV$, the CMS Collaboration showed evidence for this process with an observed\,(expected)
significance of 3.2\,(3.7) standard deviations (s.d.)~\cite{Chatrchyan:2014nva}. The ATLAS
experiment reported evidence for Higgs bosons decaying into pairs
of $\Pgt$ leptons with an observed (expected) significance of 4.5 (3.4)
s.d. for a Higgs boson mass of 125\GeV~\cite{Aad:2015vsa}.
The combination of the results from both experiments yields an observed (expected)
significance of 5.5\,(5.0) s.d.~\cite{Khachatryan:2016vau}.


A particle compatible with such a boson was observed by the ATLAS and CMS experiments at the CERN LHC
in the $\PZ\PZ$, $\Pgg \Pgg$, and $\PW\PW$ decay channels~\cite{Aad:2012tfa, Chatrchyan:2012xdj, Chatrchyan:2013lba},
during the proton-proton ($\Pp\Pp$) data taking period in 2011 and 2012
at center-of-mass energies of $\sqrt{s} = 7$ and 8\TeV, respectively.

Subsequent results from both experiments, described in
Refs.~\cite{Aad:2015gba, Khachatryan:2014jba, Chatrchyan:2012jja, Aad:2013xqa, Khachatryan:2014kca,Sirunyan:2017exp},
established that the measured properties of the new particle,
including its spin, CP properties,
and coupling strengths to SM particles, are consistent with those expected for the Higgs boson predicted by the SM.
The mass of the Higgs boson has been determined to be
$125.09\pm0.21\stat\pm0.11\syst\GeV$, from a combination of
ATLAS and CMS measurements~\cite{Aad:2015zhl}.










Wesley -
One more point about the introduction. It should introduce your thesis topic.
It should explain how your topics fit into or expand the standard model and
how your topic(s) enhance our understanding beyond what preceded. You
need to cover both the theoretical and experimental context (e.g. include
a summary of the measurements that came before).

Dasu - 
As a hint, the intro should be accessible to non specialist. So, keep it simple, written in English rather than CMSish. You should introduce SM as the most well tested theory of nature at the fundamental level. The Higgs boson completes it.


