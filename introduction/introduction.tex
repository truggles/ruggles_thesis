\chapter{Introduction}

Wesley -
One more point about the introduction. It should introduce your thesis topic.
It should explain how your topics fit into or expand the standard model and
how your topic(s) enhance our understanding beyond what preceded. You
need to cover both the theoretical and experimental context (e.g. include
a summary of the measurements that came before).

Dasu - 
As a hint, the intro should be accessible to non specialist. So, keep it simple, written in English rather than CMSish. You should introduce SM as the most well tested theory of nature at the fundamental level. The Higgs boson completes it.


FIXME
In the standard model (SM) of particle physics~\cite{Glashow:1961tr,SM1,SM3},
electroweak symmetry breaking is achieved via the Brout--Englert--Higgs
mechanism~\cite{Englert:1964et,Higgs:1964ia,Higgs:1964pj,Guralnik:1964eu,Higgs:1966ev,Kibble:1967sv},
leading, in its minimal version, to the prediction of the existence of one physical neutral scalar particle,
commonly known as the Higgs boson ($\PH$).
A particle compatible with such a boson was observed by the ATLAS and CMS experiments at the CERN LHC
in the $\PZ\PZ$, $\Pgg \Pgg$, and $\PW\PW$ decay channels~\cite{Aad:2012tfa, Chatrchyan:2012xdj, Chatrchyan:2013lba},
during the proton-proton ($\Pp\Pp$) data taking period in 2011 and 2012
at center-of-mass energies of $\sqrt{s} = 7$ and 8\TeV, respectively.
Subsequent results from both experiments, described in
Refs.~\cite{Aad:2015gba, Khachatryan:2014jba, Chatrchyan:2012jja, Aad:2013xqa, Khachatryan:2014kca,Sirunyan:2017exp},
established that the measured properties of the new particle,
including its spin, CP properties,
and coupling strengths to SM particles, are consistent with those expected for the Higgs boson predicted by the SM.
The mass of the Higgs boson has been determined to be
$125.09\pm0.21\stat\pm0.11\syst\GeV$, from a combination of
ATLAS and CMS measurements~\cite{Aad:2015zhl}.

To establish the mass generation mechanism for fermions,
 it is necessary to probe the direct coupling of
the Higgs boson to such particles.
The most promising decay channel is $\Pgt^+\Pgt^-$,
because of the large event rate expected in the SM compared to the $\Pgm^+\Pgm^-$ decay channel ($\mathcal{B}(\PH\to\Pgt^+\Pgt^-)=6.3$\% for a mass of 125.09\GeV), and of the smaller contribution from background events
with respect to the $\bbbar$ decay channel.

Searches for a Higgs boson decaying to a $\Pgt$ lepton pair were performed at the LEP~\cite{Barate:2000ts,Abdallah:2003ip,Achard:2001pj,Abbiendi:2000ac},
Tevatron~\cite{Aaltonen:2012jh, Abazov:2012zj}, and LHC colliders.
Using $\Pp\Pp$ collision data at $\sqrt{s}=7$ and $8\TeV$, the CMS Collaboration showed evidence for this process with an observed\,(expected)
significance of 3.2\,(3.7) standard deviations (s.d.)~\cite{Chatrchyan:2014nva}. The ATLAS
experiment reported evidence for Higgs bosons decaying into pairs
of $\Pgt$ leptons with an observed (expected) significance of 4.5 (3.4)
s.d. for a Higgs boson mass of 125\GeV~\cite{Aad:2015vsa}.
The combination of the results from both experiments yields an observed (expected)
significance of 5.5\,(5.0) s.d.~\cite{Khachatryan:2016vau}.

This Letter reports on a measurement of the $\PH\to\Pgt\Pgt$ signal strength.
The analysis targets both the gluon fusion and the vector boson fusion production mechanisms.
The analyzed data set corresponds to an integrated luminosity of 35.9\fbinv, and was collected in 2016 in $\Pp\Pp$ collisions at a center-of-mass energy of
13\TeV.
In the following, the symbol $\ell$ refers to electrons or muons,
the symbol $\tauh$ refers to $\Pgt$ leptons reconstructed in their hadronic decays, and
$\PH\to\Pgt^+\Pgt^-$  and $\PH\to\PW^+\PW^-$ are simply denoted as $\PH\to\Pgt\Pgt$  and $\PH\to\PW\PW$, respectively.
All possible $\Pgt\Pgt$ final states are studied, except for those with two muons or two electrons because of the low branching fraction and large background contribution. The analysis covers about 94\% of all possible $\Pgt\Pgt$ final states.

\section{Theory Context: Standard Model of Particle Physics}

\subsection{Electroweak Symmetry Breaking}

\subsubsection{W/Z Higgs Associated Production}

\subsection{Cross Sections and Decay Rates}

\subsection{QCD and Proton Structure}

\section{Experimental Context: Previous Higgs Measurements}

%\subsection{D0 and CDF}

\subsection{CMS and ATLAS at 7 and 8 TeV}





