\chapter{Analysis Strategy: Higgs Produced in Associated Production with $\PW/\PZ$}
\label{sec:vh_analysis}

\section{Overview: ZHiggs to $\ell\ell\tau\tau$}
\subsection{Channels}
\section{Data Set}
\section{MC Samples}
\section{Triggers}
\section{SVFit Algorithm}
\section{Background Estimation}
\section{Systematic Uncertainties}
\subsection{Uncertainties related to object reconstruction and identification}
\subsection{Background estimation uncertainties}
\subsection{Signal prediction uncertainties}
\subsection{Other uncertainties}
\subsection{Luminsity}
\subsection{Lepton ID and Isolation}
\subsection{Tau Fake Rate}
\subsection{Energy Scales}
\subsubsection{Tau Energy Scale}
\subsubsection{Jet Energy Scale}
\subsubsection{MET Energy Scale}
\subsection{Theoretical Uncertainties for Higgs Boson}


\section{Introduction\label{sec:intro}}

This paper summarizes the search for the SM Higgs boson 
produced in association with a $\PW$ or $\PZ$ boson in proton-proton
collisions at $\sqrt{s}=13$\TeV using a data set collected 
in 2016 with the CMS experiment corresponding to a total integrated luminosity of 35.9\fbinv.
In this search, the SM Higgs boson decays to a pair of $\Pgt$ leptons. The symbol $\tauh$ denotes $\Pgt$ leptons decaying hadronically.
For the $\PZ\PH$ search, $\PZ\rightarrow \Pe\Pe$
and $\PZ \to \Pgm\Pgm$ decays are considered combined with four possible $\Pgt\Pgt$ 
final states from the Higgs boson decay: $\Pe\tauh$, $\Pgm\tauh$,
$\Pe\Pgm$ and $\tauh\tauh$. For the $\PW\PH$ search, four final states are considered with
the $\PW$ boson decaying leptonically to an electron or muon, and at least one $\tauh$ from the Higgs leptons:
$\Pgm\Pgm\tauh$, $\Pe\Pgm\tauh$, $\Pe\tauh\tauh$ and $\Pgm\tauh\tauh$. 



\section{Simulated samples}

Signal and background processes are modeled with samples of simulated events.
The signal samples with a Higgs boson produced in association with a $\PW$ or 
$\PZ$ boson ($\PW\PH$ or $\PZ\PH$), are generated at next-to-leading order 
(NLO) in perturbative quantum chromodynamics (pQCD) with the \POWHEG 
2.0~\cite{Nason:2004rx,Frixione:2007vw, Alioli:2010xd, Alioli:2010xa, Alioli:2008tz} 
generator where the \textsc{minlo hvJ}~\cite{Luisoni:2013kna} extension of 
\POWHEG 2.0 is used. The set of parton distribution functions (PDFs) is 
NPDF30\_nlo\_as\_0118~\cite{Ball:2011uy}. Because the analysis focuses on
measuring the $\PW\PH$ and $\PZ\PH$ processes, the $\ttbar\PH$ process is 
included as background. The \pt distribution of 
the Higgs boson in the {\POWHEG2.0} simulations is tuned to match more closely
the next-to-NLO (NNLO) plus next-to-next-to-leading-logarithmic (NNLL) prediction in the
\textsc{HRes2.3} generator~\cite{deFlorian:2012mx,Grazzini:2013mca}. 
The various production cross sections and branching fractions for the SM Higgs 
boson production, and their corresponding uncertainties are taken from 
Refs.~\cite{deFlorian:2016spz,Denner:2011mq,Ball:2011mu} and references therein.

The qq$\to \PZ\PZ$, $\PW\PZ$, and $\ttbar$ processes are generated with 
$\POWHEG$, as are the $\PW\PH \to \PW\PW\PW$, $\PZ\PH \to \PZ\PW\PW$, and $\PH \to ZZ$ backgrounds. 
The $\Pg\Pg \to \PZ\PZ$ process is generated at leading order (LO) with 
\textsc{MCFM}~\cite{Campbell:2010ff}. The \MGAMCNLO generator is used for 
triboson, $\ttbar \PW$, and $\ttbar \PZ$  productions simulated either at 
NLO with the FxFx jet matching and merging~\cite{Frederix:2012ps} 
or at LO with the MLM jet matching and merging~\cite{Alwall:2007fs}.  
The generators are interfaced with \PYTHIA 8.212 ~\cite{Sjostrand:2014zea} 
to model the parton showering and fragmentation, as well as the decay of the $\Pgt$ leptons.
The \PYTHIA parameters affecting the description of the underlying event are 
set to the {CUETP8M1} tune~\cite{Khachatryan:2015pea}.

Generated events are processed through a simulation of the CMS detector based on
\textsc{Geant4}~\cite{Agostinelli:2002hh}, and are reconstructed with the same algorithms 
which are used for data.
The simulated samples include additional $\Pp\Pp$ interactions per bunch
crossing, referred to as pileup.
The effect of pileup is taken into account by generating concurrent minimum 
bias collision events. The simulated events are weighted 
such that the distribution of the number of additional pileup interactions, 
estimated from the measured instantaneous luminosity for each bunch crossing, 
matches that in data, with an average of approximately 27 interactions per bunch crossing.



\section{Event reconstruction}
\label{sec:reconstruction}

The reconstruction of observed and simulated events relies on the 
particle-flow (PF) algorithm~\cite{Sirunyan:2017ulk}. PF combines 
information from all CMS subdetectors to identify
and reconstruct the particles emerging from $\Pp\Pp$ collisions:
charged hadrons, neutral hadrons, photons, muons, and electrons.
Combinations of these PF objects are used to reconstruct
higher-level objects such as jets, $\tauh$ candidates, or
the missing transverse momentum.
The missing transverse momentum vector \etvecmiss is defined as 
the projection onto the plane perpendicular to the beam axis of the 
negative vectorial sum of the momenta of all reconstructed particle-flow 
objects in an event. Its magnitude is referred to as \ptmiss.
The reconstructed vertex with the largest value of summed physics-object 
$\pt^2$ is taken to be the primary $\Pp\Pp$ interaction vertex. 
Physics-objects are constructed by a jet finding 
algorithm~\cite{Cacciari:2008gp,Cacciari:2011ma} applied to all charged 
tracks.

Muons are identified with requirements on the quality of
the track reconstruction and on the number of measurements in the
inner tracker and the muon systems~\cite{Chatrchyan:2012xi}.
Electrons are identified with a multivariate discriminant
combining several quantities describing the track quality,
the shape of the energy deposits in the ECAL,
and the compatibility of the measurements from the tracker and the
ECAL~\cite{Khachatryan:2015hwa}.
To reject non-prompt or misidentified leptons, a relative lepton isolation is defined as:

\begin{equation}
I^{\ell} \equiv \frac{\sum_\text{charged}  \PT + \max\left( 0, \sum_\text{neutral}  \PT
                 - \frac{1}{2} \sum_\text{charged, PU} \PT  \right )}{\PT^{\ell}}.
\label{eq:reconstruction_isolation}
\end{equation}

In this expression, $\sum_\text{charged}  \PT$ is the scalar sum of the
transverse momenta of the charged particles originating from
the primary vertex and located in a cone of size
$\Delta R = \sqrt{\smash[b]{(\Delta \eta)^2 + (\Delta \phi)^2}} = 0.4$\,(0.3)
centered on the muon (electron) direction. The sum
$\sum_\text{neutral}  \PT$  represents
a similar quantity for neutral particles.
The contribution of photons and neutral hadrons originating from pileup 
vertices is estimated from the scalar sum of the transverse
momenta of charged hadrons in the cone originating from pileup vertices,
$\sum_\text{charged, PU} \PT$. This sum is multiplied by a factor of
$1/2$, which corresponds approximately to the ratio of neutral to charged
hadron production in the hadronization process
of inelastic $\Pp\Pp$ collisions, as estimated from simulation.
The expression $\PT^{\ell}$ stands for the $\pt$ of the lepton. Isolation 
requirements   used in this analysis, based on $I^{\ell}$, are listed 
in Table~\ref{tab:inclusive_selection}.

Jets are reconstructed with an anti-\kt clustering algorithm implemented
in the \FASTJET library~\cite{Cacciari:2011ma, Cacciari:fastjet2}.
It is based on the clustering of neutral and charged PF candidates within 
a distance parameter of 0.4. Charged PF candidates not associated with 
the primary vertex of the interaction are not considered when building jets. 
The combined secondary vertex (CSV) algorithm is used to identify jets 
that are likely to originate from a b quark 
(``b jets'')~\cite{1748-0221-8-04-P04013}. The algorithm 
exploits the track-based lifetime information together with the secondary 
vertices associated with the jet to provide a likelihood ratio discriminator 
for the b jet identification. A set of $\pt$-dependent correction
factors are applied to simulated events to account for differences in the 
b tagging efficiency between data and simulation. The working point chosen 
in this analysis gives an efficiency for real b jets of about 70\% for 
about 1\% of light flavor or quark jets being misidentified.

Hadronically decaying $\Pgt$ leptons are reconstructed with the hadron-plus-strips (HPS)
algorithm~\cite{Khachatryan:2015dfa, CMS-PAS-TAU-16-002}, which is
seeded with anti-\kt jets. The HPS algorithm reconstructs $\tauh$ 
candidates on the basis of the number of tracks and on the number of ECAL 
strips in the $\eta$-$\phi$ plane with energy deposits, in the 1-prong,
1-prong + $\PGpz$, and 3-prong decay modes. A
multivariate (MVA) discriminator~\cite{Hocker:2007ht}, including isolation
and lifetime information, is used to reduce the rate for  quark- and gluon-initiated jets
to be identified as $\tauh$ candidates. The three working points used in this analysis,
Very Tight, Tight, and Medium MVA ID,
have an efficiency of about 55\%, 60\%, 65\% for genuine $\tauh$,
with about 1\%, 1.5\%, 2.5\% misidentification rate for quark- and gluon-initiated jets, 
within a $\pt$ range typical of $\tauh$ originating from a $\PZ$ boson.
Electrons and muons misidentified as $\tauh$ candidates are suppressed using 
dedicated criteria based on the consistency between the measurements in the 
tracker, the calorimeters, and the muon detectors~\cite{Khachatryan:2015dfa, CMS-PAS-TAU-16-002}.
The working points of these discriminators are
decay channel specific. The $\tauh$ energy scale in simulation is corrected 
per decay mode, on the basis of a measurement in $\PZ\to\Pgt\Pgt$ events. 
The rate and the energy scale of electrons and muons misidentified as $\tauh$ 
candidates are also corrected in simulation, on the basis of a tag-and-probe 
measurement~\cite{CMS:2011aa} in $\PZ\to\ell\ell$ events.

The visible mass of the $\Pgt\Pgt$ system, $\mvis$, can be used to separate
the $\PH\to \Pgt \Pgt$ signal events
from the large contribution of irreducible $\PZ \to \Pgt \Pgt$ events.
However, the neutrinos from the $\Pgt$ lepton decays carry a large fraction of
the $\Pgt$ lepton energy and reduce the discriminating power of this variable.
The \textsc{svfit} algorithm~\cite{Bianchini:2014vza} combines the \etvecmiss 
with the four-vectors of both $\Pgt$ candidates to calculate a more accurate 
estimate of the mass of the parent boson, denoted as $\mtt$. The resolution 
of $\mtt$ is between 15 and 20\% depending on the $\Pgt\Pgt$ final state. The 
$\mtt$ variable is used in the $\PZ\PH$ channel, while $\mvis$ is used in the 
$\PW\PH$ channel because the \textsc{svfit} algorithm cannot account for the 
additional \etvecmiss from the $\PW$ boson decay. 



\section{Event selection}
\label{sec:selection}

Events in the $\PW\PH$ and $\PZ\PH$ production channels are selected using single or 
double lepton triggers, since the $\PW$ and $\PZ$ bosons ensure the presence of one or two well-isolated leptons with sufficiently 
high $\pt$. The list of the trigger and offline selection requirements for 
all possible decay modes is presented in Table~\ref{tab:inclusive_selection}.
All reconstructed objects in the events are required to be separated from each 
other by $\Delta R > 0.3$. In the case of $\tauh$, they are required to be 
separated from all other objects by $\Delta R > 0.5$. The resulting event samples are made mutually 
exclusive by discarding events that have additional loosely identified 
and isolated muons or electrons.

\begin{table*}[htbp]
\centering
\begin{small}
\begin{tabular}{llll}
%  Channel           &         Trigger ($\pt/\eta$)         &    \multicolumn{2}{c|}{Lepton selection}                 \\ \cline{3-4}
% & & $\pt/\eta$   & Isolation  \\
     \multicolumn{4}{c}{$\PW\PH$ selection requirements}                 \\ 
     \multicolumn{4}{c}{$\pt^{\tauh}>20$, $|\eta^{\tauh}|<2.3$, $I^\Pe<0.1$, $I^\Pgm<0.15$, b-veto }                 \\ 
\hline
  Channel           &         Trigger ($\pt/\eta$)         & Lepton Selection: $\pt$   & Lepton Selection: Iso.  \\
\hline
 $\Pe\Pgm\tauh$      &  $\Pgm (22/2.1)$ or $\Pe (25/2.1)$  &     $\pt^\Pe>15, \pt^\Pgm>23$ or $\pt^\Pe>26, \pt^\Pgm>15$ &  MVA $\tauh$ ID Tight  \\
 $\Pgm\Pgm\tauh$     &  $\Pgm (22/2.1)$                    &     $\pt^\Pgm>23,\pt^\Pgm>15$                              &  MVA $\tauh$  Tight  \\
 $\Pe\tauh\tauh$     &  $\Pe (25/2.1)$                     &     $\pt^\Pe>26$                                           &   MVA $\tauh$ Medium/VTight  \\
 $\Pgm\tauh\tauh$    &  $\Pgm (22/2.1)$                    &     $\pt^\Pgm>23$                                          &  MVA $\tauh$ Medium/VTight  \\
\hline \\

\\
     \multicolumn{4}{c}{$\PZ\PH$ selection requirements}                 \\ 
     \multicolumn{4}{c}{$\PZ$ boson reconstructed from opposite charge, same-flavor light leptons, $60\GeV < \textrm{m}_{\ell\ell} < 120\GeV$}  \\ 
     \multicolumn{4}{c}{$\tauh$ baseline requirements: $\pt^{\tauh}>20$, $|\eta|<2.3$, MVA $\tauh$ Medium}   \\ 
     \multicolumn{4}{c}{$\Pe$ baseline requirements: $\pt^\Pe>10$, $|\eta|<2.5$, MVA Loose ID }   \\ 
     \multicolumn{4}{c}{$\Pgm$ baseline requirements: $\pt^\Pgm>10$, $|\eta|<2.4$, ID Loose, $I^\Pgm<0.25$ }   \\ 
\hline
  Channel           &         Trigger ($\pt/\eta$)         & Lepton Selection: $\pt$   & Lepton Selection: Isolation  \\
\hline
  $\Pe\Pe\Pgm\tauh$     &                                    &                                    &  $I^\Pgm<0.15$       \\
  $\Pe\Pe\Pe\tauh$      & $\Pe(23/2.5)\,\&\,\Pe(12/2.5)$,    &  $\pt^\Pe>24\,\&\,\pt^\Pe>13$, &  $\Pe$ Tight ID, $I^\Pe<0.15$ \\
  $\Pe\Pe\tauh\tauh$    & or $\Pe(27/2.5)$                   &  or $\pt^\Pe>28$                   &  baseline selection       \\
  $\Pe\Pe\Pe\Pgm$       &                                    &                                    &  $\Pe$ Tight ID, $I^\Pe<0.15$, $I^\Pgm<0.15$ \\
\hline
  $\Pgm\Pgm\Pgm\tauh$   &                                    &                                    &  $I^\Pgm<0.15$       \\
  $\Pgm\Pgm\Pe\tauh$    &  $\Pgm(17/2.4)\,\&\,\Pgm(8/2.4)$,  &  $\pt^\Pgm>18\,\&\,\pt^\Pgm>10$,   &  $\Pe$ Tight ID, $I^\Pe<0.15$ \\
  $\Pgm\Pgm\tauh\tauh$  &   or $\Pgm(24/2.4)$                &  or $\pt^\Pgm>25$                  &  baseline selection       \\
  $\Pgm\Pgm\Pe\Pgm$     &                                    &                                    &  $\Pe$ Tight ID, $I^\Pe<0.15$, $I^\Pgm<0.15$ \\
\hline
\end{tabular}
\end{small}
\caption{Kinematic selection requirements for $\PW\PH$ and $\PZ\PH$ events.
The trigger requirement is defined by a combination of trigger candidates with 
\pt over a given threshold (in \GeV), indicated inside parentheses. The 
pseudorapidity thresholds come from trigger and object reconstruction constraints.
\label{tab:inclusive_selection}
}
\end{table*}

In the $\Pe\Pgm\tauh$ and $\Pgm\Pgm\tauh$ final states of the $\PW\PH$ channel, 
the two light leptons are required to have the same charge to reduce the $\ttbar$ 
and $\PZ+\textrm{jets}$ backgrounds where one or more jets is misidentified as a $\tauh$ 
candidate. The $\tauh$ candidate has opposite charge to the light leptons. The highest $\pt$
light lepton is considered as coming from the $\PW$ boson, while the Higgs boson 
candidate is formed from the $\tauh$ object and the subleading light lepton. The 
correct pairing is achieved in about 75\% of events. The leading light lepton is required 
to fire the single lepton triggers and to have a $\pt$ that is 1\GeV above the online 
thresholds, whereas the subleading light lepton has $\pt>15\GeV$ resulting from
optimization. Selection criteria based on three variables have been found to 
improve the results in both channels:
\begin{itemize}
\item $L_T>100\GeV$, where $L_T$ is the scalar $\pt$ sum of the three objects in the final state;
\item $\abs{\Delta\phi(\ell_1,\PH)}>2.0$, where $\ell_1$ is the leading light lepton, and 
$\PH$ is the system formed by the subleading light lepton and the $\tauh$ candidate;
\item $\abs{\Delta\eta(\ell_1,\PH)}<2.0$.
\end{itemize}


In the $\Pe\tauh\tauh$ and $\Pgm\tauh\tauh$ final states of the $\PW\PH$ channel, 
the $\tauh$ candidates are required to have opposite charge. As the result 
of an optimization, the $\tauh$ that has the same charge as the light lepton and must 
have $\pt > 35\GeV$, and for the subleading one, $\pt > 20\GeV$. This is driven 
by the fact that the $\tauh$ that has the same charge as the light lepton is almost 
always a jet misidentified as a $\tauh$ candidate, and the jet misidentification 
rate strongly decreases with $\pt$. Selection criteria based on three variables 
have been found to improve the results in both channels:
\begin{itemize}
\item $L_T>130\GeV$, where $L_T$ is the scalar $\pt$ sum of the three objects in the final state;
\item $\abs{\vec{S_T}}<70\GeV$, where $\vec{S_T}$ is the vectorial $\pt$ sum of the three objects in the final state and of $\etvecmiss$;
\item $\abs{\Delta\eta(\tauh,\tauh)}<2.0$.
\end{itemize}



In the $\PZ\PH$ final states, the $\PZ$ boson is reconstructed from the opposite charge, same-flavor
light lepton combination which has a mass closest to the $\PZ$ boson mass. Different 
electron and muon identification and isolation criteria are used for the leptons 
assigned to the $\PZ$ boson compare to those assigned to the Higgs boson. A looser
selection is applied for the $\PZ$ boson associated leptons to increase signal acceptance
while a tighter selection is applied to those assigned to the Higgs boson to
decrease the background contributions from $\PZ+\textrm{jets}$ and other reducible
backgrounds. The specific selections detailed in Table~\ref{tab:inclusive_selection},
including those chosen for the $\tauh$ candidates, where chosen based on 
optimizing for best signal sensitivity.

To further suppress the contributions from the $\PZ+\textrm{jets}$ background, the signal 
region is split into a High-$L_{T}^{\textrm{Higgs}}$ and Low-$L_{T}^{\textrm{Higgs}}$
region where $L_{T}^{\textrm{Higgs}}$ is defined as the scalar $\pt$ sum of the decay 
products of the $\PH$ boson. Because of the identical event kinematics, the 
$L_{T}^{\textrm{Higgs}}$ regions are defined based on the $\htt$ final 
states of an event. The split between the High- and Low-$L_{T}^{\textrm{Higgs}}$
regions are:
\begin{itemize}
\item $\ell\ell\Pe\tauh$: $L_{T}^{\textrm{Higgs}} = 60\GeV$
\item $\ell\ell\Pgm\tauh$: $L_{T}^{\textrm{Higgs}} = 60\GeV$
\item $\ell\ell\tauh\tauh$: $L_{T}^{\textrm{Higgs}} = 75\GeV$
\item $\ell\ell\Pe\Pgm$: $L_{T}^{\textrm{Higgs}} = 50\GeV$
\end{itemize}
For convenience, the High- and Low-$L_{T}^{\textrm{Higgs}}$ regions are plotted
side-by-side.



\section{Background estimation}
\label{sec:background_estimation}

The irreducible backgrounds ($\PZ\PZ$, $\ttbar\PZ$, $\PW\PW\PZ$, $\PW\PZ\PZ$, 
$\PZ\PZ\PZ$, as well as $\PW\PZ$ and $\ttbar \PW$ in the $\PW\PH$ channels) are 
estimated from simulations and scaled to their theoretical cross section. Higgs 
boson decays to pairs of $\PW$ or $\PZ$ bosons 
are also estimated from simulations and considered as background processes. 
Additionally, the $\ttbar\PH$ production process with all Higgs boson decay
paths is estimated from simulations and considered as background processes.

The reducible backgrounds, which have at least one jet misidentified as an electron, 
muon, or $\tauh$ candidate, are estimated from data. 
Data events meeting specific requirements detailed below are reweighted 
as a function of a misidentification rate to estimate the 
contribution of processes with jets misidentified as leptons in the signal region. 

In the $\PW\PH$ analysis, the misidentification rate of jets as electrons, muons, 
or $\tauh$ candidates is measured in $\PZ+\textrm{jets}$ events. The $\PZ$ boson is reconstructed 
in its dielectron decay mode for measuring the jet to muon  misidentification
rate, and is reconstructed in its dimuon decay mode for measuring the jet to electron
or $\tauh$ misidentification rate.
The rates are measured in bins of the lepton $\pt$, and are 
split between reconstructed decay mode for the $\tauh$ candidates. 

In the $\Pe\Pgm\tauh$ and $\Pgm\Pgm\tauh$ final states, 
events are selected for reweighting if they pass the full signal region 
selection except that the subleading light lepton or the $\tauh$ do not 
pass the isolation or identification conditions.
To remove double-counting of events, events in simulation that have a jet that is 
misidentified as the $\tauh$ or as the subleading lepton, are discarded. Simulated 
events that have a jet misidentified as the leading lepton, but two real leptons 
for the subleading lepton and the $\tauh$, are estimated from simulations as their 
contribution is not taken into account with the misidentification rate method 
described above. Such events mostly arise from $\ttbar$ and $\PZ+\textrm{jets}$ processes, 
and account for a small fraction of the total expected background in the signal region. 
In the $\Pe\tauh\tauh$ and $\Pgm\tauh\tauh$ final states of the $\PW\PH$ channels, 
the method is essentially the same, except that the object susceptible to being misidentified 
is the $\tauh$ candidate that has the same charge as the light lepton.  

In the $\PZ\PH$ analysis, a very similar misidentification rate is used
to estimate the contribution of jets misidentified as electrons, muons, or $\tauh$
candidates to the signal region events. The misidentification rate is measured in four object final states which
is dominated by $\PZ+\textrm{jets}$ events with a small contribution from 
$\ttbar$ events. Identical to the $\PW\PH$ final states, the rates are measured 
in bins of the lepton $\pt$, and are split between reconstructed decay modes for 
the $\tauh$ candidates. 
In these four lepton final states, the objects most susceptible 
to being misidentified jets are the $\PH$ boson associated objects.
In the $\PZ\PH$ analysis, data events which pass the full
signal region selection, except either or both of the $\PH$ boson associated objects
fail identification or isolation criteria, are weighted by the misidentification
rate for the failing object and assigned to the ``jet fakes'' background.
To avoid double counting, events with both $\PH$ boson associated objects failing have their weight
subtracted from the events which only have a single object failing. 
This misidentification rate method is used to estimate the yield of the reducible
backgrounds. The shape of the reducible background contribution is taken from
data in the signal free region with same charge objects assigned to the $\PH$ boson.
A high statistics, relaxed identification and isolation selection is used to reduce
statistical uncertainties. Kolmogorov, Smirnov test have been performed to validate
shape compatibility between this relaxed same charge selection and the signal region
reducible background backgrounds.



\section{Systematic uncertainties}
\label{sec:systematics}

The overall uncertainty in the $\tauh$ identification efficiency for genuine $\tauh$ 
leptons is 5\%, which has been measured with a tag-and-probe method in $\PZ\to\Pgt\Pgt$ events.
This number is not fully correlated among the various channels because the $\tauh$ 
candidates are required to pass different working points of the discriminators that 
reduce the misidentification rate of electrons and muons as $\tauh$ candidates.

An uncertainty of 1.2\% in the visible energy scale of genuine $\tauh$ leptons 
affects both the distributions and yields of the signals and backgrounds. It is 
uncorrelated among the 1-prong, 1-prong + $\PGpz$, and 3-prong decay modes.

The uncertainties in the muon and electron identification, isolation, and trigger 
efficiencies lead to a rate uncertainty of 2\% for both muons and electrons.
The uncertainty in the electron energy scale, which amounts to 2.5\% in the endcaps 
and 1\% in the barrel of the detector, affects the final distributions and rate.
In all channels, the effect of the uncertainty in
the muon energy scale is negligible.

The rate uncertainty related to discarding events with a b-tagged jet is
4.5\% for processes with heavy-flavor jets, and 0.15\% for processes with light-flavor jets.

Uncertainties from the renormalization and the factorization scales, and from the 
choice of the PDF set, are taken into account for the $\PZ\PZ$ and $\PW\PZ$ 
background processes. The uncertainty from the renormalization and factorization 
scales is determined by varying these scales between 0.5 and 2 times their nominal 
value while keeping their ratio between 0.5 and 2. It leads to yield uncertainties 
of $^{+3.2\%}_{-4.2\%}$ for the $\Pq\Pq\rightarrow \PZ\PZ$ background, and $\pm 3.2\%$ 
for the $\PW\PZ$ process. The uncertainty from the PDF set is determined following 
the PDF4LHC recommendations, and leads to yield uncertainties of $+3.1/-4.2\%$ for 
the $\Pq\Pq\rightarrow \PZ\PZ$ background, and $\pm 4.5\%$ for the $\PW\PZ$ process. 
In addition, a 10\% uncertainty in the k-factor used for the $\Pg\Pg \rightarrow 
\PZ\PZ$  prediction is considered.  The uncertainty in the cross section of the 
rare $\ttbar\PW$ and $\ttbar\PZ$ processes amounts to 25\%.

The reducible backgrounds are estimated using the measured rates for jets to be 
misidentified as electrons, muons, or $\tauh$. The misidentification rates are 
measured in different bins of lepton $\pt$, and are further split between 
reconstructed decay modes for the $\tauh$. In the $\PW\PH$ final states where
the shape of the reducible background is take from the misidentification rate
method, the statistical uncertainty in every 
bin is considered as an independent uncertainty, which is propagated to the mass 
distributions and to the yields of the reducible background estimate. Rate
uncertainties applied in the $\PZ\PH$ final states cover this possible fluctuation
in rate from these uncertainties and, with the shape taken from the same charge
region, the shape is already validated a compatible.
In both the $\PW\PH$ and $\PZ\PH$ final states, an additional
uncertainty on the misidentification rates arising from the subtraction of 
prompt leptons estimated from simulations is taken into account and propagated to 
the reducible background mass distributions. 

In the $\PW\PH$ channels, an 
additional uncertainty comes from potentially different misidentification rates 
in $\PZ$ + jets events, where the rates are measured, and in $\PW$ + jets or 
$\ttbar$ events, which constitute a large fraction of the reducible background 
in the signal region. This leads to a 10\% yield uncertainty for the reducible 
background in each final state of the $\PW\PH$ analysis. In the $\PZ\PH$ final
states a similar uncertainty is applied based on potential differences in the
measurement region versus the application region. These uncertainties
range from 26\% in the $\ell\ell\Pgm\tauh$ final states to 100\% in the
$\ell\ell\Pe\Pgm$ final states. The large uncertainty in the $\ell\ell\Pe\Pgm$ 
final states results from the very low expected reducible background yields 
which makes any comparison of the method susceptible to large statistical fluctuations.

The \etvecmiss scale uncertainties~\cite{CMS-JME-12-002}, which are computed 
event-by-event, affect the normalization of various processes through the event 
selection, as well as their distributions through the propagation of these 
uncertainties to the di-$\Pgt$ mass $\mtt$ in the $\PZ\PH$ channels. The 
\etvecmiss scale uncertainties arising from unclustered energy deposits in the 
detector come from four independent sources related to the tracker, ECAL, HCAL, 
and forward calorimeters subdetectors. Additionally, \etvecmiss scale 
uncertainties related to the uncertainties in the jet energy scale measurement, 
which lead to uncertainties in the \etvecmiss calculation, are taken into 
account. 

The rate and acceptance uncertainties for the signal processes related to the 
theoretical calculations are due to uncertainties in the PDFs, variations of 
the QCD renormalization and factorization scales, and uncertainties in the 
modeling of parton showers. The magnitude of the rate uncertainty depends on 
the production process and on the event category.

The inclusive uncertainty related to the PDFs amounts to 1.9 and 1.6\%, 
respectively, for the $\PW\PH$, and $\PZ\PH$ production modes~\cite{deFlorian:2016spz}. The
corresponding uncertainty for the variation of the renormalization and 
factorization scales is 0.7 and 3.8\%, respectively~\cite{deFlorian:2016spz}.

The uncertainty in the integrated luminosity amounts to 2.5\%~\cite{CMS-PAS-LUM-17-001}.

Uncertainties related to the finite number of simulated events, or to the 
limited number of events in data control regions, are taken into account. They 
are considered for all bins of the distributions used to extract the results.
They are uncorrelated across different samples, and across bins of a single distribution. 

The systematic uncertainties considered in the analysis are summarized in Table~\ref{tab:uncertainties}.

\begin{table*}[!ht]
\centering
\newcolumntype{x}{D{,}{\text{--}}{2.2}}
\begin{tabular}{ll}
Source of uncertainty & Magnitude \\
\hline
 $\tauh$ energy scale                & 1.2\% in energy scale\\
 $\Pe$ energy scale               & 1--2.5\%  in energy scale \\
 $\etvecmiss$ energy scale              & Dependent upon $\pt$ and $\eta$ \\
 $\tauh$ ID \& isolation & 5\% per $\tauh$  \\
 $\Pe$ ID \& isolation \& trigger  &   2\%  \\
 $\Pgm$ ID \& isolation \& trigger & 2\%  \\
 Diboson normalization & 5\% \\
 Integrated luminosity     & 2.5\%  \\
 b-tagged jet rejection & 4.5\% heavy flavor, 0.15\% light flavor \\
 Limited number of events                & Statistical uncertainty in individual bins  \\
 Signal theoretical uncertainty  & Up to 20\% \\
 Reducible background uncertainties & $\PW\PH$: shape and yield based \\
                                    & $\PW\PH$: 10\% yield \\
                                    & $\PZ\PH$: 26--100\% yield \\
\hline
\end{tabular}
\caption{Sources of systematic uncertainty.}
\label{tab:uncertainties}
\end{table*}



\section{Results}
\label{sec:results}


In the $\PZ\PH$ final states, the $\mtt$ distribution is
used for signal extraction. The Low-$L_{T}^{\textrm{Higgs}}$ and
High-$L_{T}^{\textrm{Higgs}}$ regions are plotted side-by-side
in the following distributions. Figs.~\ref{fig:zh_results_svFitLLXX} 
and \ref{fig:zh_results_svFitAll} show the
$\mtt$ distributions for each of the $\htt$ final states and
the combined distribution for all eight $\PZ\PH$ channels.
The eight $\PZ\PH$ final states are each fit separately in the global
fit; combining them together helps reduce statistical
fluctuations for visualization purposes only.
The distributions are post-fit and show full uncertainties.
The $\PW\PH$ and $\PZ\PH$ signals are shown as 5x larger than their best-fit
signal strength value of $2.5 \times$ SM.


\begin{figure}[h!]
 \begin{center}
  \includegraphics[width=0.45\textwidth]{higgs_to_taus_vh/plots/zh/llet_postfit.pdf}
  \includegraphics[width=0.45\textwidth]{higgs_to_taus_vh/plots/zh/llmt_postfit.pdf}
  \includegraphics[width=0.45\textwidth]{higgs_to_taus_vh/plots/zh/lltt_postfit.pdf}
  \includegraphics[width=0.45\textwidth]{higgs_to_taus_vh/plots/zh/llem_postfit.pdf}
 \end{center}
 \caption{The postfit $\mtt$ distributions used to extract the signal shown
  for (top left) $\ell\ell\Pe\tauh$, (top right) $\ell\ell\Pgm\tauh$, 
  (bottom left) $\ell\ell\tauh\tauh$, and (bottom right) $\ell\ell\Pe\Pgm$.
  The left half of each distribution is the Low-$L_{T}^{\textrm{Higgs}}$ region
  while the right half of each distribution is the High--$L_{T}^{\textrm{Higgs}}$ region.
  $\ell\ell$ covers both $\PZ \to \Pgm\Pgm$ and $\PZ \to \Pe\Pe$ events.
  The distributions show full uncertainties.
  The $\PW\PH$ and $\PZ\PH$ signals are shown as 5x larger than their best-fit
  signal strength value of $2.5 \times$ SM.
 }
 \label{fig:zh_results_svFitLLXX}
\end{figure}


\begin{figure}[h!]
 \begin{center}
  \includegraphics[width=0.65\textwidth]{higgs_to_taus_vh/plots/zh/zh_postfit.pdf}
 \end{center}
 \caption{The postfit $\mtt$ distributions used to extract the signal shown
  for all 8 $\PZ\PH$ channels combined.
  The distribution shows full uncertainties.
  The left half of the distribution is the Low-$L_{T}^{\textrm{Higgs}}$ region
  while the right half corresponds to the High--$L_{T}^{\textrm{Higgs}}$ region.
  The $\PW\PH$ and $\PZ\PH$ signals are shown as 5x larger than their best-fit
  signal strength value of $2.5 \times$ SM.
 }
 \label{fig:zh_results_svFitAll}
\end{figure}

The results in the $\PW\PH$ channels are obtained from the distributions of the 
visible mass of the $\tauh$ candidates in the $\ell\tauh\tauh$ channels, 
and of the visible mass of the $\tauh$ and subleading light lepton in the 
$\ell\ell\tauh$ final states. The mass distributions
are shown in Figs.~\ref{fig:mass_llt} and ~\ref{fig:mass_ltt} for the semileptonic 
and hadronic channels, respectively. Fig.~\ref{fig:mass_wh} shows all
four $\PW\PH$ final states combined together.

\begin{figure}[h!]
 \begin{center}
  \includegraphics[width=0.45\textwidth]{higgs_to_taus_vh/plots/wh/emt_postfit.pdf}
  \includegraphics[width=0.45\textwidth]{higgs_to_taus_vh/plots/wh/mmt_postfit.pdf}
 \end{center}
 \caption{Postfit mass distributions in the $\Pe\Pgm\tauh$ (left) and 
 $\Pgm\Pgm\tauh$ (right) final states.
 The distributions show full uncertainties.
 The $\PW\PH$ and $\PZ\PH$ signals are shown as 5x larger than their best-fit
 signal strength value of $2.5 \times$ SM.
 }
 \label{fig:mass_llt}
\end{figure}

\begin{figure}[h!]
 \begin{center}
  \includegraphics[width=0.45\textwidth]{higgs_to_taus_vh/plots/wh/ett_postfit.pdf}
  \includegraphics[width=0.45\textwidth]{higgs_to_taus_vh/plots/wh/mtt_postfit.pdf}
 \end{center}
 \caption{Postfit mass distributions in the $\Pe\tauh\tauh$ (left) 
 and $\Pgm\tauh\tauh$ (right) final states.
 The distributions show full uncertainties.
 The $\PW\PH$ and $\PZ\PH$ signals are shown as 5x larger than their best-fit
 signal strength value of $2.5 \times$ SM.
 }
 \label{fig:mass_ltt}
\end{figure}

\begin{figure}[h!]
 \begin{center}
  \includegraphics[width=0.65\textwidth]{higgs_to_taus_vh/plots/wh/wh_postfit.pdf}
 \end{center}
 \caption{Postfit mass distributions of the four $\PW\PH$ final states
 combined together. 
 The distributions show full uncertainties.
 The $\PW\PH$ and $\PZ\PH$ signals are shown as 5x larger than their best-fit
 signal strength value of $2.5 \times$ SM.
 }
 \label{fig:mass_wh}
\end{figure}


\begin{table*}
\centering
\begin{small}
\newcolumntype{x}{D{,}{\,\pm\,}{5.5}}
\begin{tabular}{lxxxx}
Process & \multicolumn{1}{c}{$\PW\PH, \Pe\Pgm\tauh$ } & \multicolumn{1}{c}{$\PW\PH, \Pgm\Pgm\tauh$ } & \multicolumn{1}{c}{$\PW\PH, \Pe\tauh\tauh$} & \multicolumn{1}{c}{$\PW\PH, \Pgm\tauh\tauh$}  \\
\hline
$\PZ\PZ$                  & 1.56, 0.05    & 0.93, 0.03  & 0.82, 0.04  & 1.18, 0.05   \\
$\PW\PZ$                  & 7.92, 0.28    & 6.69, 0.24  & 4.83, 0.25  & 8.38, 0.42   \\
Jet Fakes                 & 10.09, 1.61   & 12.19, 1.72 & 10.68, 1.27 & 19.80, 1.87  \\
Rare                      & 2.28, 0.61    & 3.77, 0.84  & 1.71, 1.08  & 1.76, 0.90   \\
Total backgrounds         & 21.85, 1.75   & 23.58, 1.92 & 18.04, 1.67 & 31.12, 2.12  \\
\hline
$\PW\PH, \PH \to\Pgt\Pgt$ & 4.28, 0.72    & 4.25, 0.73  & 3.51, 0.62  &  5.45, 0.97  \\
$\PZ\PH, \PH \to\Pgt\Pgt$ & 0.42, 0.07    & 0.40, 0.08  & 0.33, 0.07  &  0.44, 0.10  \\
Total signal              & 4.70, 0.72    & 4.65, 0.73  & 3.84, 0.92  &  5.98, 0.98  \\
\hline
Observed &  \multicolumn{1}{c}{28 $\pm$ 5.3} &  \multicolumn{1}{c}{29 $\pm$ 5.4} &  \multicolumn{1}{c}{23 $\pm$ 4.8} &  \multicolumn{1}{c}{38 $\pm$ 6.2}  \\
\hline
\end{tabular}
\end{small}
\caption{Background and signal expectations for the $\PW\PH$ channels, 
together with the number of observed 
events, for the post-fit signal region distributions.
$S$ and $B$ are, respectively, the number of expected signal events for a Higgs boson 
with a mass $\mH = 125.09\GeV$ and of expected background events, in those bins. 
The background uncertainty accounts for all sources of background uncertainty, 
systematic as well as statistical, after the global fit. The contribution from 
``Rare'' includes events from triboson, $\ttbar + \PW$/$\PZ$, $\ttbar\PH$ production,
and other rare processes.
}
\label{tab:sb_wh}
\end{table*}

\begin{table*}
\centering
\begin{small}
\newcolumntype{x}{D{,}{\,\pm\,}{5.5}}
\begin{tabular}{lxxxx}
Process & \multicolumn{1}{c}{$\ell\ell\Pe\tauh$} &  \multicolumn{1}{c}{$\ell\ell\Pgm\tauh$} &  \multicolumn{1}{c}{$\ell\ell\tauh\tauh$} &  \multicolumn{1}{c}{$\ell\ell\Pe\Pgm$} \\
\hline
$\PZ\PZ$                        & 14.40, 0.36 & 26.91, 0.55 & 25.58, 1.05 & 9.33, 0.18 \\   
Rare                            & 0.62, 0.08  & 1.54, 0.61  & 0.81, 0.42  & 3.02, 0.23 \\
Jet Fakes                       & 14.01, 1.55 & 17.58, 1.17 & 58.05, 2.87 & 3.66, 4.60 \\
Total backgrounds               & 29.03, 1.59 & 46.03, 1.43 & 84.44, 3.08 & 16.01, 4.61\\             
\hline
$\PW\PH, \PH \to\Pgt\Pgt$       & 0.008, 0.002  & 0.01, 0.003  & 0.016, 0.005  & 0.002, 0.001 \\
$\PZ\PH, \PH \to\Pgt\Pgt$       & 2.83, 0.39  & 5.31, 1.30  & 5.29, 1.17  & 1.62, 0.20 \\
Total signal                    & 2.84, 0.39  & 5.32, 0.70  & 5.31, 1.17  & 1.62, 0.20 \\
\hline
Observed &  \multicolumn{1}{c}{33 $\pm$ 5.75} &  \multicolumn{1}{c}{53 $\pm$ 7.28} &  \multicolumn{1}{c}{87 $\pm$ 9.33} &  \multicolumn{1}{c}{20 $\pm$ 4.47}  \\
\hline
\end{tabular}
\end{small}
\caption{Background and signal expectations for the $\PZ\PH$ channels, 
together with the number of observed 
events, for the post-fit signal region distributions. The $\PZ\PH$ final states
are each grouped according to the Higgs boson decay products. 
$\ell\ell$ covers both $\PZ \to \Pgm\Pgm$ and $\PZ \to \Pe\Pe$ events.
$S$ and $B$ are, respectively, the number of expected signal events for a Higgs boson 
with a mass $\mH = 125.09$\GeV and of expected background events, in those bins. 
The background uncertainty accounts for all sources of background uncertainty, 
systematic as well as statistical, after the global fit. The contribution from 
``Rare'' includes events from triboson, $\ttbar + \PW$/$\PZ$, $\ttbar\PH$ production,
and other rare processes.
}
\label{tab:sb_zh}
\end{table*}


Grouping events in the signal regions by their decimal logarithm of the ratio of the 
signal ($S$) to signal-plus-background ($S+B$) in each bin, Fig.~\ref{fig:sb}, 
an excess of observed events with respect to the SM background expectation is 
visible in the most sensitive bins of the analysis.

\begin{figure}[!ht]
 \begin{center}
  \includegraphics[width=0.45\textwidth]{higgs_to_taus_vh/plots/combined/wh_vs_zh_sbweight.pdf}
 \end{center}
 \caption{
 Distribution of the decimal logarithm of the ratio between the expected signal and the 
 sum of expected signal and expected background in each bin of the mass distributions 
 used to extract the results, in all signal regions. The background contributions are 
 separated based on the final state channels, $\PW\PH$ versus $\PZ\PH$. The inset 
 shows the corresponding difference between the 
 observed data and expected background distributions divided by the background expectation, 
 as well as the signal expectation divided by the background expectation.
 }
 \label{fig:sb}
\end{figure}



The best fit signal
strength from this dedicated $\PW\PH$ and $\PZ\PH$ associated production analysis is 
$\mu = 2.54 ^{+1.35} _{-1.26}$ ($\mu = 1.00 ^{+1.08} _{-0.97}$ expected) 
for a significance of 2.3 standard deviations (1.0 expected).

To fully exploit the $\PH \to \Pgt\Pgt$ data in the 2016 CMS dataset, the results
of this dedicated $\PW\PH$ and $\PZ\PH$ associated production analysis are combined with the prior
$\PH \to \Pgt\Pgt$ analysis~\cite{HIG-16-043}, which targeted the Gluon Fusion and
VBF Higgs boson production processes. By combining these two analyses of
2016 CMS data, we have signal regions targeted each of the four leading Higgs 
boson production processes. The resulting signal strenghts, significance and, Higgs
boson couplings can be probed with greater precision than either analysis alone.
The best fit signal strength for from the combination is $\mu = 1.24 ^{+0.29} _{-0.27}$.
For reference, the best fit signal strength for the $ggH$ and VBF targeted analysis
is $\mu = 1.09 ^{+0.27} _{-0.26}$.
The signal strength from the combination can be decomposed by Higgs boson production 
process, Fig.~\ref{fig:mu_higgs_processes}. The combination leads to an 
observed significance of 5.5 standard deviations (4.8 expected). 

\begin{figure}[!ht]
 \begin{center}
  \includegraphics[width=0.45\textwidth]{higgs_to_taus_vh/plots/combined/mu_higgs_procs.pdf}
 \end{center}
 \caption{
 Best fit signal strength per $\PH$ production process, for $\mH = 125.09$\GeV.
 A combination of the $\PW\PH$ and $\PZ\PH$ analysis detailed in this paper
 with the $\htt$ CMS analysis~\cite{HIG-16-043} is used to
 fully exploit the $\htt$ data at CMS and constrain the
 Gluon Fusion and Vector Boson Scattering $\PH$ production processes as fully
 The constraints from the combined global fit are used to extract each of the 
 individual best fit signal strengths. The combined best fit signal strength 
 is $\mu = 1.24 ^{+0.29} _{-0.27}$.
 as is possible.
 }
 \label{fig:mu_higgs_processes}
\end{figure}

This same combination of this dedicated $\PW\PH$ and $\PZ\PH$ analysis with the prior
$\PH \to \Pgt\Pgt$ analysis~\cite{HIG-16-043}, can place the tightest
$\PH \to \Pgt\Pgt$ in the $(\kappa_\text{V}$,$\kappa_\text{f})$ parameter space.
A likelihood scan is performed for $\mH=125.09\GeV$ in the ($\kappa_\text{V}$,$\kappa_\text{f}$) 
parameter space, where $\kappa_\text{V}$ and $\kappa_\text{f}$ quantify, respectively, 
the ratio between the measured and the SM value for the couplings of the Higgs boson to 
vector bosons and fermions, with the methods described in Ref.~\cite{Chatrchyan:2014nva}. 
For this scan only, Higgs boson decays to pairs of $\PW$ or $\PZ$ bosons, $\hww$ or $\hzz$,
 are considered as part of 
the signal. The $\ttbar\PH$ production process is still treated as background because
the MC sample we use is not split by Higgs boson decay mode. All nuisance 
parameters are profiled for each point of the scan. As shown in 
Fig.~\ref{fig:kVkf}, the observed likelihood contour is consistent with the SM expectation 
of $\kappa_\text{V}$ and $\kappa_\text{f}$ equal to unity.

\begin{figure}[!ht]
 \begin{center}
  \includegraphics[width=0.45\textwidth]{higgs_to_taus_vh/plots/combined/kFkV_HIG-18-007_plus_HIG-16-043.pdf}
 \end{center}
 \caption{Scan of the negative 
 log-likelihood difference as a function of $\kappa_V$ and $\kappa_f$, for 
 $\mH = 125.09$\GeV.  All nuisance parameters are profiled for each point. 
 This scan is a combination of the $\PW\PH$ and $\PZ\PH$ targeted analysis detailed in this paper
 with the $\htt$ CMS analysis~\cite{HIG-16-043}.
 For this scan, the included $\hww$ and $\hzz$ production processes 
 are treated as signal processes.
 }
 \label{fig:kVkf}
\end{figure}



\section{Summary}
A search for the standard model Higgs boson based on data collected in proton-proton collisions by the
CMS detector in 2016 at a center-of-mass energy of 13\TeV focusing on the
two $\PW\PH$ and $\PZ\PH$ associated production processes has been presented. Event
categories have been split into three lepton final states targeting $\PW\PH$ production
and four lepton final states targeting $\PZ\PH$ production. The results are extracted
via maximum likelihood fits using the visible di-$\Pgt$ mass for the $\PW\PH$
channels and full di-$\Pgt$ mass for the $\PZ\PH$ channels. 
%Observed limits of 4.7 
%(expected 2.0) are placed on the Higgs boson associated production processes 
%times the SM prediction for a Higgs boson mass of 125.09\GeV. 
The best fit signal
strength is $\mu = 2.54 ^{+1.35} _{-1.26}$ ($\mu = 1.00 ^{+1.08} _{-0.97}$ expected) 
for a significance of 2.3 standard deviations (1.0 expected).

Combining this analysis with the previous 13\TeV $ggH$ and VFB targeted $\htt$ 
analysis~\cite{HIG-16-043}, we place the tightest constraints
on the $\htt$ process. 
The best fit signal strength is $\mu = 1.24 ^{+0.29} _{-0.27}$ leading to an
observed significance of 5.5 standard deviations (4.8 expected). 
The combination leads to a significant increase in constraint for the coupling
of the Higgs boson to vector bosons, the coupling to fermions is not greatly
affected. The resulting measured couplings are consistent with SM predictions
within one standard deviation.

\clearpage


The signal extraction distributions are shown for all eight $\PZ\PH$ final states,
Figs.~\ref{fig:zh_all_eight1} and ~\ref{fig:zh_all_eight2}.

\begin{figure}[h!]
 \begin{center}
  \includegraphics[width=0.45\textwidth]{higgs_to_taus_vh/plots/zh/eeet_postfit.pdf}
  \includegraphics[width=0.45\textwidth]{higgs_to_taus_vh/plots/zh/emmt_postfit.pdf}
  \includegraphics[width=0.45\textwidth]{higgs_to_taus_vh/plots/zh/eemt_postfit.pdf}
  \includegraphics[width=0.45\textwidth]{higgs_to_taus_vh/plots/zh/mmmt_postfit.pdf}
 \end{center}
 \caption{The postfit $\mtt$ distributions used to extract the signal shown
  for the (top left) $\Pe\Pe\Pe\tauh$, (top right) $\Pgm\Pgm\Pe\tauh$, 
  (bottom left) $\Pe\Pe\Pgm\tauh$, and (bottom right) $\Pgm\Pgm\Pgm\tauh$
  final states. The final state is listed in the
  top left corner of each distribution.
  The distributions show full uncertainties.
  The $\PW\PH$ and $\PZ\PH$ signal are shown as 5x larger than their best-fit
  signal strength value of $2.5 \times$ SM.
 }
 \label{fig:zh_all_eight1}
\end{figure}

\begin{figure}[h!]
 \begin{center}
  \includegraphics[width=0.45\textwidth]{higgs_to_taus_vh/plots/zh/eett_postfit.pdf}
  \includegraphics[width=0.45\textwidth]{higgs_to_taus_vh/plots/zh/mmtt_postfit.pdf}
  \includegraphics[width=0.45\textwidth]{higgs_to_taus_vh/plots/zh/eeem_postfit.pdf}
  \includegraphics[width=0.45\textwidth]{higgs_to_taus_vh/plots/zh/emmm_postfit.pdf}
 \end{center}
 \caption{The postfit $\mtt$ distributions used to extract the signal shown
  for the (top left) $\Pe\Pe\tauh\tauh$, (top right) $\Pgm\Pgm\tauh\tauh$, 
  (bottom left) $\Pe\Pe\Pe\Pgm$, and (bottom right) $\Pgm\Pgm\Pe\Pgm$
  final states. The final state is listed in the
  top left corner of each distribution.
  The distributions show full uncertainties.
  The $\PW\PH$ and $\PZ\PH$ signal are shown as 5x larger than their best-fit
  signal strength value of $2.5 \times$ SM.
 }
 \label{fig:zh_all_eight2}
\end{figure}


