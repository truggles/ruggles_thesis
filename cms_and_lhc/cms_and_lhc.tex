\chapter{The CMS Experiment and the CERN LHC}

\section{The LHC}

\subsection{LHC Pre-Acceleration}

\subsection{LHC Acceleration}

\section{The CMS Experiment}

The central feature of the CMS apparatus is a superconducting solenoid of 6 meters internal diameter, providing a magnetic field of 3.8 Tesla. Within the solenoid volume, there are a silicon pixel and strip tracker, a lead tungstate crystal electromagnetic calorimeter (ECAL), and a brass and scintillator hadron calorimeter (HCAL), each composed of a barrel and two endcap sections. Forward calorimeters extend the pseudorapidity coverage provided by the barrel and endcap detectors. Muons are detected in gas-ionization chambers embedded in the steel flux-return yoke outside the solenoid.

Events of interest are selected using a two-tiered trigger system~\cite{Khachatryan:2016bia}. The first level (L1), composed of custom hardware processors, uses information from the calorimeters and muon detectors to select events at a rate of around 100 kHz within a time interval of less than 4 microseconds. The second level, known as the high-level trigger (HLT), consists of a farm of processors running a version of the full event reconstruction software optimized for fast processing, and reduces the event rate to about 1 kHz before data storage.

Significant upgrades of the L1 trigger during the first long shutdown of the LHC have benefitted this analysis, especially in the $\tauh\tauh$ channel. These upgrades improved the $\tauh$ identification at L1 by giving more flexibility to object isolation, allowing new techniques to suppress the contribution from additional $\Pp\Pp$ interactions per bunch
crossing, and to reconstruct the L1 $\tauh$ object in a fiducial region that matches more closely that of a true hadronic $\Pgt$ decay. The flexibility is achieved by employing high bandwidth optical links for data communication and large field-programmable gate arrays (FPGAs) for data processing.

A more detailed description of the CMS detector, together with a definition of the coordinate system used and the relevant kinematic variables, can be found in Ref.~\cite{Chatrchyan:2008zzk}.

\subsection{Geometry}

\subsection{Magnet}
\subsection{Inner Tracking System}
\subsubsection{Pixels}
\subsubsection{Strips}
\subsection{Electromagnetic Calorimeter}
\subsection{Hadronic Calorimeter}
\subsection{Muon System}
\subsubsection{Drift Tubes}
\subsubsection{Cathode Strip Chambers}
\subsubsection{Resistive Plate Chambers}
\subsection{Trigger and Data Acquisition}
\subsubsection{Level-1 Trigger}
\subsubsection{Aside for CaloL1 Duties and Online SW}
\subsubsection{HLT}
\subsubsection{Aside for 2018 Tau Trigger Updates}
\subsubsection{Aside for Phase-2 L1EG Discussion}
