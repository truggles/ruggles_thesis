\chapter{Object Reconstruction and Selection}

The reconstruction of observed and simulated events relies on the particle-flow (PF) algorithm~\cite{Sirunyan:2017ulk},
which combines the information from the CMS subdetectors to identify
and reconstruct the particles emerging from $\Pp\Pp$ collisions:
charged hadrons, neutral hadrons, photons, muons, and electrons.
Combinations of these PF objects are used to reconstruct
higher-level objects such as jets, $\tauh$ candidates, or
missing transverse energy.
The reconstructed vertex with the largest value of summed physics-object $\pt^2$ is taken to be the primary $\Pp\Pp$ interaction vertex. The physics objects are the objects constructed by a jet finding algorithm~\cite{Cacciari:2008gp,Cacciari:2011ma} applied to all charged tracks associated with the vertex, including tracks from lepton candidates, and the corresponding associated missing transverse energy.

Muons are identified with requirements on the quality of
the track reconstruction and on the number of measurements in the
tracker and the muon systems~\cite{Chatrchyan:2012xi}.
Electrons are identified with a multivariate discriminant
combining several quantities describing the track quality,
the shape of the energy deposits in the ECAL,
and the compatibility of the measurements from the tracker and the
ECAL~\cite{Khachatryan:2015hwa}.
To reject non-prompt or misidentified leptons, a relative lepton isolation is defined as:
\begin{equation}
I^{\ell} \equiv \frac{\sum_{charged}  \PT + \max\left( 0, \sum_{neutral}  \PT
                                         - \frac{1}{2} \sum_{charged, PU} \PT  \right )}{\PT^{\ell}}.
\label{eq:reconstruction_isolation}
\end{equation}
In this expression, $\sum_charged  \PT$ is the scalar sum of the
transverse energy of the charged particles originating from
the primary vertex and located in a cone of size
$\Delta R = \sqrt{\smash[b]{(\Delta \eta)^2 + (\Delta \phi)^2}} = 0.4$\,(0.3)
centered on the muon (electron) direction. The sum
$\sum_{neutral}  \PT$  represents
a similar quantity for neutral particles.
The contribution of photons and neutral hadrons originating from pileup vertices is estimated from the scalar sum of the transverse
energy of charged hadrons in the cone originating from pileup vertices,
$\sum_{charged, PU} \PT$. This sum is multiplied by a factor of
$1/2$, which corresponds approximately to the ratio of neutral to charged
hadron production in the hadronization process
of inelastic $\Pp\Pp$ collisions, as estimated from simulation.
The expression $\PT^{\ell}$ stands for the $\pt$ of the lepton. Isolation requirements used in this analysis, based on $I^{\ell}$, are listed in Table~\ref{tab:inclusive_selection}.

Jets are reconstructed with an anti-\kt clustering algorithm implemented
in the \FASTJET library~\cite{Cacciari:2011ma, Cacciari:fastjet2}.
It is based on the clustering of neutral and charged PF candidates within a distance parameter of 0.4. Charged PF candidates
not associated with the primary vertex of the interaction
are not considered when building jets. An offset correction is applied to jet energies to take into account the contribution from additional $\Pp\Pp$ interactions within the same or nearby bunch crossings. The energy of a jet is calibrated based on simulation and
data through correction factors~\cite{CMS-JME-10-011}.
In this analysis, jets are required to
have $\pt$ greater than 30\GeV and $\abs{\eta}$ less than 4.7, and
are separated from the selected leptons by a $\Delta R$ of at least 0.5.
The combined secondary vertex (CSV) algorithm is used to identify jets that are likely to originate from a b quark (``b jets"). The algorithm exploits the track-based lifetime information together with the secondary vertices associated with the jet to provide a likelihood ratio discriminator for the b jet identification. A set of $\pt$-dependent correction
factors are applied to simulated events to account for differences in the b tagging efficiency
between data and simulation. The working point chosen in this analysis gives an efficiency for real b jets of about 70\%, and for about 1\% of light flavor or quark jets being misidentified.

Hadronically decaying $\Pgt$ leptons
are reconstructed with the hadron-plus-strips (HPS)
algorithm~\cite{Khachatryan:2015dfa, CMS-PAS-TAU-16-002}, which is
seeded with anti-\kt jets.
The HPS algorithm reconstructs $\tauh$ candidates on the basis of the
number of tracks and of the number of ECAL strips in the $\eta$-$\phi$ plane with energy deposits, in the 1-prong,
1-prong + $\PGpz$(s), and 3-prong decay modes. A
multivariate (MVA) discriminator~\cite{Hocker:2007ht}, including isolation
and lifetime information, is used to reduce the rate for  quark- and gluon-initiated jets
to be identified as $\tauh$ candidates. The working point used in this analysis
has an efficiency of about 60\% for genuine $\tauh$,
with about 1\% misidentification rate for quark- and gluon-initiated jets, for a $\pt$ range typical of $\tauh$ originating from a $\PZ$ boson.
Electrons and muons misidentified as $\tauh$ candidates are suppressed using dedicated criteria
based on the consistency between the measurements in the tracker, the calorimeters, and the muon detectors~\cite{Khachatryan:2015dfa, CMS-PAS-TAU-16-002}.
The working points of these discriminators depend on the
decay channel studied.
The $\tauh$ energy scale in simulation is corrected per decay mode, on the basis of a measurement in $\PZ\to\Pgt\Pgt$ events. The rate and the
energy scale of electrons and muons misidentified as $\tauh$ candidates are also corrected in simulation, on the basis of a tag-and-probe measurement~\cite{CMS:2011aa} in $\PZ\to\ell\ell$ events.

All particles reconstructed in the event are used to determine the missing transverse energy,
\etvecmiss. The missing transverse momentum is defined as the negative vectorial sum of the transverse energy of
all PF candidates~\cite{Khachatryan:2014gga}. It is adjusted for the effect of jet energy corrections.
Corrections to the $\etvecmiss$ are applied to reduce the mismodeling of the simulated
$\PZ$+jets, $\PW$+jets and Higgs boson samples.
The corrections are applied to the simulated events on the basis of the vectorial difference
of the measured missing transverse energy and total transverse energy of neutrinos
originating from the decay of the $\PZ$, $\PW$, or Higgs boson. Their average effect is the reduction of the $\etvecmiss$ obtained from simulation by a few \GeV.

The visible mass of the $\Pgt\Pgt$ system, $\mvis$, can be used to separate
the $\PH\to \Pgt \Pgt$ signal events
from the large contribution of irreducible $\PZ \to \Pgt \Pgt$ events.
However, the neutrinos from the $\Pgt$ lepton decays carry a large fraction of
the $\Pgt$ lepton energy and reduce the discriminating power of this variable.
The \textsc{svfit} algorithm combines the \etvecmiss with the four-vectors of both $\Pgt$ candidates
to calculate a more accurate estimate of the mass of the parent boson, denoted as $\mtt$. The resolution of $\mtt$ is between 15 and 20\% depending on the $\Pgt\Pgt$ final state.
A detailed description of the algorithm can be found
in Ref.~\cite{Bianchini:2014vza}. Both variables are used in the analysis, as detailed in Section.~\ref{sec:categories}, and $\mvis$ is preferred over $\mtt$ when the background from $\PZ \to \ell\ell$ events is large.

\section{Track and Primary Vertex Reconstruction}
\section{Particle Flow Reconstruction}
\subsection{PF Candidates}
\subsubsection{Muons}
\subsubsection{Electrons and Prompt Photons}
\subsubsection{Charged and Neutral Hadrons}
\subsection{Jets}
\subsection{Taus}
\subsection{Missing Transverse Energy}
\pagebreak
\section{Object Identification and Selection}
\subsection{Muons}
\subsection{Electrons}
\subsection{Taus}
\subsection{b-jet ID and Secondary Vertex}



