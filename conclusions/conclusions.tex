\chapter{Conclusions}
\label{sec:conclusion}

In this thesis I have presented two analyses of the standard model (SM) Higgs boson decaying 
to a $\Pgt$ lepton pair using data collected by the CMS experiment in 2016.
The analyses target the four leading Higgs boson production processes at the LHC:
gluon fusion, vector boson fusion, and $\PW$ and $\PZ$ associated production.
Specialized categories are used to target the unique event topologies and 
characteristics of each of these four production processes to maximize our
sensitivity to the Higgs boson. 

These analyses studying the $\htt$ process provide a 5.5 standard deviation (4.8 expected)
observation of the Higgs boson decaying to fermions at 13\TeV center-of-mass energy. 
This is an important benchmark for the CMS experiment and the high energy
particle physics community as a whole.
The best fit signal strength for the $\htt$ process is measured to be 
$\mu = 1.24 ^{+0.29} _{-0.27}$, consistent with SM predictions.
The Higgs boson couplings to fermions and vector bosons are measured and are found to be
consistent with SM predictions within 1$\sigma$.
%All results from these analyses and their combination are consistent
%with the SM prediction for the $\htt$ process. At the current level of precision,
%the Higgs boson branching ratio to a $\Pgt$ lepton pair and the
%Higgs boson couplings to fermions and vector bosons show no significant
%deviations. 
%These results contribute to the vast body of previous experimental results
%supporting the SM and the 125\GeV Higgs boson.

There is still room for improvement in these measurements, which will come with
including the additional 13\TeV data from the remainder of the LHC Run-II. 
Techniques have been developed to probe the Higgs boson properties as a function
of event topology in unique phase spaces defined at the generator 
level using characteristics such as the number of jets
in the event, the $\pt$ of the Higgs boson, the $\mjj$ in vector boson
fusion type events, and the $\pt$ of the
vector boson in associated production events. In the simplified template
cross section (STXS) method, the decomposition has been performed to focus on 
characteristics of the production topologies which are sensitive to higher
order theoretical corrections as well as possible physics beyond the standard
model~\cite{Tackmann:2138079}. Using only the 2016 dataset, we are too
statistically limited to take full advantage of the STXS method. However, with the
full Run-II dataset, the $\htt$ channel will be able to probe details of the
Higgs boson $\pt$ spectrum and more.

The vector boson fusion and associated production processes with $\htt$ are 
powerful mechanisms for studying the couplings of the Higgs boson to vector bosons
(HVV couplings). This is shown in Figure~\ref{fig:cmb_kFkV} under the assumption
that the HVV and fermionic couplings behave according to the SM Lagrangian. 
The results presented here are consistent with the SM.

Instead of measuring the consistency of the results with the SM, we can make alternative
physics assumptions and test their consistency against data.
Ongoing analyses with 2016 data are studying possible
anomalous HVV couplings from additional coupling terms introduced into the SM Lagrangian. 
These studies are expected to provide the tightest constraints at the 2$\sigma$ 
level on possible anomalous HVV couplings.

The impressive agreement of these Higgs boson results and other previous results
with SM predictions, and the lack of any observed physics signatures beyond
the SM, point to the need to measure the details of the Higgs boson to increasing
accuracy. The Higgs boson is the most recently discovered fundamental particle and occupies a very
unique position in the SM compared to the other particles. For these reasons,
the Higgs boson should continue to be used as a tool to probe the details of the SM and beyond.




