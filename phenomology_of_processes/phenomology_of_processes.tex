\chapter{Phenomology of Processes}
\label{sec:pheno}

Dasu -
The pheno chapter need not start from the Dirac equation and build up. It should have crisp intro to Higgs phenomenology starting from that portion of the Lagrangian. You don’t need all the myriad details of SM like the quark mixing matrices, etc.

\section{Higgs Yukawa Couplings}

\section{Higgs Production}

\subsection{Gluon Fusion}

\subsection{Vector Boson Fusion}

\subsection{Associated Production}

\section{Higgs Decays}

\subsection{Higgs to $\tau\tau$ Decay Process}

CROSS SECTIONS
The various production cross sections and branching fractions for the SM Higgs 
boson production, and their corresponding uncertainties are taken from 
References.~\cite{deFlorian:2016spz,Denner:2011mq,Ball:2011mu} and references therein.



\subsection{Electroweak Symmetry Breaking}

electroweak symmetry breaking is achieved via the Brout--Englert--Higgs
mechanism~\cite{Englert:1964et,Higgs:1964ia,Higgs:1964pj,Guralnik:1964eu,Higgs:1966ev,Kibble:1967sv},
leading, in its minimal version, to the prediction of the existence of one physical neutral scalar particle,
commonly known as the Higgs boson ($\PH$).



\subsubsection{W/Z Higgs Associated Production}

\subsection{Cross Sections and Decay Rates}

\subsection{QCD and Proton Structure}



To establish the mass generation mechanism for fermions,
 it is necessary to probe the direct coupling of
the Higgs boson to such particles.
The most promising decay channel is $\Pgt^+\Pgt^-$,
because of the large event rate expected in the SM compared to the $\Pgm^+\Pgm^-$ decay channel ($\mathcal{B}(\PH\to\Pgt^+\Pgt^-)=6.3$\% for a mass of 125.09\GeV), and of the smaller contribution from background events
with respect to the $\bbbar$ decay channel.

